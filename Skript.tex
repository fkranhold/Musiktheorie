\documentclass[ngerman]{scrartcl}
%% Serifen in Überschriften
\KOMAoptions{paper=b5,numbers=enddot,DIV=11,headings=standardclasses}

%% Encoding
\usepackage[utf8]{inputenc}
\usepackage[T1]{fontenc}

%% Sprache
\usepackage{babel}
\usepackage[cleanlook]{isodate}

%% Schriftpakete
\usepackage{lmodern}
\usepackage[osf,sc]{mathpazo}
\usepackage{classico}
\usepackage[scaled=.85]{beramono}

%% Palatino braucht mehr Platz
\usepackage{setspace}
\setstretch{1.07}
\renewcommand{\arraystretch}{1.07} % Für tabular- und array-Umgebungen

%% Mikrotypographie!
\usepackage{ellipsis}
\usepackage[babel,tracking=true]{microtype}
\UseMicrotypeSet[tracking]{smallcaps}
\SetTracking{encoding=*,shape=sc}{30}

%% Akronyme
\newcommand{\acr}[1]{\textls[40]{\textsc{\MakeLowercase{#1}}}}

%% Eigene Listing styles mit kurzer Definition
\usepackage[shortlabels]{enumitem}

%% Captions for figures and tables
\setkomafont{captionlabel}{\footnotesize\bfseries}
\setkomafont{caption}     {\footnotesize}
\setcapwidth{.9\textwidth}
\renewcommand*{\captionformat}{.\ }
\setcapindent{0em}
\addtokomafont{caption}{\setstretch{1.07}}

%% Fancy tables
\usepackage{booktabs}

%% Mehrspaltinge Aufzählungen
\usepackage{multicol}

%% Bibliographie
\usepackage{biblatex}
\bibliography{bibliography.bib}

%% Footnote-Design
\deffootnote{1.5em}{1.5em}{\thefootnotemark.\ }
\addtokomafont{footnote}{\setstretch{1.07}}

%% Todo
\usepackage[size=tiny]{todonotes}

%% Grundlegende mathematische Symbole
\usepackage{amssymb,amsmath} % Standard
\usepackage{mathtools}       % \coloneqq, \eqqcolon

%% TikZ and TikZ-CD
\usepackage{tikz-cd}
\usepackage{tikz}
\usetikzlibrary{arrows,calc}

%% Mini-Indizes (\scaleto{…}{3pt})
\usepackage{scalerel}

%% Bold
\newcommand{\bm} [1]{\mathbold{#1}}

%% TikZ: centerarc
 \def\centerarc[#1](#2)(#3:#4:#5)% Syntax: [draw options] (center) (initial angle:final angle:radius)
     { \draw[#1] ($(#2)+({#5*cos(#3)},{#5*sin(#3)})$) arc (#3:#4:#5); }

%% Musikalische Symbole
\usepackage{musicography}

%% Harmonische Funktionen
\newcommand{\mut}         {\text{t}}
\newcommand{\mus}         {\text{s}}
\newcommand{\mud}         {\text{d}}
\newcommand{\muT}         {\text{T}}
\newcommand{\muS}         {\text{S}}
\newcommand{\muD}         {\text{D}}
\newcommand{\mup}         {\text{p}}
\newcommand{\muP}         {\text{P}}
\newcommand{\mug}         {\text{g}}
\newcommand{\muG}         {\text{G}}
\newcommand{\muDD}        {\mbox{\muD\hspace{-5.5px}\raisebox{1.3px}{\muD}}}
\newcommand{\muDDZ}       {\mbox{\muD\hspace{-6.1px}\raisebox{1.3px}{\muD}}}
\newcommand{\muSS}        {\raisebox{1.5px}{\muS}\hspace{-4px}\muS}
\newcommand{\muss}        {\raisebox{1.5px}{\mus}\hspace{-3px}\mus}
\newcommand{\musDD}       {\muD\hspace{-6px}\raisebox{2px}{\sout{\muD}}}
\newcommand{\vermsept}    {\hspace{1mm}\raisebox{4px}{\mus}\hspace{-3px}\muD}
\newcommand{\sevsept}     {\hspace{1mm}\raisebox{3px}{\muS}\hspace{-5px}\muD}
\newcommand{\dvermsept}   {\hspace{1mm}\raisebox{4px}{\mut}\hspace{-1px}\muD\hspace{-6px}\raisebox{2px}{\muD}}

%% Hyperref and Cleveref
\usepackage{amsthm}
\usepackage[pdfusetitle,bookmarksnumbered=true]{hyperref}
\usepackage[noabbrev]{cleveref}

%% Aufgaben
\theoremstyle{definition}
\newtheorem{aufg}{Aufgabe}[section]

%\usepackage[dvipsnames]{xcolor}
%\newcommand{\rb}[1]{\textcolor{Maroon}{#1}}
\newcommand{\rb}[1]{#1}


\begin{document}

\subject {CdE-WinterAkademie 2025\,·\,26}
\author  {Florian Kranhold\and Charlotte Mertz}
\title   {Musiktheorie}
\subtitle{Skript}
\hypersetup{pdftitle={Musiktheorie: Skript}}

\maketitle

Dieses Kurzskript umfasst die Themen des geplanten Musiktheoriekurses auf der
ersten Hälfte der WinterAkademie 2025\,·\,26: Die ersten Abschnitte beschäftigen
sich mit Tönen und ihrer Notation, Intervallen und Tonarten, während die
hinteren Abschnitte Satzregeln sowie spannende Akkorde und Wendungen behandeln.

Wir folgen hierbei im Wesentlichen dem Standardwerk von Diether de la Motte
\cite{deLaMotte}, das wir den an einer detaillierteren Abhandlung interessierten
Leser:innen ans Herz legen möchten. Die Konzepte werden am Beispiel
vierstimmiger Chorsätze studiert, sind aber leicht auf andere Satzformen
übertragbar.

Ein Wort der Warnung: Dieses Skript ist stellenweise vereinfachend.  Es
versteckt einige spannende Theoriedebatten, um leichter verständlich zu
sein. Darüber hinaus wird die historische Entwicklung von Satztechniken fast
vollständig außer Acht gelassen (außer in \cref{None}). Wir behandeln
hauptsächlich die strengen Satzregeln der Musik Bachs, weil wir überzeugt sind,
dass sich mit ihnen auch die Musik jüngerer Epochen untersuchen und beschreiben
lässt.

\section{Töne und Skalen}

In diesem Abschnitt beschreiben wir die kleinsten Bausteine, mit denen wir
arbeiten: die \emph{Töne}. Für uns ist die Höhe eines Tons gegeben durch die
\emph{Frequenz}, mit der er schwingt. Auf diese Weise lässt sich leicht
definieren, was eine \emph{Oktave} ist. Die Möglichkeiten, eine Oktave zu
durchschreiten, lernen wir als \emph{Skalen} kennen.

\subsection{Töne und Noten}
Ein \emph{Ton} beschreibt eine physikalische Schwingung. Das menschliche Ohr
fasst etwa einen Frequenzbereich von 20\,Hz bis 16\,000\,Hz. In der Musik
interessieren uns Frequenz und Dauer eines Tons. Dies bezeichnen wir als
\emph{Tonhöhe} und \emph{Tonlänge}.  Ein Ton wird in Form einer \emph{Note}
notiert.

Grundlegende Regeln zur rhythmischen Notation (wie Notenwerte, Punktierung,
Pausen, Takte oder Haltebögen) sollten bekannt sein und werden hier nicht
besprochen. Wir verweisen auf
\href{https://de.wikipedia.org/wiki/Metrum_(Musik)}{\textsf{de.wikipedia.org/wiki/Metrum\_(Musik)}}.

\subsection{Oktave und Halbtonschritt}
Wir wollen Töne zueinander in Relation setzen:
\begin{itemize}
\item Der \emph{Abstand} zwischen zwei Tönen entspricht einem
  \emph{Verhältnis} zwischen zwei Frequenzen: Addition von \emph{Abständen}
  ist Multiplikation von \emph{Verhältnissen}.
\item Eine \emph{Oktave} ist ein Abstand zwischen zwei Tönen, der einer
  Verdopplung der Frequenz entspricht. Unterscheiden sich zwei Töne um
  eine Oktave, schwingt der höhere doppelt so schnell.
\item Wir unterteilen den Abstand der Oktave gleichmäßig in 12 kleinere
  Abstände, die \emph{Halbtonschritte} genannt werden. Ein Halbtonschritt
  entspricht also dem Frequenzverhältnis $\mkern-4mu\sqrt[12]{2}\kern.3px:\kern-.9px1$.  Für uns
  ist der Halbtonschritt der kleinste Abstand zwischen zwei Tönen.
\item Ein \emph{Intervall} ist die Beschreibung eines Tonabstands durch die
  Anzahl an Halbtonschritten (\acr{HTS}): Ein Intervall bestehend aus
  $n$\,\acr{HTS} entspricht folglich einem Frequenzverhältnis von
  $(\kern-2px\sqrt[12]{2})^n\kern-.6px:\kern-.6px1$.
\end{itemize}

Die oben beschriebene Berechnung ist nur eine von vielen historisch gewachsenen
Stimmungssystemen. Unsere Methode wird \emph{gleichstufig} genannt, weil bei ihr jeder
Halbtonschritt gleich groß ist. Andere Stimmungssysteme haben andere Vorzüge.

\subsection{Skala}
Ein Intervall, das aus zwei Halbtonschritten besteht, heißt
\emph{Ganztonschritt}. Allgemeiner heißt ein Intervall, das aus einem Halb-
oder Ganztonschritt besteht, \emph{Schritt}.  Möchte man die Oktave in sieben
Schritte zerlegen, so braucht man dazu genau fünf Ganz- und zwei
Halbtonschritte.

Eine solche Zerlegung heißt
\emph{(heptatonische) Skala} oder \emph{Kirchenskala}, wenn zwischen den
beiden Halbtonschritten zwei oder drei Ganztonschritte liegen. Wir nummerieren die
Töne von 1 bis 8 durch. Spielt man alle Fälle durch, erhält man untenstehende
Möglichkeiten der Positionierung von Halbtonschritten:
\begin{multicols}{2}
  \begin{itemize}[nosep]
  \item 1\,–\,2 und 4\,–\,5: \emph{Lokrisch}
  \item 1\,–\,2 und 5\,–\,6: \emph{Phrygisch}
  \item 2\,–\,3 und 5\,–\,6: \emph{Aeolisch}\,·\,\emph{Moll}
  \item 2\,–\,3 und 6\,–\,7: \emph{Dorisch}
  \item 3\,–\,4 und 6\,–\,7: \emph{Mixolydisch}
  \item 3\,–\,4 und 7\,–\,8: \emph{Ionisch}\,·\,\emph{Dur}
  \item 4\,–\,5 und 7\,–\,8: \emph{Lydisch}
  \item[\vspace{\fill}]
  \end{itemize}
\end{multicols}

\section{Namen für Töne}

Wir wollen konkreten Tönen Namen geben, und zwar so, dass Töne, die sich nur um
Oktaven voneinander unterscheiden, ähnlich heißen.

\subsection{Eingestrichene Oktave}
Einen Ton der Frequenz 440\,Hz bezeichnen wir als
\emph{Kammerton}.\footnote{Auch hier gibt es andere, historisch gewachsene
  Konventionen. So wird zum Beispiel für „alte Musik“ oft eine
  Kammertonfrequenz von 415\,Hz verwendet.} Den Ton neun Halbtonschritte
darunter, mit der Frequenz
\[\mathopen{}\big(\mkern-2mu\sqrt[12]{2}\kern.5px\big)\mathclose{}^{-9}%
  \mkern-2mu\cdot\hspace*{1px} 440\,\mathrm{Hz} \,\approx\, 262\,\mathrm{Hz},\]
nennen wir \emph{eingestrichenes c} (notiere c’). Wir setzen nun auf das
eingestrichene c eine Durskala und benennen die Töne der Skala wie in
\cref{tab:eOk}. Auf diese Weise wird der Kammerton zum
eingestrichenen a. Zwischen e und f sowie zwischen h und c liegen jeweils
Halbtonschritte, zwischen den übrigen Tönen Ganztonschritte.


\begin{table}[h]
  \centering
  \begin{tabular}{cccccccc}
    \toprule
    1 & 2 & 3 & 4 & 5 & 6 & 7 & (8)\\
    \midrule
    c’ & d’ & e’ & f’ & g’ & a’ & h’ & (c”)\\
    \bottomrule
  \end{tabular}
  \caption{Die Töne der eingestrichenen Oktave}\label{tab:eOk}
\end{table}

\subsection{Oktavraum}
Töne, die sich von den oben genannten um Oktaven unterscheiden, werden mit dem
gleichen Buchstaben (etwa: c) benannt. Die Dekoration (etwa: eingestrichen)
richtet sich nach dem sich von c bis h erstreckenden \emph{Oktavraum}, in dem
sie sich befinden, siehe \cref{tab:OkR}.


\begin{table}[h]
  \centering
  \begin{tabular}{lrrl}
    \toprule
    Name                   &     Beginn &       Ende & Erster Ton\\
    \midrule
    Subkontraoktave        &     16\,Hz &     32\,Hz & C„\\
    Kontraoktave           &     33\,Hz &     62\,Hz & C,\\
    Große Oktave           &     65\,Hz &    123\,Hz & C\\
    Kleine Oktave          &    130\,Hz &    246\,Hz & c\\
    Eingestrichene Oktave  &    262\,Hz &    493\,Hz & c’\\
    Zweigestrichene Oktave &    523\,Hz &    987\,Hz & c”\\
    Dreigestrichene Oktave & 1\,046\,Hz & 1\,975\,Hz & c”\kern-.8px’\\
    Viergestrichene Oktave & 2\,093\,Hz & 3\,951\,Hz & c”\kern-.8pt”\\
    \bottomrule
  \end{tabular}
  \caption{Die verschiedenen Oktavräume und ihre zugehörigen
    Frequenzbereiche}\label{tab:OkR}
\end{table}

\subsection{Alteration}
Unsere bisherigen Töne können um einen oder sogar zwei Halbtöne erhöht
oder erniedrigt werden: dies nennt sich \emph{Alteration}.
Die Namen der Töne werden dann wie folgt verändert:
\begin{itemize}
\item Für jede Erhöhung um 1\,\acr{HTS} wird jeweils ein "`-is"'
  angehängt.\vspace*{-3px}
\item Für jede Erniedrigung um 1\,\acr{HTS} wird jeweils ein "`-(e)s"'
  angehängt. (Der Bindevokal "`e"' entfällt, wenn der Name ein Vokal ist.)
\end{itemize}
Ausnahme bildet das h: Hier heißt die einfache Erniedrigung "`b"' statt
„hes“. Die zweifache Erniedrigung heißt wieder regelkonform "`heses"'.

Für einen gegebenen Tonnamen (etwa: fisis) heißt seine unalterierte Form (in
diesem Beispiel: f) \emph{Ausgangston}.

\subsection{Klaviatur}
Es ergibt sich das vertraue Bild der \emph{Klaviatur}, siehe
\cref{fig:klaviatur}, wobei die schwarzen Tasten diejenigen sind, die nur durch
Alteration erreicht werden können.

\begin{figure}[h]
  \centering
  \begin{tikzpicture}[xscale=1.2]
  \draw (0,-1) rectangle (1,3);
  \draw (1,-1) rectangle (2,3);
  \draw (2,-1) rectangle (3,3);
  \draw (3,-1) rectangle (4,3);
  \draw (4,-1) rectangle (5,3);
  \draw (5,-1) rectangle (6,3);
  \draw (6,-1) rectangle (7,3);
  \draw (7,-1) rectangle (8,3);
  \filldraw[fill=black] (0.7,0.8) rectangle (1.3,3);
  \filldraw[fill=black] (1.7,0.8) rectangle (2.3,3);
  \filldraw[fill=black] (3.7,0.8) rectangle (4.3,3);
  \filldraw[fill=black] (4.7,0.8) rectangle (5.3,3);
  \filldraw[fill=black] (5.7,0.8) rectangle (6.3,3);
  \node[text depth=.25ex, text height = 1.5ex] at (0.5,0.3) {\small c};
  \node[text depth=.25ex, text height = 1.5ex] at (1.5,0.3) {\small d};
  \node[text depth=.25ex, text height = 1.5ex] at (2.5,0.3) {\small e};
  \node[text depth=.25ex, text height = 1.5ex] at (3.5,0.3) {\small f};
  \node[text depth=.25ex, text height = 1.5ex] at (4.5,0.3) {\small g};
  \node[text depth=.25ex, text height = 1.5ex] at (5.5,0.3) {\small a};
  \node[text depth=.25ex, text height = 1.5ex] at (6.5,0.3) {\small h};
  \node[text depth=.25ex, text height = 1.5ex] at (7.5,0.3) {\small c};
  \node[text depth=.25ex, text height = 1.5ex] at (0.5,-0.15) {\small his};
  \node[text depth=.25ex, text height = 1.5ex] at (1.5,-0.15) {\small cisis};
  \node[text depth=.25ex, text height = 1.5ex] at (2.5,-0.15) {\small disis};
  \node[text depth=.25ex, text height = 1.5ex] at (3.5,-0.15) {\small eis};
  \node[text depth=.25ex, text height = 1.5ex] at (4.5,-0.15) {\small fisis};
  \node[text depth=.25ex, text height = 1.5ex] at (5.5,-0.15) {\small gisis};
  \node[text depth=.25ex, text height = 1.5ex] at (6.5,-0.15) {\small aisis};
  \node[text depth=.25ex, text height = 1.5ex] at (7.5,-0.15) {\small his};
  \node[text depth=.25ex, text height = 1.5ex] at (0.5,-0.6) {\small deses};
  \node[text depth=.25ex, text height = 1.5ex] at (1.5,-0.6) {\small eses};
  \node[text depth=.25ex, text height = 1.5ex] at (2.5,-0.6) {\small fes};
  \node[text depth=.25ex, text height = 1.5ex] at (3.5,-0.6) {\small geses};
  \node[text depth=.25ex, text height = 1.5ex] at (4.5,-0.6) {\small asas};
  \node[text depth=.25ex, text height = 1.5ex] at (5.5,-0.6) {\small heses};
  \node[text depth=.25ex, text height = 1.5ex] at (6.5,-0.6) {\small ces};
  \node[text depth=.25ex, text height = 1.5ex] at (7.5,-0.6) {\small deses};
  \node[text depth=.25ex, text height = 1.5ex,scale=.9] at (1,2) {\scriptsize\textcolor{white}{cis}};
  \node[text depth=.25ex, text height = 1.5ex,scale=.9] at (1,1.3) {\scriptsize\textcolor{white}{hisis}};
  \node[text depth=.25ex, text height = 1.5ex,scale=.9] at (2,2) {\scriptsize\textcolor{white}{dis}};
  \node[text depth=.25ex, text height = 1.5ex,scale=.9] at (2,1.3) {\scriptsize\textcolor{white}{feses}};
  \node[text depth=.25ex, text height = 1.5ex,scale=.9] at (4,2) {\scriptsize\textcolor{white}{fis}};
  \node[text depth=.25ex, text height = 1.5ex,scale=.9] at (4,1.3) {\scriptsize\textcolor{white}{eisis}};
  \node[text depth=.25ex, text height = 1.5ex,scale=.9] at (5,2) {\scriptsize\textcolor{white}{gis}};
  \node[text depth=.25ex, text height = 1.5ex,scale=.9] at (6,2) {\scriptsize\textcolor{white}{ais}};
  \node[text depth=.25ex, text height = 1.5ex,scale=.9] at (1,1.65) {\scriptsize\textcolor{white}{des}};
  \node[text depth=.25ex, text height = 1.5ex,scale=.9] at (2,1.65) {\scriptsize\textcolor{white}{es}};
  \node[text depth=.25ex, text height = 1.5ex,scale=.9] at (4,1.65) {\scriptsize\textcolor{white}{ges}};
  \node[text depth=.25ex, text height = 1.5ex,scale=.9] at (5,1.65) {\scriptsize\textcolor{white}{as}};
  \node[text depth=.25ex, text height = 1.5ex,scale=.9] at (6,1.65) {\scriptsize\textcolor{white}{b}};
  \node[text depth=.25ex, text height = 1.5ex,scale=.9] at (6,1.3) {\scriptsize\textcolor{white}{ceses}};
\end{tikzpicture}

  \caption{Die Tonnamen, angeordnet auf einer Klaviatur}\label{fig:klaviatur}
\end{figure}

\subsection{Enharmonik}
Es gibt nun für einen Ton mehrere Bezeichnungen (z.\,B. f und eis). Dieses
Phänomen heißt \emph{Enharmonik} und macht es notwendig, zwischen
\emph{klingendem Ton} und \emph{Tonnamen} zu unterscheiden.

Es wird sich zeigen, dass die korrekte enharmonische Bezeichnung eines Tones
wichtig ist, da ein Ton je nach harmonischem Kontext unterschiedlich
klingen kann (und zum Beispiel „nach oben“ oder „nach unten“ weitergedacht
wird).

\subsection{Enharmonik von Skalen}

Fixieren wir einen Startton (z.\,B. e’) und einen Modus (z.\,B. Moll), so sind
die klingenden Töne (bzw. die Klaviertasten) der dadurch definierten Skala
eindeutig festgelegt. Um sie auch enharmonisch eindeutig zu benennen, führen wir
die Konvention ein, dass alle Ausgangstöne verwendet werden müssen, und zwar in
aufsteigender Reihenfolge vorkommen müssen. Im Beispiel ergibt sich also als
einzige Möglichkeit einer Mollskala auf e’ die Tonfolge e’, fis‘, g’, a’, h’,
c”, d”, (e”).

\section{Notenschrift}

Um einen Ton zu notieren, platzieren wir eine \emph{Note} mit passendem Wert
(z.\,B.\ $\musHalf$, $\musQuarter$, $\musEighth$) in einem Notensystem mit fünf
Linien (und ggf. Hilfslinien). % Die vertikale Position des Notenkopfes
% hängt dabei vom Ausgangston (also der unalterierten Form) ab.

\subsection{Notenschlüssel}
Durch einen \emph{Notenschlüssel} wird eine der fünf Linien mit einem
spezifischen Ton verknüpft. Die Position der übrigen Töne ergibt sich dann wie
gewohnt durch Abzählen. Wir verwenden folgende Schlüssel:
\begin{itemize}
\item Der \textit{g-Schlüssel} gibt die Position des eingestrichenen g an. Je
  nach Position heißt er \emph{französischer Violinschlüssel} oder einfach nur
  \emph{Violinschlüssel}.
\item Der \textit{c-Schlüssel} gibt die Position des eingestrichenen c an. Je
  nach Position heißt er \emph{Sopran-}, \emph{Mezzosopran-}, \emph{Alt-} oder
  \emph{Tenorschlüssel}.
\item Der \textit{f-Schlüssel} gibt die Position des kleinen f an. Je nach
  Position heißt er \emph{Bariton-}, \emph{Bass-} oder \emph{Subbass"|schlüssel}.
\end{itemize}
Während in früheren Jahrhunderten zur Notation von Chormusik neben dem
Bass\-schlüssel sämtliche c-Schlüssel verwendet wurden (daher ihre Namen),
werden heutzutage vor allem Violin- und Bass"|schlüssel benutzt.

Außerdem hat sich, zum Beispiel für den Tenor, der \emph{oktavierte Violinschlüssel}
etabliert, der also die Position des \emph{kleinen} g umrundet.


\begin{figure}[h]
  \centering
  \raisebox{-5.02px}{\includegraphics{ly/s-clef-g}}\quad
  \raisebox{0px}{\includegraphics{ly/s-clef-c}}\quad
  \raisebox{0px}{\includegraphics{ly/s-clef-f}}
  \caption{Der g-Schlüssel als Violinschlüssel, der c-Schlüssel als Altschlüssel
    und der f-Schlüssel als Bass"|schlüssel. In allen Beispielen ist ein c’
    notiert.}\label{fig:Schl}
\end{figure}

\subsection{Vorzeichen}\label{Vorzeichen}
Um Alterationen zu notieren, werden den Noten \emph{Vorzeichen} vorangestellt,
siehe \cref{tab:VZ}.  Grundsätzlich gilt ein Vorzeichen einen Takt lang. Darüber
hinaus können Vorzeichen zu Beginn eines Stückes vermerkt sein; dann gelten sie
dauerhaft. In diesem Falle sprechen wir von \emph{Generalvorzeichen}.

Sollen Vorzeichen vorzeitig ihre Wirkung verlieren, so notiert man dies durch
ein \emph{Auflösungs\-zeichen} $\natural$. Dieses gilt ebenso einen Takt lang.


\begin{table}[h]
  \centering
  \begin{tabular}{rcl}
    \toprule
    Größe & Zeichen & Name\\
    \midrule
    \textminus\kern1pt 2\,\acr{HTS} & $\dflat$  & Doppel-B\\
    \textminus\kern1pt 1\,\acr{HTS} & $\flat$   & B \\
    +\kern1pt 1\,\acr{HTS}          & $\sharp$  & Kreuz\\
    +\kern1pt 2\,\acr{HTS}          & $\dsharp$ & Doppelkreuz\\
    \bottomrule
  \end{tabular}
  \caption{Die vier Vorzeichen}\label{tab:VZ}
\end{table}

\section{Intervalle}

Wie wir bereits im ersten Abschnitt besprochen haben, ist \emph{klingend} der
Abstand zwischen zwei Tönen vollständig beschrieben durch die Anzahl der
Halbtonschritte, die er umfasst (zumindest in unserem gleichstufigen
Stimmungssystem).

Allerdings tritt auch hier das Phänomen der Enharmonik zutage: Im musikalischen
Kontext ist es für unseren Höreindruck entscheidender, um wie viele Schritte in
einer Skala die Ausgangstöne auseinanderliegen. Es scheint also vonnöten, ein
genaueres Begriffssystem zu entwickeln.

\subsection{Grundintervalle}
In einer heptatonischen Skala können wir der Anzahl an Schritten Namen
zuordnen, die wir als \emph{Grundintervalle} bezeichnen. Hierfür zählen wir
den Abstand vom ersten zum betreffenden Ton der Skala wie in
\cref{tab:GrundI}.

Betrachten wir Intervalle wieder nur „bis auf Oktaven“, so benötigen
wir keine weiteren Grundintervalle. Ausnahme bildet die \emph{None}, die den
9. Ton bezeichnet. Wir werden sie später verwenden, weil wir den „Ton über der
Oktave“ betrachten wollen. Sie wird wie eine Sekunde behandelt, ist nur eine
Oktave größer.


\begin{table}[h]
  \centering
  \begin{tabular}{cl}
    \toprule
    Ton & Intervallname\\
    \midrule
    1. Ton & Prime (gleicher Ton)\\
    2. Ton & Sekunde\\
    3. Ton & Terz\\
    4. Ton & Quarte\\
    5. Ton & Quinte\\
    6. Ton & Sexte\\
    7. Ton & Septime\\
    8. Ton & Oktave\\
    \bottomrule
  \end{tabular}
  \caption{Die acht Grundintervalle}\label{tab:GrundI}
\end{table}

\subsection{Reine Intervalle}
Prime, Quarte, Quinte und Oktave heißen \emph{rein}, wenn sie exakt die Abstände
aus \cref{tab:rInt} bezeichnen.

Haben sie einen Halbtonschritt weniger, werden sie als \emph{vermindert} (v5
usf.)  bezeichnet; haben sie einen Halbtonschritt mehr, werden sie als
\emph{übermäßig} (ü5 usf.) bezeichnet. Diese Fälle sind Ausnahmen, die man
deutlich hört.

Eine ü4 und eine v5 bestehen beide aus sechs Halbtonschritten. In beiden Fällen
nennen wir dieses Intervall auch \emph{Tritonus}.


\begin{table}[h]
  \centering
  \begin{tabular}{rlc}
    \toprule
    Größe         & Name         & Kürzel\\
    \midrule
    0\,\acr{HTS}  & reine Prime  & r1\\
    5\,\acr{HTS}  & reine Quarte & r4\\
    7\,\acr{HTS}  & reine Quinte & r5\\
    12\,\acr{HTS} & reine Oktave & r8\\
    \bottomrule
  \end{tabular}
  \caption{Halbtonschritte reiner Intervalle}\label{tab:rInt}
\end{table}

\subsection{Große und kleine Intervalle}
Für die übrigen Intervalle gibt es zwei Ausprägungsformen, die etwa gleich oft
vorkommen, nämlich groß und klein: Sie heißen \emph{groß}, wenn sie exakt die
Abstände aus \cref{tab:gInt} bezeichnen.

Haben sie einen Halbtonschritt weniger, heißen sie \emph{klein} (k2 usf.). Davon
ausgehend gibt es wieder Ausnahmen in beide Richtungen: Haben sie einen
Halbtonschritt weniger als klein, heißen sie \emph{vermindert} (v2 usf.); haben
sie einen Halbtonschritt mehr als groß, dann heißen sie \emph{übermäßig} (ü2
usf.).

Eine übermäßige Sekunde (ü2) wird auch als \emph{Hiatus} bezeichnet.
Sie besteht wie die k3 aus drei Halbtonschritten, klingt im Kontext aber
sehr ungewöhnlich und sollte daher bei der Melodieführung vermieden werden.


\begin{table}[h]
  \centering
  \begin{tabular}{rlrl}
    \toprule
    Größe         & Name          & Kürzel\\
    \midrule
    2\,\acr{HTS}  & große Sekunde & g2\\
    4\,\acr{HTS}  & große Terz    & g3\\
    9\,\acr{HTS}  & große Sexte   & g6\\
    11\,\acr{HTS} & große Septime & g7\\
    \bottomrule
  \end{tabular}
  \caption{Halbtonschritte großer Intervalle}\label{tab:gInt}
\end{table}

\begin{table}[h]
  \centering
  \begin{tabular}{cccccccccccccccc}
    \toprule
    \textminus\kern.2pt 1 & 0 & 1 & 2 & 3 & 4 & 5 & 6 & 7 & 8 & 9 & 10 & 11 & 12 & 13\\
    \midrule
    v1 & r1 & ü1\\
       & v2 & k2 & g2 & ü2\\
       &    &    & v3 & k3 & g3 & ü3\\
       &    &    &    &    & v4 & r4 & ü4\\
       &    &    &    &    &    &    & v5 & r5 & ü5\\
       &    &    &    &    &    &    &    & v6 & k6 & g6 & ü6\\
       &    &    &    &    &    &    &    &    &    & v7 & k7 & g7 & ü7\\
       &    &    &    &    &    &    &    &    &    &    &    & v8 & r8 & ü8\\
    \bottomrule
  \end{tabular}
  \caption{Eine Aufschlüsselung sämtlicher Intervallnamen}\label{tab:ints}
\end{table}

\subsection{Komplementärintervall}
Für ein gegebenes Intervall heißt das, was zur Oktave fehlt,
\emph{Komplementärintervall}. So ist zum Beispiel die Terz komplementär zur
Sexte.

Für Komplementärintervalle gilt: Rein bleibt rein, groß wird klein, klein wird
groß, übermäßig wird vermindert und vermindert wird übermäßig.


\begin{figure}
  \centering
  \includegraphics{ly/s-intervalle}\\[-1px]
  \small \hspace*{6.1px}%
  k3\hspace*{21.7px}%
  g6\hspace*{22.3px}%
  k7\hspace*{22.7px}%
  r4\hspace*{22.4px}%
  v5\hspace*{22.3px}%
  ü8\hspace*{22.8px}%
  ü7\hspace*{21px}%
  v6\hspace*{-.3px}%\hspace*{3px}
  \caption{Eine Reihe willkürlich gewählter Intervalle und ihre Namen. Man mache
    sich klar, dass diese Benennung die einzig mögliche ist. Zum Beispiel könnte
    das letzte Intervall nicht als r5 bezeichnet werden (obwohl es 7\,\acr{HTS}
    umfasst), weil gis’ und es” als Grundintervall eine Sexte bilden.}
\end{figure}


\subsection{Intervalle hören}
Ohne musikalischen Kontext können wir natürlich nur den Abstand in \acr{HTS}
wahrnehmen. Wollen wir also zur Übung Intervalle hören, so machen wir uns den
Umstand zunutze, dass das Spektrum von 0 bis 12\,\acr{HTS} bis auf den Tritonus
eindeutig durch reine, kleine und große Intervalle abgedeckt ist:


\begin{table}[h]
  \centering
  \begin{tabular}{ccccccccccccc}
    \toprule
    $0$ & $1$ & $2$ & $3$ & $4$ & $5$ & $6$ & $7$ & $8$ & $9$ & $10$ & $11$ & $12$\\
    \midrule
    r1 & k2 & g2 & k3 & g3 & r4 & trit. & r5 & k6 & g6 & k7 & g7 & r8\\
    \bottomrule
  \end{tabular}
  \caption{Standardbezeichnungen für kontextfreie klingende Intervalle}
\end{table}

\section{Tonarten und der Quintenzirkel}

Ein Musikstück hat in der Regel einen Ton mit herausgehobener Bedeutung
(Zentrum, Schlusston) und nutzt (bis auf einige Ausnahmen) die Töne einer
bestimmten Skala, die auf diesem Ton aufbaut.

\subsection{Tonart}
Eine \emph{Tonart} ist gegeben durch einen \emph{Grundton} (etwa fis) und ein
\emph{Tongeschlecht} (Dur oder Moll). Im Prinzip könnte jede Kirchenskala
Tongeschlecht sein; wir beschränken uns aber auf Dur und Moll.
Im Fall von Dur notieren wir den Grundton mit einem beginnenden
Großbuchstaben, im Fall von Moll mit Kleinbuchstaben. Insgesamt schreiben wir
also zum Beispiel „Fis-Dur“ oder „fis-Moll“.

Üblicherweise weisen wir einem Stück eine Tonart zu, und sagen etwa, das Stück
„steht in fis-Moll“. In diesem Fall nutzt das Stück bis auf Ausnahmen nur die
Töne der entsprechenden Skala. Für die in der Skala standardmäßig alterierten
Töne verwenden wir die bereits in Definition~\ref{Vorzeichen} erwähnten
Generalvorzeichen.

Wir fragen uns nun, nach welchem Muster Generalvorzeichen hinzutreten. Dies
führt uns zum Begriff des Quintenzirkels.

\subsection{Quintenzirkel}
Starten wir mit einem c und bewegen uns im Quintabstand in beide Richtungen,
so ergibt sich eine unendliche Kette aus Tönen. In dieser ist jedoch
jedes 12. Glied bis auf Enharmonik gleich, sodass ein Kreis entsteht.

Die Besonderheit ist nun, dass auf diese Weise die oben erwähnten Tonarten
geordnet werden: Baut man auf den entsprechenden Tönen eine Durskala auf,
kommt mit jedem Schritt im Kreis nach „unten“ ein Generalvorzeichen hinzu:
links von c Tiefalterationen und rechts von c Hochalterationen.

Das gleiche können wir für Moll machen: Hier fangen wir bei a an, weil a-Moll
die Molltonart ohne Generalvorzeichen ist. Es ergeben sich zwei Kreise, die wir
ineinander zeichnen können. Zu jeder vollen Stunde gibt es also eine Dur- und
eine Molltonart mit den gleichen Generalvorzeichen.

\begin{figure}[h]
  \centering
  \begin{tikzpicture}[scale=1]
  \node at ({4.4*sin(15)},{4.4*cos(15)}) {\footnotesize+\,fis};
  \node at ({4.4*sin(45)},{4.4*cos(45)}) {\footnotesize+\,cis};
  \node at ({4.4*sin(75)},{4.4*cos(75)}) {\footnotesize+\,gis};
  \node at ({4.4*sin(105)},{4.4*cos(105)}) {\footnotesize+\,dis};
  \node at ({4.4*sin(135)},{4.4*cos(135)}) {\footnotesize+\,ais};
  \node at ({4.3*sin(165)},{4.3*cos(165)}) {\footnotesize+\,eis};
  \node at ({-4.4*sin(15)},{4.4*cos(15)}) {\footnotesize+\,b};
  \node at ({-4.4*sin(45)},{4.4*cos(45)}) {\footnotesize+\,es};
  \node at ({-4.4*sin(75)},{4.4*cos(75)}) {\footnotesize+\,as};
  \node at ({-4.4*sin(105)},{4.4*cos(105)}) {\footnotesize+\,des};
  \node at ({-4.4*sin(135)},{4.4*cos(135)}) {\footnotesize+\,ges};
  \node at ({-4.3*sin(165)},{4.3*cos(165)}) {\footnotesize+\,ces};
  \foreach \y in {0,...,5} {
    \pgfmathsetmacro\XA{84 - \y*30};
    \pgfmathsetmacro\XB{\XA -18};
    \pgfmathsetmacro\YA{83 - \y*30};
    \pgfmathsetmacro\YB{\YA -16};
    \pgfmathsetmacro\SA{96 + \y*30};
    \pgfmathsetmacro\SB{\SA +18};
    \pgfmathsetmacro\TA{97 + \y*30};
    \pgfmathsetmacro\TB{\TA +16};
    \centerarc[-to](0,0)(\XA:\XB:4);
    \centerarc[-to](0,0)(\YA:\YB:3);
    \centerarc[-to](0,0)(\SA:\SB:4);
    \centerarc[-to](0,0)(\TA:\TB:3);
  }
  \foreach \y in {0,...,11} {
    \draw ({3.3*sin(\y*30)},{3.3*cos(\y*30)}) -- ({3.7*sin(\y*30)},{3.7*cos(\y*30)});
  }
  \filldraw[white] (-.1,-3.28) rectangle (.1,-3.4);
  \filldraw[white] (-.1,-3.64) rectangle (.1,-3.72);
  \node at (0,4) {C};
  \node at (4,0) {A};
  \node at (0,-3.83) {Ges};
  \node at (0.0,-4.18) {Fis};
  \node at (-4,0) {Es};
  \node at (2,3.46) {G};
  \node at (3.46,2) {D};
  \node at (3.46,-2) {E};
  \node at (2,-3.46) {H};
  \node at (-2,3.46) {F};
  \node at (-3.46,2) {B};
  \node at (-3.46,-2) {As};
  \node at (-2,-3.46) {Des};
  \node at (0,3) {a};
  \node at (3,0) {fis};
  \node at (0,-2.83) {es};
  \node at (0,-3.17) {dis};
  \node at (-3,0) {c};
  \node at (1.5,2.6) {e};
  \node at (2.6,1.5) {h};
  \node at (2.6,-1.5) {cis};
  \node at (1.5,-2.6) {gis};
  \node at (-1.5,2.6) {d};
  \node at (-2.6,1.5) {g};
  \node at (-2.6,-1.5) {f};
  \node at (-1.5,-2.6) {b};
\end{tikzpicture}

  \caption{Der Quintenzirkel}\label{fig:QZ}
\end{figure}

\subsection{Parallelen und Varianten}\label{Parallele}
Zwei Tonarten mit gleichen Generalvorzeichen heißen \emph{parallel}. So
ist zum Beispiel a-Moll die Paralleltonart von C-Dur. Die
Grundtöne paralleler Tonarten sind stets um eine kleine Terz versetzt.

Zwei Tonarten mit gleichem Grundton heißen \emph{variant}. So ist zum Beispiel
a-Moll die Variante von A-Dur und umgekehrt. Variante Tonarten unterscheiden
sich um drei Generalvorzeichen.

\subsection{Transposition und Modulation}
Der Wechsel des Grundtons eines Stücks bei Beibehaltung des Tongeschlechtes
(d.\,h. jeder Ton wird um das gleiche Intervall verschoben) heißt
\emph{Transposition}. Dies ist ein rein technischer Vorgang und kann aus
verschiedenen Gründen notwendig sein (etwa um ein Stück singbar zu bekommen).

Hingegen heißt der möglichst fließende und kunstvolle Wechsel der Tonart im
Verlauf eines Stückes \emph{Modulation}. Es gibt verschiedene Techniken, so
etwas zu erreichen, von denen wir einige besprechen werden.

\section{Dreiklänge, Stufen und Funktionen}

Ein \emph{Akkord} ist das gleichzeitige Erklingen von mindestens drei Tönen.
Eine Spezialform bildet der Dreiklang, der wie folgt konstruiert wird:

\subsection{Dreiklang}
Schichtet man auf einem gegebenen Ton zwei Terzen, so bildet sich ein
\emph{Dreiklang}. Hier gibt es vier Möglichkeiten, siehe
\cref{tab:Drei}. Beim verminderten und beim übermäßigen Dreiklang
tritt die Besonderheit auf, dass die Quinte zwischen den Randtönen nicht rein
ist.


\begin{table}[h]
  \centering
  \begin{tabular}{cl}
    \toprule
    Terzschichtung & Name\\
    \midrule
    k3\,+\,k3      & Verminderter Dreiklang\\
    k3\,+\,g3      & Molldreiklang\\
    g3\,+\,k3      & Durdreiklang\\
    g3\,+\,g3      & Übermäßiger Dreiklang\\
    \bottomrule
  \end{tabular}
  \caption{Möglichkeiten, Dreiklänge durch das Schichten von Terzen zu
    bilden}\label{tab:Drei}
\end{table}

\subsection{Terzverwandtschaften}
Zwei Dreiklänge heißen \emph{terzverwandt}, wenn sie eine Terz gemeinsam haben.
\begin{itemize}
\item Die Dreiklänge von Paralleltonarten sind terzverwandt
  (z.\,B.\ C-Dur und a-Moll); sie teilen eine große Terz (c\,–\,e). Man
  spricht vom \emph{Parallelklang}.
\item Ein Durdreiklang ist außerdem noch terzverwandt zu der
  Molltonart, die auf der g3 darüber aufbaut (z.\,B.\ C-Dur und e-Moll).
  Diese Dreiklänge teilen eine kleine Terz (e\,–\,g). Man spricht
  vom \emph{Gegenklang}.
\end{itemize}


\subsection{Stufen in Dur}
Innerhalb einer Dur-Tonart können wir auf jedem Skalenton einen Dreiklang
aufbauen. Wir nummerieren sie lateinisch durch:

\begin{figure}[h]
  \centering
  \includegraphics{ly/s-stufen-dur}\\[-3px]
  \textsc{\hspace*{7px}%
    i\hspace{26.8px}%
    ii\hspace{23.2px}%
    iii\hspace{21.2px}%
    iv\hspace{23.8px}%
    v\hspace{24px}%
    vi\hspace{20.1px}%
    vii\hspace{19.7px}%
    (i)}
  \caption{Die Stufen in Dur}
\end{figure}


\noindent Hierbei sehen wir, dass auf den Stufen \textsc{i}, \textsc{iv} und
\textsc{v} Durdreiklänge stehen, während auf den Stufen \textsc{ii},
\textsc{iii} und \textsc{vi} Molldreiklänge stehen. Auf der Stufe \textsc{vii}
steht ein verminderter Dreiklang, den wir vorerst vernachlässigen.

Die drei Durstufen, genannt \emph{Hauptfunktionen}, haben einen sehr
charakteristischen Klang. Wir nennen sie \emph{Tonika}, \emph{Subdominante} und
\emph{Dominante} wie in \cref{tab:HF}.

Auf den Stufen \textsc{vi}, \textsc{ii} bzw.\ \textsc{iii} stehen die parallelen
Mollklänge von Tonika, Subdominante bzw. Dominante. Wir nennen diese
Stufen folglich \emph{Tonika-}, \emph{Subdominant-} und \emph{Dominantparallele}
(Tp, Sp, Dp).


\begin{table}[h]
  \centering
  \begin{tabular}{clcl}
    \toprule
    Stufe       & Name         & Symbol & Charakter\\
    \midrule
    \textsc{i}  & Tonika       & T      & Tonales Zentrum\\
    \textsc{iv} & Subdominante & S      & Entfernung vom Zentrum\\
    \textsc{v}  & Dominante    & D      & Spannung zum Zentrum\\
    \bottomrule
  \end{tabular}
  \caption{Hauptfunktionen in Dur}\label{tab:HF}
\end{table}

Bevor wir eine ähnliche Konstruktion innerhalb einer gegebenen Molltonart
durchführen, müssen wir noch einen besonderen Ton erläutern, dem auch in
Moll eine wichtige Bedeutung zukommt.

\subsection{Leitton}
Der Ton, der eine kleine Sekunde (k2) unter dem Grundton liegt (unabhängig
davon, ob er skaleneigen ist oder nicht), heißt \emph{Leitton}. Er hat eine
starke Strebetendenz aufwärts zum Grundton und ist daher ein wichtiges Mittel, um
auf ein Ende hinzuarbeiten.

Das, was im Falle von Dur der Dominante einen spannungsreichen Charakter gibt,
ist der Umstand, dass seine Terz der Leitton der Skala ist. In Moll
muss man, um dies zu erreichen, die Dominante ebenfalls nach Dur setzen. Dafür
benötigen wir die g7 der Skala, die nicht leitereigen ist.


\subsection{Stufen in Moll}
Wir erhalten in Moll folgende Stufen:\todo{align}
\begin{figure}[h]
  \centering
  \includegraphics{ly/s-stufen-moll}\\
  \textsc{\hspace*{10px}%
    i\hspace{27.3px}%
    ii\hspace{23.7px}%
    iii\hspace{21.7px}%
    iv\hspace{24.8px}%
    v\hspace{24.5px}%
    vi\hspace{23.6px}%
    vii\hspace{21px}%
    (i)}
  \caption{Die Stufen in Moll}
\end{figure}

\noindent Der Dreiklang \textsc{ii} ist vermindert und wird vorerst
vernachlässigt. Wieder kommt den Stufen \textsc{i}, \textsc{iv} und \textsc{v}
eine zentrale Bedeutung zu. Wir bezeichnen sie als \emph{Tonika}~(t),
\emph{Subdominante}~(s) und \emph{Dominante}~(D). Bemerke, dass Tonika und
Subdominante Molldreiklänge sind, während die Dominante künstlich nach Dur
gesetzt ist.

Wieder sehen wir, dass die Stufen \textsc{iii}, \textsc{vi} bzw.\ \textsc{vii}
die parallelen Durklänge von Tonika, Subdominante bzw.\ (Moll-)Dominante sind.
Wir bezeichnen sie daher wieder als \emph{Tonika-}, \emph{Subdominant-}
und \emph{Dominantparallele} (tP, sP, dP).


\subsection{Tonvorrat in Moll}
Durch den skalenfremden Leitton im Dominantklang wird es notwendig,
je nach Kontext große und kleine Sexten und Septimen zu nutzen (hier am Beispiel a-Moll):
\begin{itemize}
\item Soll der Leitton von unten erreicht werden und ein Hiatus zwischen k6
  und g7 vermieden werden, kann die Sexte mitalteriert werden
  (e’\,–\,\textit{fis’}\,–\,\textit{gis’}\,–\,a’).
\item Bei Abwärtsbewegungen vom Grundton aus, bei denen
  die Dominante gar nicht erklingt, wird die normale Skala genutzt
  (a’\,–\,\textit{g’}\,–\,\textit{f’}\,–\,e’).
\item Ein Hiatus kann auch durch einen Lagenwechsel
  vermieden werden. In diesem Fall kann auch die für Moll charakteristische k6
  in Kombination mit der g7 eingesetzt werden
  (e’\,–\,\textit{f’}\,–\,\textit{gis}\,–\,a).
\end{itemize}
So stehen je nach Kontext andere Töne zur Verfügung, was das Notensetzen in
Molltonarten reizvoller macht.

Manchmal werden die oben beschriebenen drei Situationen als verschiedene
„Moll\-skalen“ bezeichnet. Da jedoch in einem Stück sämtliche Situationen
auftreten können, scheint es uns sinnvoller, von einem großen Tonvorrat zu
sprechen, aus dem je nach Kontext geschöpft wird.


\subsection{Varianten und Gegenklänge als Funktionen}
Taucht in einer Tonart die Variante einer Funktion auf, so ist dies leicht zu
notieren, indem der entsprechende Buchstabe groß statt klein (oder umgekehrt)
geschrieben wird. So könnte man Es-Dur in C-Dur als „tP“ bezeichnen, als
\emph{Tonikavariantparallele}. Hier ist die Reihenfolge entscheidend: Die
\emph{Tonikaparallelvariante}, notiert „TP“, lautet A-Dur. Solche Funktionen
sind allerdings sehr selten.

Eine Ausnahme, die tatsächlich recht häufig vorkommt, ist die Dur-Variante
$\muT$ am Ende eines Stücks, das in Moll steht. Die darin enthaltene große
Terz wird \emph{picardische Terz} genannt. Dieser Klang hat einen
„erlösenden“ Charakter.

Darüber hinaus könnte man einige Stufen auch als Gegenklänge anderer Stufen
auffassen, zum Beispiel T\kern-.5pxp\,=\,Sg. In der Praxis ist das nur an
einer Stelle relevant: Die sP in Moll ist oft ein „Ersatz“ für die Tonika
und wird daher auch als tG, also als \emph{Tonikagegenklang}, bezeichnet.


% \pagebreak

\subsection{Kadenz}
Eine \emph{Kadenz} ist eine Akkordverbindung, die zur Schlussbildung verwendet
wird. In ihrer Grundform lautet sie D\,–\,T in Dur und D\,–\,t in Moll. Es
gibt nun verschiedene Erweiterungen und Abwandlungen:
\begin{itemize}
\item Endet eine Kadenz nicht wie erwartet in der Tonika, spricht man von
  einem \emph{Trugschluss}. Die üblichen Formen sind D\,–\,Tp in Dur
  und D\,–\,tG in Moll.
\item Oft taucht eine längere Version der Kadenz auf, die sich
  \emph{Vollkadenz} nennt und die Form T\,–\,S\,–\,D\,–\,T in Dur
  bzw. t\,–\,s\,–\,D\,–\,t in Moll hat.
\item In Dur taucht auch die Schlussbildung S\,–\,T auf, die \emph{plagale
    Kadenz} genannt wird. Sie klingt wie ein „Abstieg“ zum Ausgangspunkt.
\end{itemize}


\subsection{Kadenzraum}
Die sechs Grundfunktionen einer Tonart sind im Quintenzirkel benachbart. Auf
diese Weise bilden sie den \emph{Kadenzraum}, siehe \cref{fig:KadR}.\vspace*{-5pt}


\begin{figure}[h]
  \centering
  \tikzstyle{very densely dashed}=[dash pattern=on 1.8pt off 1.3pt]
  \begin{tikzpicture}
    \node at (0,4) {T};
    \node at (-2,3.46) {S};
    \node at (2,3.46) {D};
    \node at (0,3) {Tp};
    \node at (-1.5,2.6) {Sp};
    \node at (1.5,2.6) {Dp};
    \draw (0,3.32) -- (0,3.75);
    \draw (1.64,2.84) -- (1.87,3.22);
    \draw (-1.68,2.91) -- (-1.87,3.22);
    \centerarc[-to](0,0)(85:65:4);
    \centerarc[-to](0,0)(95:115:4);
    \centerarc[-to](0,0)(83:68:3);
    \centerarc[-to](0,0)(97:113:3);
    \centerarc[very densely dashed](0,0)(44:56:4);
    \centerarc[very densely dashed](0,0)(124:136:4);
    \centerarc[very densely dashed](0,0)(44:54:3);
    \centerarc[very densely dashed](0,0)(126:136:3);
  \end{tikzpicture}
  \hspace*{1cm}
  \begin{tikzpicture}
    \node at (0,4) {tP};
    \node at (-2,3.46) {sP\kern-.8px/\kern-.1pxtG};
    \node at (2,3.46) {dP};
    \node at (0,3) {t};
    \node at (-1.5,2.6) {s};
    \node at (1.5,2.6) {(D)};
    \draw (0,3.3) -- (0,3.7);
    \draw (1.66,2.875) -- (1.87,3.22);
    \draw (-1.62,2.805) -- (-1.87,3.22);
    \centerarc[-to](0,0)(85:65:4);
    \centerarc[-to](0,0)(95:111:4);
    \centerarc[-to](0,0)(86:68:3);
    \centerarc[-to](0,0)(94:115:3);
    \centerarc[very densely dashed](0,0)(44:56:4);
    \centerarc[very densely dashed](0,0)(126:136:4);
    \centerarc[very densely dashed](0,0)(44:54:3);
    \centerarc[very densely dashed](0,0)(124:136:3);
  \end{tikzpicture}
  \caption{Der Kadenzraum in Dur (links) und in Moll (rechts)}\label{fig:KadR}
\end{figure}

\section{Struktur von Notensätzen}

Nachdem wir uns in den vorangegangenen Abschnitten eine Sprache zur Beschreibung
von Tönen und ihren Verhältnissen erarbeitet haben, wollen wir nun \emph{in}
dieser Sprache Methoden erarbeiten, um Notensätze zu schreiben.

Hierbei stellen wir viele „Regeln“ auf, die unser historisch gewachsenes
Ästhetikempfinden wiedergeben. Natürlich kann es Teil der individuellen
Ausdrucksform sein, sie bewusst zu brechen – man sollte nur wissen,
was man tut.

\subsection{Horizontale Struktur}
Eine Bewegung um eine Sekunde heißt \emph{Schritt}, eine größere
\emph{Sprung}; eine nach unten \emph{Fallen}, eine nach oben \emph{Steigen},
keine \emph{Liegenbleiben}. Grundlegende Regeln für die Gestaltung einer Melodie sind
folgende:
\begin{itemize}
\item Es sind die Töne des Tonvorrats einer Tonart.
\item Sprünge, die größer als eine Oktave sind, sind unzulässig. Sprünge, die
  größer als eine Quinte sind, sollten individuell geprüft werden.% und sollten
  % nur eingesetzt werden, wenn der musikalische Geschmack es rechtfertigt.
\item Große Sprünge sollten durch an sie anschließende Schritte in die andere
  Richtung „abgefedert“ werden.
\end{itemize}


\begin{table}
  \centering
  \setlength{\belowrulesep}{1pt}
  \renewcommand{\arraystretch}{1.3}
  \begin{tabular}{lcc}
    \toprule
    Stimme & Beginn & Ende\\
    \midrule
    Sopran & c’     & g”\\
    Alt    & g      & d”\\
    Tenor  & d      & g’\\
    Bass   & F      & e’\\
    \bottomrule
  \end{tabular}
  \caption{Die Tonumfänge verschiedener Chorstimmen}\label{tab:Tonumfaenge}
\end{table}

\subsection{(Chor-)satz}
Ein \emph{Satz} ist eine Kombination aus mehreren gleichzeitig erklingenden
eigenständigen Stimmen, deren vertikale Reihenfolge bis auf wenige Ausnahmen
(die wir \emph{Stimmkreuzung} nennen) gleich bleibt. Im vierstimmigen Chorsatz
bewegen sich die Stimmen innerhalb des in \cref{tab:Tonumfaenge} beschriebenen
Bereichs.

In der kompakten Notation werden diese vier Stimmen in zwei Systeme
eingetragen, wobei die jeweils obere Stimme nach oben „gehalst“ wird, die
anderen nach unten. Auf diese Weise können auch unterschiedliche Rhythmen
notiert werden:

\begin{figure}[h]
  \centering
  \includegraphics{ly/s-palestrina}\\
  \caption{G.\,P.\,d.\,Palestrina, \emph{Sicut cervus},
    Psalm 42:1–3, \acr{T}.\,19–23}
\end{figure}

\noindent Natürlich können wir nicht beliebige Melodien kombinieren. Wir
werden Regeln erarbeiten, welche Kombinationen möglich sind. Dabei beschränken
wir uns zunächst auf den \emph{homophonen} Fall, in dem alle Stimmen den
gleichen Rhythmus haben.


\subsection{Vertikale Struktur}Folgende Regeln erledigen schon einmal das
Gröbste:
\begin{itemize}
\item Bis auf noch zu erläuternde Ausnahmen ergeben die Stimmen einen Akkord,
  der eine Funktion der jeweiligen Tonart ist. Dabei darf die Quinte fehlen.
\item Grundton und Terz dürfen verdoppelt werden, die Quinte nicht.
\item Der Abstand S\,–\,A sowie A\,–\,T darf maximal eine Oktave betragen.
\item Im Bass darf der Grundton oder die Terz stehen; die Quinte hingegen nur
  in Ausnahmefällen, etwa wenn sie „auf dem Weg“ liegt. Oft möchten wir bei
  der Funktionsbezeichnung notieren, welcher Dreiklangston im Bass steht. Dies
  geschieht als Index, z.\,B. $\muT_3$.
\item Bei jeder Akkordverbindung sollten wenigstens zwei der
  Bewegungsrichtungen (Steigen, Fallen, Liegen) vorkommen. Zu
  vermeiden ist eine Verbindung, bei der sich alle Stimmen in die gleiche
  Richtung bewegen (\emph{Satzrutsch}).
\end{itemize}


\subsection{Quint- und Oktavparallelen}
Es ist zu vermeiden, dass sich zwei Stimmen \emph{parallel} im Abstand von
Quinten oder Oktaven bewegen.

Dies wird in \cref{fig:QPb} an einigen Beispielen verdeutlicht, die
alle als Quintparallelen gesehen werden und unzulässig sind (selbiges gilt für
Oktaven). Ins\-besondere sind auch folgende Situationen problematisch:

\begin{itemize}[itemsep=0em]
\item Parallelen, die durch zusätzliche Zwischennoten entstehen,
\item Parallelen, die nur durch unbetonte Noten unterbrochen werden (\emph{Akzentparallele}),
\item Parallelen, die bis auf Oktaven bestehen (\emph{Antiparallele}),
\item Wechsel von v5 nach r5 (\emph{vermindert\,–\,rein}).
\end{itemize}

\begin{figure}[h]
  \centering
  \includegraphics{ly/s-parallelen1}\vspace*{-3pt}
  \caption{Problematische Quintparallelbewegungen}\label{fig:QPb}
\end{figure}

Einige ähnliche Situationen sind hingegen nicht unüblich, siehe \cref{fig:QPg}:
\begin{itemize}[itemsep=0em]
\item Quartparallelen (also Quintparallelen bis auf „negative“ Oktaven),
\item Wechsel von r5 nach v5 (\emph{rein\,–\,vermindert}),
\item Parallelen aus zeitversetzten Wechseln (hier: e”\,–\,d” kommt vor a’\,–\,g’).
\end{itemize}
\begin{figure}[h]
  \centering
  \includegraphics{ly/s-parallelen2}\vspace*{-5pt}
  \caption{Unproblematische Quintparallelbewegungen}\label{fig:QPg}
\end{figure}

\subsection{Auflösung des Leittons}
Folgt nach der Dominante die Tonika, muss die Terz (also der Leitton) aufwärts
in den Grundton geführt werden. Aus diesem Grund führte eine Verdopplung des
Leittones zu Oktavparallelen und ist daher unzulässig.

Es gibt eine Ausnahme (die jedoch keine Leittonverdopplung rechtfertigt): Der
Grundton kann in gleicher Lage von einer anderen Stimme über\-nommen werden,
siehe \cref{fig:lat}. Dies heißt \emph{latente Auflösung}.

\begin{figure}[h]
  \centering
  \includegraphics{ly/s-latent}
  ~\\[-46px]~~\hspace*{.5px}\rotatebox{45}{\rule{3.2mm}{.7px}}\vspace*{27px}
  \caption{Eine latente Auflösung}\label{fig:lat}
\end{figure}

\section{Harmoniefremde Töne}

Im letzten Abschnitt haben wir strenge Regeln besprochen, die
einen Satz klassisch und geordnet klingen lassen. Ab jetzt werden wir
verschiedene Möglichkeiten kennen lernen, diese Regeln aufzuweichen.

Vor allem werden wir Situationen besprechen, in denen zusätzlich zum
vorliegenden Dreiklang andere Töne erklingen können: zum Beispiel weil sie eine
Linie weiterführen oder bewusst „verzögern“, oder weil sie dem Akkord einen
besonderen Charakter (Wärme, Spannung) verleihen.

\subsection{Zwischennoten}
Es gibt die Möglichkeit, \emph{zwischen} den Schlägen in einzelnen Stimmen weitere
Noten unterzubringen, um die Melodie auszugestalten.

\begin{itemize}
\item Ein Terzsprung kann mithilfe einer \emph{Durchgangsnote} überbrückt werden:
  \begin{figure}[h]
    \centering
    \includegraphics{ly/s-durchgang}
    \caption{Ein Notenbeispiel mit etlichen Durchgangsnoten}
  \end{figure}
\item Zwischen Tonwiederholungen kann eine \emph{Wechselnote}, also ein
  Schritt nach oben oder unten (und wieder zurück), eingebaut werden:
  \begin{figure}[h]
    \centering
    \includegraphics{ly/s-wechsel}
    \caption{Ein Notenbeispiel mit einigen Wechselnoten}
  \end{figure}
\end{itemize}

Abschließend einige Anmerkungen zu diesen Zwischennoten:

\begin{itemize}
\item Wir raten dazu, einen Satz zunächst ohne Zwischennoten zu entwerfen und
  sie danach sparsam „drüberzustreuen“.
\item Durch Zwischennoten können Quint- und Oktavparallelen entstehen.
\item Tauchen in mehreren Stimmen gleichzeitig Zwischennoten auf, so müssen
  sie eine Terz oder Sexte bilden (bis auf Oktaven).
\end{itemize}

Während Zwischennoten eine Möglichkeit darstellen, auf unbetonten Zeiten
harmoniefremde Töne zu platzieren, bieten Vorhalte eine Möglichkeit,
\emph{auf} der Zählzeit einen fremden Ton zu setzen.

\subsection{Vorhalte}
Anstelle eines Dreiklangstones kann im Rahmen eines \emph{Vorhalts} in einer
der drei Oberstimmen der Ton einen Schritt darüber erklingen, sofern dieser
danach schrittweise abwärts in den „richtigen“ Ton geführt wird.

Dies kann für alle Dreiklangstöne (3, 5 oder 8) geschehen; es gibt demnach
Quart-, Sext- und Nonenvorhalte. Oft möchten wir Vorhalte auch in die
Funktionsbezeichnung mit aufnehmen. Dies geschieht durch Hochstellung, z.\,B.\
$\muD{}^{\kern.5px 4\kern1px3}$.
Bei der Verwendung von Vorhalten sollte man folgendes beachten:
\begin{itemize}
\item %\textbf{Konsonante Einführung der Vorhaltsnote}\\
  Vorhalte müssen vorbereitet werden, d.\,h. die Vorhaltsnote muss in einem
  zuvor erklingenden Akkord konsonant eingeführt werden und liegen
  bleiben. Oft wird diese Note dann übergebunden und die Auflösung kommt
  versetzt vor dem nächsten Schlag,
  siehe \cref{fig:Vorhalt}.
\item %\textbf{Skaleneigene Vorhaltsnoten}\\
  Es sind skaleneigene Vorhaltsnoten zu verwenden. Dies hat den Effekt, dass
  die Quart im $\muS{}^{\smash{\kern.5px 4\kern1px3}}$ übermäßig ist, weil sie der
  skaleneigenen g7 entspricht.
\item %\textbf{Keine Vorwegnahme der Auflösung}\\
  Vorenthaltene Terzen und Quinten dürfen nicht bereits in einer anderen Stimme
  erklingen. Vorenthaltene Oktaven sind unproblematisch.
\end{itemize}
Wir wollen abschließend auf eine Besonderheit hinweisen: Ein Quartsextvorhalt
auf der Dominante, also $\muD{}^{\kern.5px\smash{^6_4\kern1px{}^5_3}}$, ist
genauso aufgebaut wie die Tonika mit Quinte im Bass, also $\muT_5$
bzw. $\mut_5$.  Dies erklärt, weshalb Akkorde mit Quinte im Bass „instabil“
klingen: Man erwartet die Auflösung zweier Vorhaltsnoten.

\begin{figure}[h]
  \centering
  \includegraphics{ly/s-vorhalt}\vspace*{-3pt}
  \caption{Viele verschiedene Vorhalte mit konsonanter Einführung}\label{fig:Vorhalt}
\end{figure}

\section{Charakteristische Zusatztöne}

Für die speziellen Funktionen der Subdominante und der Dominante haben sich im
Laufe der Zeit zusätzliche Töne herausgebildet, die den Charakter der Funktion
verstärken: Bei der Dominante die Septime und bei der Subdominante die
Sexte. Wir wollen diese Vierklänge im Detail studieren.

\subsection{Dominantseptakkord}
Fügt man der Dominante die leitereigene Septime (sowohl in Dur als auch in Moll
klein) hinzu, nennt man den entstehenden Vierklang \emph{Dominantseptakkord},
kurz $\muD^{\kern.4px 7}$.  Durch den zwischen g3 und k7 bestehenden Tritonus
wird die Strebetendenz zur Tonika und somit der dominantische Charakter
verstärkt. Einige Regeln:
\begin{itemize}
\item Die Septime muss abwärts in die Terz der Tonika geführt werden. Eine
  latente Auflösung ist möglich; dennoch darf die Septime nicht verdoppelt
  werden.
\item Wie zuvor muss die Terz aufwärts aufgelöst werden; die Quinte darf
  fehlen.
\item Im Bass darf der Grundton, die Terz und sogar die Septime liegen.
\end{itemize}
Durch die Septime kann ein Quartvorhalt der Tonika vorbereitet
werden. Außerdem entspricht die Septime im $\muD^{\kern.4px 7}$
der Oktave der Subdominante, sodass bei einer Vollkadenz bei
der Verbindung $\muS$\,–\,$\muD^{\kern.4px 7}$ ein Ton liegen
bleiben kann.

% Hier einige Beispiele, die die verschiedenen Möglichkeiten zeigen:
\begin{figure}[h]
  \centering
  \includegraphics{ly/s-d7}\\[-57.9px]\hspace*{-1.98cm}\rotatebox{-55}{\rule{3.5mm}{.7px}}~\\[32px]
  \caption{Diverse Formen des Dominantseptakkords und seiner Auflösung. Im
    letzten Beispiel kann das d im Alt auch als Durchgangsnote aufgefasst
    werden.}
\end{figure}

\subsection{Verkürzter Dominantseptakkord}
Eine Variation des Dominantseptakkord erlaubt es, den Grundton auszulassen. In
diesem Fall muss die Quinte doppelt vorkommen, im Bass liegen und sich dort
schrittweise abwärts in den Grundton der Tonika auflösen. Dies geht in Dur wie
in Moll, siehe \cref{fig:vD}.

Den sich dadurch ergebenden Akkord nennen wir \emph{verkürzten
  Dominantseptakkord}, notiert
\mbox{\raisebox{-.2px}{\rotatebox{40}{\rule{4mm}{.7px}}}\hspace*{-8.4px}$\muD_5^7$}.
Dieser Akkord besteht aus zwei kleinen Terzen und ist somit ein
verminderter Dreiklang. In Dur besteht er genau aus den Tönen der Stufe \textsc{vii}.

\begin{figure}[h]
  \centering
  \includegraphics{ly/s-d7verk}
  \caption{Der verkürzte Dominantseptakkord und seine Auflösung}\label{fig:vD}
\end{figure}

\subsection{Sixte ajoutée}
Fügt man der Subdominante die leitereigene Sexte (sowohl in Dur als auch in
Moll groß) hinzu, so nennt man den entstehenden Vierklang \textit{Sixte ajoutée},
kurz $\muS^{\kern.4px{}^6_5}$ bzw.\ $\mus^{\kern.4px{}^6_5}$. Da die
Sexte der Subdominante der Quinte der Dominante entspricht, wird die
Strebetendenz hin zur Dominante und somit der subdominantische Charakter
verstärkt. Einige Regeln:
\begin{itemize}
\item Alle vier Töne müssen vorkommen; die Quinte darf nicht weggelassen werden
  (andernfalls läge im Falle von Dur stattdessen eine Umkehrung der Sp vor).
\item Im Bass darf der Grundton oder die Terz liegen, in Moll auch die
  Sexte. Letzteres sollte in Dur vermieden werden, um eine Verwechslung
  mit der Sp auszuschließen.
\item Die Quinte und die Sexte des Akkords sollten in unterschiedliche
  Richtungen weitergeführt werden (oder einer der beiden Töne bleibt liegen).
\end{itemize}
Die Sixte ajoutée in Moll hat eine ü4 (Tritonus) zwischen der k3
und der g6.


\begin{figure}[h]
  \centering
  \includegraphics{ly/s-s6}
  \caption{Einige Erscheinungsformen der Sixte ajoutée}
\end{figure}

\section{Erweiterung des Kadenzraums}

Um einen Satz harmonisch reichhaltiger zu gestalten, können auch Funktionen
außerhalb des Kadenzraumes verwendet werden. Dabei nutzen wir die Bewegung
D\,–\,T relativ zu verschiedenen Grundfunktionen.

Wie wir sehen werden, sind hierbei die Übergänge zu Modulationen, also zu
vollständigen Veränderungen der Tonart, fließend.

\subsection{Doppeldominante}
Sowohl in Dur als auch in Moll heißt die Durdominante der Dominante
\emph{Doppeldominante} $\muDD$. Um sie zu bilden, braucht man eine
skalenfremde Quarte. Dies ist genau das Generalvorzeichen, das wir einführen
wollen, wenn wir unseren Kadenzraum „nach rechts“ verschieben.

Analog können wir den Doppeldominantsept sowie seine Verkürzung verwenden. Die
Doppeldominante führt zur Dominante; es gelten die gleichen Regeln zur
Auflösung der einzelnen Töne wie bei der Dominante selbst.

\begin{figure}[h]
  \centering
  \includegraphics{ly/s-dd}\\[-3px]
  \small\hspace*{31px}%
  $\muT$\hspace*{14px}%
  $\muS^{\kern.5px{}^6_5}$\hspace*{3.5px}%
  $\muDD\kern.3px{}^7_3$\hspace*{4px}%
  $\muD^{\kern.5px{}^4_6\hspace*{4px}\kern1px{}^3_5\hspace*{2px}}$\hspace*{13px}%
  $\muT$\hspace*{7px}%
  $\muDD$\hspace*{7px}%
  $\muD$\hspace*{6px}%
  $\muDD\kern.3px{}^7_3$\hspace*{9px}%
  $\muD$\hspace*{5px}%
  \raisebox{-.45px}{\rotatebox{40}{\rule{4.7mm}{.7px}}}\hspace*{-9.7px}$\muDD\kern.3px{}^7_3$\hspace*{3.6px}%
  $\muD$\hspace*{8.4px}%
  $\muT$\hspace*{13px}%
  $\muDD_7$\hspace*{4.5px}%
  $\muD_{\kern.4px3}^{\kern.4px 8\,\hspace*{1px}7\hspace*{-1.4px}}$\hspace*{6.5px}%
  $\muT$
  \caption{Ein Notenbeispiel mit vielen Doppeldominanten, aus \cite[S.\,199]{deLaMotte}}
\end{figure}

\subsection{Zwischendominante}
Allgemeiner können zu allen Funktionen Do\-mi\-nan\-ten gebildet werden, die
wir \emph{Zwischendominanten} nennen. Sie werden mit (D) notiert; die Klammer
bedeutet, dass die Dominante relativ zur Folgefunktion zu verstehen ist.  Wir
wollen einige übliche Zwischendominanten
hervorheben:
\begin{itemize}
\item In Dur hat der Dominantsept zur Subdominante eine skalenfremde Septime.
  Dies ist genau das Generalvorzeichen, das wir einführen wollen, wenn wir unseren
  Kadenzraum „nach links“ verschieben.
\item In Dur sind die Zwischendominanten zu den Parallelklängen Sp bzw.\ Tp
  besonders beliebt. Hierzu wird der Grundton bzw.\ die Quinte hochalteriert.
\item In Moll ist die Zwischendominante zur Subdominante s besonders beliebt:
  Diese entspricht der Variante der Tonika.\vspace*{-5pt}
\end{itemize}

\begin{figure}[h]
  \centering
  \includegraphics{ly/s-zwischend}\\[-3px]
  \small\hspace*{10.2mm}%
  $\muT$\hspace*{.2mm}%
  $(\hspace*{-.1mm}\muD\kern.3px{}^7\hspace*{-.1mm})$\hspace*{.2mm}%
  $\muS$\hspace*{3.6mm}%
  $\muD^{\kern.3px 8\hspace*{.3mm}7}$\hspace*{3.9mm}%
  $\muT$\hspace*{13.4mm}%%%%%%%%%5
  $\muT$\hspace*{1.1mm}%
  $(\hspace*{-.1mm}\muD\kern.3px{}_3^7\hspace*{-.1mm})$\hspace*{.2mm}%
  $\muS\mup$\hspace*{.5mm}%
  $\muD\kern.3px{}_3$\hspace*{3.5mm}%
  $\muT$\hspace*{16.1mm}%%%%%%%
  $\mut$\hspace*{1.5mm}%
  $(\hspace*{-.2mm}\muD\kern.3px{}_7\hspace*{-.4mm})$\hspace*{.2mm}%
  $\mus$\hspace*{4.6mm}%
  $\muDD\kern.3px{}_3^7$\hspace*{1.6mm}%
  $\muD^{\kern.3px{}_{4\hspace*{1.5mm}3}^{\hspace{4.7mm}7\hspace*{.3mm}}}$\hspace*{1.2mm}%
  $\muT$\hspace*{.3mm}
  \caption{Einige übliche Zwischendominanten}\label{fig:ZD}
\end{figure}

\subsection{Zwischenkadenz}
Manchmal ist eine ganze Kadenz relativ zu einer bestimmten Zielfunktion gesetzt,
siehe \cref{fig:ZK}.

\begin{figure}[h]
  \centering
  \includegraphics{ly/s-zwischenkadenz}
  \small \hspace*{14.8mm}%
  $\muT$\hspace*{4.6mm}%
  $(\mus^{\kern.3px{}^6_5}$\hspace*{1.5mm}%
  $\muD^{\kern.3px 43})$\hspace*{.3mm}%
  $\muD\mup$\hspace*{1.2mm}%
  $(\muD^{7}_3)$\hspace*{3.2mm}%
  $\muS^{\kern.3px{}^6_5}$\hspace*{1.7mm}%
  $\muD^{\kern.3px8\kern5px7\kern-1px}$\hspace*{1.9mm}%
  $\muT$\hspace*{5.8mm}
  \caption{Eine Zwischenkadenz zur Dominantparallele}\label{fig:ZK}
\end{figure}

\section{Dominanten mit Nonen}\label{None}

Es gibt drei sehr beliebte Abwandlungen der Dominante (oder einer beliebigen
Zwischendominante), die alle gemeinsam haben, dass sie eine None enthalten. Sie
alle haben einen sehr spezifischen Klang und sind ziemlich eindeutig einer
Musikepoche zuzuordnen.

\subsection{Barock: Verminderter Septakkord}
Im Dominantseptakkord kann der Grundton durch die \emph{kleine} None ersetzt werden,
um einen weiteren Strebeton zu haben. Während die k9 der Dominante in Moll
leitereigen ist (sie entspricht der k6 der Skala), wird sie in Dur durch
Tiefalteration erreicht. Im Bass kann jeder der verbliebenen Töne (3, 5, 7
oder 9) liegen.

Die None muss abwärts in die Quinte des Folgeakkords aufgelöst werden; die
Auflösungsregeln von Leitton (Terz) und Septime bleiben unverändert. Nur
die Quinte des Akkords ist in ihrer Bewegung frei; allerdings können
Quintparallelen der Form „vermindert\,–\,rein“ entstehen, falls 5\,–\,1
unterhalb von 9\,–\,5 liegt.

Im Falle der Doppeldominante können None bzw.\ Septime in einen Sext- bzw.\
Quartvorhalt geführt werden. In Dur entsteht hier ein sehr interessanter
Halbtonschritt von der k9 der Doppeldominante zur g6 der Dominante.

\begin{figure}[h]
  \centering
  \includegraphics{ly/s-Dv1}
  \caption{Einige verminderte Septakkorde und ihre Auflösungen}
\end{figure}

\noindent Die Frage, wie dieser Akkord funktional bezeichnet und notiert werden soll, ist
nicht leicht zu beantworten: Unsere Beschreibung als Dominante ohne Grundton ist
künstlich, weil sie auf einem Ton aufbaut, der nicht vorhanden ist.
Historisch richtiger wäre, ihn als Vierklang, der auf dem Leitton fußt und
aus \emph{drei} kleinen Terzen besteht, zu sehen. Deswegen wird er
\emph{verminderter Septakkord} genannt. Relativ zum Leitton enthält
er also die r1, k3, v5 und v7.

Nachteil dieser historisch adäquaten Sichtweise ist, dass sich die Rollen der
beteiligten Töne um eine Terz verschöben, wenn der Leitton als
Fundament des Akkords bezeichnet würde. Außerdem würden die
Auflösungsbewegungen nicht erklärt.  Wir wollen daher die Bestandteile
weiterhin relativ zum nicht vorhandenen Grundton bezeichnen, also als g3, r5,
k7 und k9. Wir notieren den Akkord als $\muD^v$ und schreiben den Basston in
den Index: Es gibt also $\muD^v_3$, $\muD_5^v$, $\muD_7^v$ und $\muD_9^v$.


\begin{figure}[h]
  \centering
  \includegraphics{ly/s-Dv2}\\[-3px]
  \small\hspace*{16.7mm}%
  $\mut$\hspace*{2.8mm}%
  $\muD^v_3$\hspace*{1.55mm}%
  $\mut$\hspace*{2.45mm}%
  $\muD_7^v$\hspace*{2.65mm}%
  $\mut_3$\hspace*{1.35mm}%
  $\muD^v_5$\hspace*{1.43mm}%
  $\mut_{3}$\hspace*{6.27mm}%
  $\muD^v_9$\hspace*{1.1mm}%
  $\mut_5$\hspace*{2mm}%
  $\mus^{^5_6}$\hspace*{1.3mm}%
  $\muDD^v_3$\hspace*{1.5mm}%
  $\muD^{\kern.3px{}^{4\hspace*{2mm}3}_{8\hspace*{2mm}7}}$\hspace*{3.4mm}%
  $\mut$\hspace*{10mm}
  \caption{Beispiel mit etlichen verminderten Septakkorden, frei nach
    \cite[S.\,99]{deLaMotte}}
\end{figure}

\subsection{Klassik: $\muDDZ^{\bm{v}}$ mit tiefalterierter Quinte}
Der zur Doppeldominante gehörende verminderte Septakkord kann weiter
angeschäft werden, indem man die Quinte tiefalteriert. Damit hat nun jeder
der vier Töne eine Strebetendenz um eine k2 nach oben oder unten.  Die
tiefalterierte Quinte der Doppeldominante entspricht der k6 der Skala und ist
daher in Dur skalenfremd, jedoch in Moll skaleneigen.

Die tiefalterierte Quinte muss im Bass liegen und zur Auflösung abwärts in den
Grundton der Dominante geführt werden. Auf diese Weise entstehen
Quintparallelen, die nur durch einen Quartsextvorhalt der Dominante verhindert
werden können, oft aber in Kauf genommen werden (immerhin sind wir mit diesem
Akkord schon in der Zeit nach Bach). Der Akkord wird notiert als
$\muDD^v_{5-}$.\vspace*{-5px}

\begin{figure}[h]
  \centering
  \includegraphics{ly/s-Dv5-}
  \caption{Verminderte Septakkorde mit tiefalterierter Quinte und ihre
    Auflösungen}
\end{figure}

\subsection{Romantik: Dominantseptnonakkord}
In Dur kann dem Dominantseptakkord auch die skaleneigene g9 hinzugefügt
werden. Dadurch entsteht ein Fünfklang, von dem die Quinte oder der Grundton
ausgelassen werden kann. Wird die Quin\-te ausgelassen, sprechen wir vom
\emph{Dominantseptnonakkord} $\muD^{\kern.3px{}^9_7}$, andernfalls vom
\emph{verkürzten Dominantseptnonakkord}
\mbox{\raisebox{-.2px}{\rotatebox{40}{\rule{4mm}{.7px}}}\hspace*{-8.4px}$\muD^{\kern.3px{}^9_7}$}. Beide
machen den Dominantklang „wärmer“.

Wieder kann jeder der Töne im Bass liegen. Die None und die Septime müssen
schritt\-weise abwärts geführt werden, der Leitton aufwärts.

\begin{figure}[h]
  \centering
  \includegraphics{ly/s-D79}
  \caption{Dominantseptnonakkorde in G-Dur sowie ihre Auflösungen}
\end{figure}

\section{Größere Linien}

In diesem letzten Abschnitt wollen wir den Blick etwas weiten und
einige grobe Gestaltungsprinzipien zur Inspiration studieren.

\subsection{Gegenbewegungen}%
Führt man Sopran und Bass in entgegengesetzte Richtun\-gen, können damit nicht
nur Parallelen vermieden werden; es klingt auch gut, siehe \cref{fig:gegen}.

\begin{figure}[h]
  \centering
  \includegraphics{ly/s-gegen}\vspace*{-3pt}
  \caption{G.\,F.\,Händel, \emph{Tochter Zion}, Siegeschor
    aus \emph{Judas Maccabäus} (\acr{HWV}\,63)}\label{fig:gegen}
\end{figure}


\subsection{Chromatik im Bass}
Wenn es sich harmonisch anbietet, können chromatisch steigende Basslinien
dem Satz einen „anschwellenden“ Charakter verleihen.%\vspace*{-5pt}
\begin{figure}[h]
  \centering
  \includegraphics{ly/s-chroma}\vspace*{-3pt}
  \caption{Standard"|idee für einen Satz zu \emph{We wish you a merry christmas}}
\end{figure}

\subsection{Quintfälle}
Im weitesten Sinne sind damit Akkordketten gemeint, bei denen der Grundton stets
um eine Quinte abnimmt. Hierzu bietet es sich an, in Stufen zu denken: Eine in
Moll typische Verbindung ist
\textsc{i}\,–\,\textsc{iv}\,–\,\textsc{vii}\,–\,\textsc{iii}\,–\,\textsc{vi}. Da
hierbei permanent ein Dominante-Tonika-Verhältnis angedeutet wird, klingen
solche Ketten sehr „folgerichtig“, manchmal sogar erwartbar.

Fast immer hat in diesen Fällen der Bass sämtliche Grundtöne und springt damit
entweder eine r4 aufwärts oder eine r5 abwärts.

Wenn die Kette hinreichend lang weitergeführt, aber dabei nicht moduliert werden
soll, muss zwangsläufig eine Quinte vermindert bzw. eine Quarte übermäßig
sein. Auch in Instrumentalwerken ist dieses Modell sehr verbreitet.
\begin{figure}[h]
  \centering
  \includegraphics{ly/s-quintfall}
  \caption{Ein Quintfall in: A.\,Vivaldi, \emph{Herbst}, Satz 1 in F-Dur aus
    \emph{Le quattro stagioni} (op.\,8, \acr{RV}\,269).}
\end{figure}

\subsection{Sequenz}
Eine andere Möglichkeit, den Eindruck von Folgerichtigkeit zu erzeugen,
stellen Sequenzen dar: Hier wird ein melodisches oder rhythmisches Motiv
nacheinander auf unterschiedlichen Höhen gespielt.

\begin{figure}[h]
  \centering
  \includegraphics{ly/s-sequenz}
  \caption{Etliche kleine Sequenzen in: J.\,S.\,Bach, Präludium fis-Moll aus dem
    \emph{Wohltemperierten Klavier 1} (\acr{BWV}\,859).}
\end{figure}

\printbibliography

\end{document}

%%% Local Variables:
%%% TeX-engine: default
%%% End:
