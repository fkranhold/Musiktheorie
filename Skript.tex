\documentclass[ngerman]{scrartcl}
%% Serifen in Überschriften
\KOMAoptions{paper=b5,numbers=enddot,DIV=11,headings=standardclasses}

%% Encoding
\usepackage[utf8]{inputenc}
\usepackage[T1]{fontenc}

%% Sprache
\usepackage{babel}
\usepackage[cleanlook]{isodate}

%% Schriftpakete
\usepackage{lmodern}
\usepackage[osf,sc]{mathpazo}
\usepackage{classico}
\usepackage[scaled=.85]{beramono}

%% Palatino braucht mehr Platz
\usepackage{setspace}
\setstretch{1.07}
\renewcommand{\arraystretch}{1.07} % Für tabular- und array-Umgebungen

%% Mikrotypographie!
\usepackage{ellipsis}
\usepackage[babel,tracking=true]{microtype}
\UseMicrotypeSet[tracking]{smallcaps}
\SetTracking{encoding=*,shape=sc}{30}

%% Akronyme
\newcommand{\acr}[1]{\textls[40]{\textsc{\MakeLowercase{#1}}}}

%% Eigene Listing styles mit kurzer Definition
\usepackage[shortlabels]{enumitem}

%% Captions for figures and tables
\setkomafont{captionlabel}{\footnotesize\bfseries}
\setkomafont{caption}     {\footnotesize}
\setcapwidth{.9\textwidth}
\renewcommand*{\captionformat}{.\ }
\setcapindent{0em}
\addtokomafont{caption}{\setstretch{1.07}}

%% Fancy tables
\usepackage{booktabs}

%% Mehrspaltinge Aufzählungen
\usepackage{multicol}

%% Bibliographie
\usepackage{biblatex}
\bibliography{bibliography.bib}

%% Footnote-Design
\deffootnote{1.5em}{1.5em}{\thefootnotemark.\ }
\addtokomafont{footnote}{\setstretch{1.07}}

%% Todo
\usepackage[size=tiny]{todonotes}

%% Grundlegende mathematische Symbole
\usepackage{amssymb,amsmath} % Standard
\usepackage{mathtools}       % \coloneqq, \eqqcolon

%% TikZ and TikZ-CD
\usepackage{tikz-cd}
\usepackage{tikz}
\usetikzlibrary{arrows,calc}

%% Mini-Indizes (\scaleto{…}{3pt})
\usepackage{scalerel}

%% Bold
\newcommand{\bm} [1]{\mathbold{#1}}

%% TikZ: centerarc
 \def\centerarc[#1](#2)(#3:#4:#5)% Syntax: [draw options] (center) (initial angle:final angle:radius)
     { \draw[#1] ($(#2)+({#5*cos(#3)},{#5*sin(#3)})$) arc (#3:#4:#5); }

%% Musikalische Symbole
\usepackage{musicography}

%% Harmonische Funktionen
\newcommand{\mut}         {\text{t}}
\newcommand{\mus}         {\text{s}}
\newcommand{\mud}         {\text{d}}
\newcommand{\muT}         {\text{T}}
\newcommand{\muS}         {\text{S}}
\newcommand{\muD}         {\text{D}}
\newcommand{\mup}         {\text{p}}
\newcommand{\muP}         {\text{P}}
\newcommand{\mug}         {\text{g}}
\newcommand{\muG}         {\text{G}}
\newcommand{\muDD}        {\mbox{\muD\hspace{-5.5px}\raisebox{1.3px}{\muD}}}
\newcommand{\muDDZ}       {\mbox{\muD\hspace{-6.1px}\raisebox{1.3px}{\muD}}}
\newcommand{\muSS}        {\raisebox{1.5px}{\muS}\hspace{-4px}\muS}
\newcommand{\muss}        {\raisebox{1.5px}{\mus}\hspace{-3px}\mus}
\newcommand{\musDD}       {\muD\hspace{-6px}\raisebox{2px}{\sout{\muD}}}
\newcommand{\vermsept}    {\hspace{1mm}\raisebox{4px}{\mus}\hspace{-3px}\muD}
\newcommand{\sevsept}     {\hspace{1mm}\raisebox{3px}{\muS}\hspace{-5px}\muD}
\newcommand{\dvermsept}   {\hspace{1mm}\raisebox{4px}{\mut}\hspace{-1px}\muD\hspace{-6px}\raisebox{2px}{\muD}}

%% Hyperref and Cleveref
\usepackage{amsthm}
\usepackage[pdfusetitle,bookmarksnumbered=true]{hyperref}
\usepackage[noabbrev]{cleveref}

%% Aufgaben
\theoremstyle{definition}
\newtheorem{aufg}{Aufgabe}[section]

%\usepackage[dvipsnames]{xcolor}
%\newcommand{\rb}[1]{\textcolor{Maroon}{#1}}
\newcommand{\rb}[1]{#1}


\begin{document}

%\subject {CdE-WinterAkademie 2025\,·\,26}
\author  {Florian Kranhold\and Charlotte Mertz}
\title   {Musiktheorie}
\subtitle{Skript, Version 4.1}
\hypersetup{pdftitle={Musiktheorie: Skript}}

% Wenn Marginalien verschwinden sollen:
\renewcommand{\marginpar}[1]{}

\maketitle

Dieses Skript entstand im Zuge mehrerer Musiktheorie-Kurse, die wir auf
CdE-Akademien geleitet haben (zuletzt auf der CdE-WinterAkademie 2025\,·\,26 in
Windischleuba). Es behandelt die theoretischen Grundlagen mehrstimmiger
klassischer Musik, wobei die Konzepte zumeist am Beispiel vierstimmiger
Chorsätze im Stile Bachs erläutert werden.

Unsere Aufbereitung der Thematik orientiert sich größtenteils an dem
Standardwerk von Diether de la Motte \cite{deLaMotte}, das wir den an einer
detaillierteren Abhandlung interessierten Leser:innen ans Herz legen möchten.

Wir danken Thomas Drögemüller für zahlreiche Anmerkungen und Korrekturen, durch
die das Skript deutlich an Präzision gewonnen hat, sowie Maria Ravvina für
diverse erhellende Diskussionen über Satzregeln.

\section{Töne und Noten}

In diesem Abschnitt beschreiben wir die kleinsten Bausteine, mit denen wir
arbeiten: \emph{Töne}.  Um alle begrifflich auf den gleichen Stand zu bringen,
beginnen wir mit einer sehr komprimierten Übersicht, wie Töne benannt werden.
Damit dabei der Fokus des Kurses nicht aus den Augen zu verloren wird,
verzichten wir auf eine Erläuterung der mathematischen\footnote{Irgendwann
  in den nächsten zehn Jahren wird es hierüber einen Kurs »Mathematische
  Musiktheorie« geben, zu dem ich (Florian) schon jetzt herzlich einlade.} und
historischen Hintergründe, auch wenn dies unweigerlich zur Folge hat, dass
einige Konventionen quasi vom Himmel fallen. An diesen Stellen werden wir uns
die Formulierung „aus Gründen™“ zueigen machen.

% Für uns ist die Höhe eines Tons gegeben durch die
% \emph{Frequenz}, mit der er schwingt. Auf diese Weise lässt sich leicht
% definieren, was eine \emph{Oktave} ist. Die Möglichkeiten, eine Oktave zu
% durchschreiten, lernen wir als \emph{Skalen} kennen.

\subsection{Tonlänge und Tonhöhe}
Ein Ton beschreibt eine physikalische Schwingung, etwa einer Saite oder einer
Luftsäule.  In der Musik interessieren uns dabei zwei Aspekte:
\begin{itemize}[nosep]
\item die Dauer der Schwingung: dies ist  die \emph{Tonlänge},
\item die Frequenz der Schwingung: dies ist die \emph{Tonhöhe}.
\end{itemize}
Die Länge von Tönen wird üblicherweise in Form von \emph{Notenwerten} (ganz,
halb, viertel, achtel usf.) angegeben.  Dieses Konzept dürfte den meisten
bekannt sein, weswegen wir es es nicht gesondert wiederholen. Wer eine
Auffrischung wünscht, sei auf
\href{https://de.wikipedia.org/wiki/Notenwert}{\textsf{de.wikipedia.org/wiki/Notenwert}}
verwiesen.\looseness-1

Nicht so leicht hingegen wollen wir es uns bei der Benennung von Tonhöhen
machen.  Die Frequenz eines Tones wird in Hertz (Schwingungen pro Sekunde, kurz Hz)
angegeben. Je größer die Frequenz, desto höher klingt der Ton für
uns.  Unsere erste Aufgabe ist es, bestimmten Frequenzen Namen zu
geben, damit über konkrete Tonhöhen gesprochen werden kann.

\subsection{Kammerton}

Die erste Tonbezeichnung, die wir einführen wollen, ist die folgende: Einen Ton
der Frequenz 440\,Hz bezeichnen wir als \emph{Kammerton}.\footnote{Im Laufe der
  Jahrhunderte wurden auch andere Konventionen verwendet.}  Die allermeisten
Stimmgabeln erzeugen genau diesen Ton.\looseness-1

Alle übrigen Notennamen werden \emph{in Relation} zum Kammerton angegeben.
Hierfür wollen wir eine musikalisch sinnvolle »kleinste Einheit« festsetzen; dann
können wir so etwas sagen wie »drei Einheiten über dem Kammerton«.

\subsection{Oktaven und Halbtonschritte}

Für die Art und Weise, wie das menschliche Ohr den Abstand zwischen zwei Tönen
wahrnimmt, ist das \emph{Verhältnis} ihrer entsprechenden Frequenzen
entscheidend (und nicht etwa ihre Differenz).

Der grundlegendste Abstand zwischen zwei Tönen ist der einer
\emph{Oktave}:\footnote{Woher der Bezug zur Zahl 8 (lat. \emph{octavus} »der
  Achte«) kommt, klärt sich in \cref{subsec:int}.} er entspricht dem
Frequenzverhältnis 1\,:\,2.  Der Ton eine Oktave über dem Kammerton hat folglich
eine Frequenz von 880\,Hz.  Das menschliche Ohr nimmt Töne, die sich nur um
Oktaven unterscheiden, als sehr ähnlich wahr; dieses Phänomen nennt sich
\emph{Oktavidentität}.

Wir unterteilen nun aus Gründen™ den Abstand einer Oktave gleichmäßig in 12
Schritte, die \emph{Halbtonschritte} (\acr{HTS}) genannt werden.\footnote{Für
  Mathematikinteressierte wenigstens ein Trostpflaster: Bei dieser 12er-Teilung
  sind 7\,\acr{HTS} nah an 2\,:\,3, 5\,\acr{HTS} nah an 3\,:\,4 und 4\,\acr{HTS}
  nah an 4\,:\,5. Solch gute Näherungen an einfache Frequenzverhältnisse gäbe es
  erst wieder bei einer 53er-Teilung.}  Ein Halbtonschritt entspricht also dem
Frequenzverhältnis 1\,:$\sqrt[\mkern-1mu\text{12}\mkern4mu]{\text{2}}$.  Für uns
ist ein Halbtonschritt genau die oben genannte kleinste musikalisch sinnvolle
Einheit.\footnote{Wir lassen hierbei Mikrotonalität und andere historisch
  gewachsene Stimmungssysteme komplett außer Acht. Der hier beschriebene Ansatz
  wird auch als \emph{gleichstufig} bezeichnet.}\looseness-1

\subsection{Klaviatur}

Die für uns relevanten Töne sind also diejenigen, die vom Kammerton aus durch
eine bestimmte Anzahl von Halbtonschritten erreicht werden können.  Ein Klavier
hat für jeden dieser Töne genau eine Taste, und zwar so, dass es »nach rechts«
immer höher wird.  Beschränkt man sich dabei auf den musikalisch nutzbaren
Frequenzbereich, kommt man bei 88 Tasten heraus.

Aus Gründen™ legt man aber nicht alle Tasten gleichmäßig nebeneinander, sondern
unterscheidet zwischen \emph{weißen} und \emph{schwarzen} Tasten wie in
\cref{fig:klav} dargestellt.\footnote{Den Grund für diese Anordnung erfahren wir
  in \cref{sec:Tonart}.}  Das sich daraus ergebende Schema nennt sich
\emph{Klaviatur}.  Wir bemerken, dass sich das Muster der Klaviatur alle 12
Tasten wiederholt, sodass Oktavabstände leicht darzustellen sind.

\begin{figure}[h]
  \centering
  \begin{tikzpicture}[xscale=.24,yscale=1.2]
    \fill[black!20] (26,1) rectangle (27,0);
    \draw (-2,1) -- (-2,0);
    \fill (-1.3,1) rectangle (-.7,.3);
    \draw (-1,1) -- (-1,0);
    \foreach \o in {0,...,6} {
      \pgfmathtruncatemacro{\lab}{\o+1}
      \draw ({7*\o+.05},1.1) -- ({7*\o+.1},1.2) -- ({7*\o+6.9},1.2) -- ({7*\o+6.95},1.1);
      \draw[fill=white] ({7*\o+3.5},1.2) ellipse (.6 and .12);
      \node at ({7*\o+3.5},1.2) {\tiny \lab};
      \foreach \t in {0,...,6} {
        \draw ({7*\o+\t},0) -- ({7*\o+\t},1);
      }
      \fill ({7*\o+0.7},1) rectangle ({7*\o+1.3},.3);
      \fill ({7*\o+1.7},1) rectangle ({7*\o+2.3},.3);
      \fill ({7*\o+3.7},1) rectangle ({7*\o+4.3},.3);
      \fill ({7*\o+4.7},1) rectangle ({7*\o+5.3},.3);
      \fill ({7*\o+5.7},1) rectangle ({7*\o+6.3},.3);
    }
    \draw (49,1) -- (49,0);
    \draw (50,1) -- (50,0);
    \draw (-2,0) -- (50,0);
  \end{tikzpicture}
  \caption{Die Tasten eines Klaviers, kurz \emph{Klaviatur}, mit weißen und
    schwarzen Tasten.  Die grau markierte Taste entspricht dem
    Kammerton. Die obige Gruppierung von je 12 aufeinanderfolgenden Tasten
    wird in \cref{subsec:or} erläutert.}\label{fig:klav}
\end{figure}

\subsection{Oktavräume}
\label{subsec:or}

Unsere Aufgabe ist nach wie vor, Tonhöhen zu benennen, d.\,h. den Tasten der
Klaviatur Namen zu geben.  Dabei soll berücksichtigt werden, dass Töne, die sich
nur um Oktaven unterscheiden, für uns sehr ähnlich klingen.

Um dies zu tun, gruppieren wir jeweils 12 aufeinanderfolgende Tasten zu einem
\emph{Oktavraum}, siehe die »Klammern« in \cref{fig:klav}. (Die vier Tasten, die
von keiner Klammer abgedeckt sind, können für unseren Kurs vernachlässigt
werden.) Der Tonname setzt sich nun aus zwei Teilen zusammen:
\begin{itemize}[nosep]
\item dem Oktavraum, zu dem der Ton gehört,
\item einer sogenannten »Tonklasse«, die unabhängig vom Oktavraum ist.
\end{itemize}
Im ersten Schritt müssen wir also die Oktavräume benennen.  In der
deutschsprachigen Musiktheorie haben sich hierfür gewisse Adjektive
(bzw. Präfixe) durchgesetzt, die sich in \cref{tab:OkR} finden, einschließlich
der Frequenz des jeweils tiefsten und höchsten Tons.  Wir bemerken, dass sich
der Kammerton in der vierten Klammer befindet und folglich ein
\emph{eingestrichener} Ton ist.

\begin{table}[h]
  \centering
  \begin{tabular}{clrr}
    \toprule
    Klammer & Name     &     Beginn &       Ende\\
    \midrule
    1 & Kontra-        &     33\,Hz &     62\,Hz\\
    2 & groß           &     65\,Hz &    123\,Hz\\
    3 & klein          &    130\,Hz &    246\,Hz\\
    4 & eingestrichen  &    262\,Hz &    493\,Hz\\
    5 & zweigestrichen &    523\,Hz &    987\,Hz\\
    6 & dreigestrichen & 1\,046\,Hz & 1\,975\,Hz\\
    7 & viergestrichen & 2\,093\,Hz & 3\,951\,Hz\\
    \bottomrule
  \end{tabular}
  \caption{Die verschiedenen Oktavräume und ihre zugehörigen
    Frequenzbereiche. Die Nummerierung der Klammern bezieht sich auf
    \cref{fig:klav}.}\label{tab:OkR}
\end{table}

\subsection{Die weißen Tasten}

\enlargethispage{\baselineskip} Wir widmen uns nun dem zweiten Bestandteil des
Tonnamens, nämlich der zuvor erwähnten \emph{Tonklasse}, die Oktaven außer Acht
lässt.  Dabei beschränken wir uns zunächst auf die weißen Tasten: Wir fixieren
einen beliebigen Oktavraum und weisen den darin befindlichen weißen Tasten die
in \cref{fig:weiNam} dargestellten Namen (\rb c, \rb d, \rb e, \rb f, \rb g, \rb
a, \rb h) zu.\footnote{Man würde vielleicht b statt h erwarten, und im
  Englischen ist das auch so. Der deutsche Sonderweg hat (wie könnte es anders sein) historische Gründe.} Es
sei nochmal darauf hingewiesen, dass diese Namen den Ton nur \emph{bis auf
  Oktaven} bestimmen:

\begin{quote}
  Ein begeisterter Tenor: »Ich kann heute ein a singen!«\\
  Ein verwunderter Bass: »Also das kann ich auch.«
\end{quote}

\noindent Im Kombination mit dem Oktavraum aber wird die Benennung der weißen Tasten
eindeutig. So ist zum Beispiel der Kammerton ein \emph{eingestrichenes a} – und
für Tenöre durchaus herausfordernd hoch!

Möchte man einen Tonnamen effizient notieren, so hat es sich etabliert, den
zugehörigen Oktavraum nicht auszuschreiben, sondern stattdessen den Buchstaben
der Tonklasse zu dekorieren oder zu kapitalisieren:
\begin{itemize}[nosep]
\item Bei $n$-gestrichenen Tönen werden $n$ Apostrophe angefügt.
\item Töne der Kontra-/großen Oktave werden mit Großbuchstaben notiert.
\item Bei Tönen der Kontraoktave wird ein Komma angefügt.
\end{itemize}
Zur Verdeutlichung: Die verschiedenen Töne in der Tonklasse c würde man wie
in aufsteigender Abfolge so notieren: C, C c c’ c” c”\kern-.75px’ c”\kern-.75px”.

\begin{figure}
  \centering
  \begin{tikzpicture}[xscale=.8,yscale=.8]
  \draw (0,-1) rectangle (1,3);
  \draw (1,-1) rectangle (2,3);
  \draw (2,-1) rectangle (3,3);
  \draw (3,-1) rectangle (4,3);
  \draw (4,-1) rectangle (5,3);
  \draw (5,-1) rectangle (6,3);
  \draw (6,-1) rectangle (7,3);
  \filldraw[fill=black] (0.7,0.8) rectangle (1.3,3);
  \filldraw[fill=black] (1.7,0.8) rectangle (2.3,3);
  \filldraw[fill=black] (3.7,0.8) rectangle (4.3,3);
  \filldraw[fill=black] (4.7,0.8) rectangle (5.3,3);
  \filldraw[fill=black] (5.7,0.8) rectangle (6.3,3);
  \node[text depth=.25ex, text height = 1.5ex] at (0.5,-0.2) {\small c};
  \node[text depth=.25ex, text height = 1.5ex] at (1.5,-0.2) {\small d};
  \node[text depth=.25ex, text height = 1.5ex] at (2.5,-0.2) {\small e};
  \node[text depth=.25ex, text height = 1.5ex] at (3.5,-0.2) {\small f};
  \node[text depth=.25ex, text height = 1.5ex] at (4.5,-0.2) {\small g};
  \node[text depth=.25ex, text height = 1.5ex] at (5.5,-0.2) {\small a};
  \node[text depth=.25ex, text height = 1.5ex] at (6.5,-0.2) {\small h};
\end{tikzpicture}

  \caption{Die Tonklassen für die weißen Tasten.}\label{fig:weiNam}
\end{figure}

\subsection{Alterationen}

Was uns noch zu unserem Glück fehlt, sind Tonklassen für die schwarzen Tasten.
Aber anstatt hierfür komplett neue Buchstaben zu nutzen, wechselt man die
Perspektive und führt das Konzept der \emph{Alteration} ein.  Damit ist gemeint,
dass man ausgehend von einer weißen Taste (dem \emph{Ausgangston})
einen\footnote{Es ist auch möglich, einen Ausgangston um \emph{zwei}
  Halbtonschritte zu alterieren. Das wollen wir hier allerdings ausklammern.}
Halbtonschritt nach oben oder unten geht.  Man sagt dann auch, dass der
Ausgangston \emph{hoch-} oder \emph{tiefalteriert} wird. So entspricht
beispielsweise ein hochalteriertes f genau der schwarzen Taste zwischen f und g.

Alterationen werden wie folgt benannt: Bei einer Hochalteration wird der
Tonklasse ein -\rb{is} an den Buchstaben angehängt, während bei einer
Tiefalteration ein -\rb{es} (bei Vokalen nur ein -\rb{s}) angehängt wird. Eine
Ausnahme gibt es:\footnote{Im Englischen heißt die Tiefalteration ganz
  regelkonform »B flat«.} Die Tiefalteration von h heißt zu allem Überfluss
\rb{b}. Der Oktavraum wird in der Benennung vom Ausgangston »geerbt«. So wird
beispielsweise die hochalterierte Form von c’ als cis’ bezeichnet.\looseness-1

Sämtliche sich daraus ergebenden Tonklassennamen sind in \cref{fig:klaviatur}
dargestellt.  Wir bemerken, dass nun jede Taste mindestens einen Namen trägt,
d.\,h. wir können endlich über jeden Ton sprechen.

Abschließend sei erwähnt, dass man durch Alteration in einen anderen Oktavraum
rutschen kann. Die Benennung richtet sich aber weiterhin nach dem Ausgangston.
So bezeichnen beispielsweise his’ und c” dieselbe Taste.

\begin{figure}[h]
  \centering
  \begin{tikzpicture}[xscale=1.2]
  \draw (0,-1) rectangle (1,3);
  \draw (1,-1) rectangle (2,3);
  \draw (2,-1) rectangle (3,3);
  \draw (3,-1) rectangle (4,3);
  \draw (4,-1) rectangle (5,3);
  \draw (5,-1) rectangle (6,3);
  \draw (6,-1) rectangle (7,3);
  \draw (7,-1) rectangle (8,3);
  \filldraw[fill=black] (0.7,0.8) rectangle (1.3,3);
  \filldraw[fill=black] (1.7,0.8) rectangle (2.3,3);
  \filldraw[fill=black] (3.7,0.8) rectangle (4.3,3);
  \filldraw[fill=black] (4.7,0.8) rectangle (5.3,3);
  \filldraw[fill=black] (5.7,0.8) rectangle (6.3,3);
  \node[text depth=.25ex, text height = 1.5ex] at (0.5,0.3) {\small c};
  \node[text depth=.25ex, text height = 1.5ex] at (1.5,0.3) {\small d};
  \node[text depth=.25ex, text height = 1.5ex] at (2.5,0.3) {\small e};
  \node[text depth=.25ex, text height = 1.5ex] at (3.5,0.3) {\small f};
  \node[text depth=.25ex, text height = 1.5ex] at (4.5,0.3) {\small g};
  \node[text depth=.25ex, text height = 1.5ex] at (5.5,0.3) {\small a};
  \node[text depth=.25ex, text height = 1.5ex] at (6.5,0.3) {\small h};
  \node[text depth=.25ex, text height = 1.5ex] at (7.5,0.3) {\small c};
  \node[text depth=.25ex, text height = 1.5ex] at (0.5,-0.15) {\small his};
  \node[text depth=.25ex, text height = 1.5ex] at (1.5,-0.15) {\small cisis};
  \node[text depth=.25ex, text height = 1.5ex] at (2.5,-0.15) {\small disis};
  \node[text depth=.25ex, text height = 1.5ex] at (3.5,-0.15) {\small eis};
  \node[text depth=.25ex, text height = 1.5ex] at (4.5,-0.15) {\small fisis};
  \node[text depth=.25ex, text height = 1.5ex] at (5.5,-0.15) {\small gisis};
  \node[text depth=.25ex, text height = 1.5ex] at (6.5,-0.15) {\small aisis};
  \node[text depth=.25ex, text height = 1.5ex] at (7.5,-0.15) {\small his};
  \node[text depth=.25ex, text height = 1.5ex] at (0.5,-0.6) {\small deses};
  \node[text depth=.25ex, text height = 1.5ex] at (1.5,-0.6) {\small eses};
  \node[text depth=.25ex, text height = 1.5ex] at (2.5,-0.6) {\small fes};
  \node[text depth=.25ex, text height = 1.5ex] at (3.5,-0.6) {\small geses};
  \node[text depth=.25ex, text height = 1.5ex] at (4.5,-0.6) {\small asas};
  \node[text depth=.25ex, text height = 1.5ex] at (5.5,-0.6) {\small heses};
  \node[text depth=.25ex, text height = 1.5ex] at (6.5,-0.6) {\small ces};
  \node[text depth=.25ex, text height = 1.5ex] at (7.5,-0.6) {\small deses};
  \node[text depth=.25ex, text height = 1.5ex,scale=.9] at (1,2) {\scriptsize\textcolor{white}{cis}};
  \node[text depth=.25ex, text height = 1.5ex,scale=.9] at (1,1.3) {\scriptsize\textcolor{white}{hisis}};
  \node[text depth=.25ex, text height = 1.5ex,scale=.9] at (2,2) {\scriptsize\textcolor{white}{dis}};
  \node[text depth=.25ex, text height = 1.5ex,scale=.9] at (2,1.3) {\scriptsize\textcolor{white}{feses}};
  \node[text depth=.25ex, text height = 1.5ex,scale=.9] at (4,2) {\scriptsize\textcolor{white}{fis}};
  \node[text depth=.25ex, text height = 1.5ex,scale=.9] at (4,1.3) {\scriptsize\textcolor{white}{eisis}};
  \node[text depth=.25ex, text height = 1.5ex,scale=.9] at (5,2) {\scriptsize\textcolor{white}{gis}};
  \node[text depth=.25ex, text height = 1.5ex,scale=.9] at (6,2) {\scriptsize\textcolor{white}{ais}};
  \node[text depth=.25ex, text height = 1.5ex,scale=.9] at (1,1.65) {\scriptsize\textcolor{white}{des}};
  \node[text depth=.25ex, text height = 1.5ex,scale=.9] at (2,1.65) {\scriptsize\textcolor{white}{es}};
  \node[text depth=.25ex, text height = 1.5ex,scale=.9] at (4,1.65) {\scriptsize\textcolor{white}{ges}};
  \node[text depth=.25ex, text height = 1.5ex,scale=.9] at (5,1.65) {\scriptsize\textcolor{white}{as}};
  \node[text depth=.25ex, text height = 1.5ex,scale=.9] at (6,1.65) {\scriptsize\textcolor{white}{b}};
  \node[text depth=.25ex, text height = 1.5ex,scale=.9] at (6,1.3) {\scriptsize\textcolor{white}{ceses}};
\end{tikzpicture}

  \caption{Die Tonklassen einschließlich der alterierten Formen.}\label{fig:klaviatur}
\end{figure}


\subsection{Enharmonik}

Was in \cref{fig:klaviatur} direkt auffällt, ist die Tatsache, dass unsere
Benennung \emph{überbestimmt} ist: Für viele Tasten gibt es nun mehrere
Bezeichnungen.  Anders gesagt: Zwei Töne können unterschiedlich benannt sein,
aber trotzdem gleich klingen.  Dieses Phänomen nennt sich \emph{Enharmonik}.

Warum nun sollte man bei der Tonbenennung fein säuberlich zwischen cis und des
trennen, wenn am Ende doch sowieso die gleiche Taste betätigt wird? Eine
berechtigte Frage, und sie beispielhaft zu beantworten wird ein Ziel dieses
Kurses sein. Der springende Punkt ist: \emph{Tonhöhe ist nicht alles.}  Je nach
harmonischem Kontext kann ein und derselbe Ton vollkommen anders auf uns wirken,
was grob gesagt daran liegt, dass unser Hirn beim Verarbeiten von Musik unterbewusst
Ausgangstöne mitzählt.

%\section{Notenschrift}
% (und ggf. Hilfslinien).% Hierbei legt der
% Ausgangston die vertikale Positionierung des Notenkopfes im System fest.

% Die vertikale Position des Notenkopfes
% hängt dabei vom Ausgangston (also der unalterierten Form) ab.

\subsection{Noten und Schlüssel}

Um einen Ton (mit Länge und Höhe) zu notieren, setzen wir eine \emph{Note} mit
passendem Wert (z.\,B.\ $\musHalf$, $\musQuarter$, $\musEighth$) in ein
Notensystem mit fünf Linien.

Durch einen \emph{Notenschlüssel} wird eine der fünf Linien mit einem
spezifischen Ton verknüpft. Die Position der übrigen Ausgangstöne ergibt sich
dann durch Abzählen: Pro Ausgangston geht es eine »halbe Linie« nach oben oder
unten, sodass Notenköpfe sowohl \emph{auf} als auch \emph{zwischen} den Linien
platziert werden . Wir verwenden folgende Schlüssel, siehe \cref{fig:Schl}:
\begin{itemize}[itemsep=0em]
\item Der \textit{g-Schlüssel} umkreist die Linie des eingestrichenen g.
\item Der \textit{c-Schlüssel} hat die Linie des eingestrichenen c als Spiegelachse.
\item Der \textit{f-Schlüssel} fasst die Linie des kleinen f zwischen zwei
  Punkten.
\end{itemize}
Diese drei Schlüssel können selbst unterschiedlich platziert werden, und
abhängig von ihrer Platzierung haben sie nochmal spezifischere Namen. Die in
\cref{fig:Schl} verwendete Platzierung ist die heutzutage übliche. In diesem
Fall heißen sie \emph{Violin-}, \emph{Alt-} bzw. \emph{Bassschlüssel}.

\begin{figure}[h]
  \centering
  \raisebox{-6.1px}{\includegraphics{ly/s-clef-g}}\quad
  \raisebox{0px}{\includegraphics{ly/s-clef-c}}\quad
  \raisebox{0px}{\includegraphics{ly/s-clef-f}}\quad
  \raisebox{-10px}{\includegraphics{ly/s-clef-g8}}
  \caption{Der g-Schlüssel als Violinschlüssel, der c-Schlüssel als Altschlüssel
    und der f-Schlüssel als Bass"|schlüssel. Schließlich noch ein
    oktavierter Violinschlüssel, in dem heutzutage oft Tenöre lesen dürfen. In
    allen Beispielen ist ein c’ notiert.\looseness-1}\label{fig:Schl}
\end{figure}

\subsection{Vorzeichen}\label{Vorzeichen}
Um Alterationen zu notieren, werden den Noten \emph{Vorzeichen} vorangestellt:
Ein \emph{Kreuz} ($\sharp$) für Hochalteration und ein \emph{B} (\kern.5px$\flat$\kern-.5px) für
Tiefalteration.  Grundsätzlich gilt ein Vorzeichen einen Takt lang. Darüber
hinaus können Vorzeichen zu Beginn eines Stückes vermerkt sein; dann gelten sie
dauerhaft. (Situationen, in denen dies erforderlich ist, sind in
\cref{sec:Tonart} erläutert.) In diesem Falle sprechen wir von
\emph{Generalvorzeichen}.

Sollen Vorzeichen (sowohl einzelne als auch Generalvorzeichen) vorzeitig ihre
Wirkung verlieren, so notiert man dies durch ein \emph{Auflösungs\-zeichen}
($\natural$). Dieses gilt ebenso einen Takt lang (d.\,h. im Folgetakt haben die
Generalvorzeichen wieder ihre volle Gültigkeit).

\section{Intervalle}

In diesem Abschnitt lernen wir, wie man den Abstand zwischen zwei Tönen, genannt
\emph{Intervall}, musikalisch sinnvoll angibt. Man mag sich fragen, wo hier die
Schwierigkeit ist: Wir können doch einfach Halbtonschritte zählen.  So beträgt
beispielsweise der Abstand zwischen es’ und h’ genau 8 Halbtonschritte.  Wieder
ist das Problem die Enharmonik: \emph{Tonhöhe ist nicht alles.} Eindrücklich
hört man das, wenn man die beiden Beispiele aus \cref{fig:enh} vergleicht: In
beiden Fällen endet man mit denselben zwei Tasten am Klavier, aber der
Höreindruck ist unterschiedlich.  Grund ist, dass unser Hirn Ausgangstöne
mitzählt: im ersten Beispiel sind fünf Ausgangstöne involviert, im zweiten
Beispiel hingegen sechs.

\begin{figure}[h]
  \centering
  \includegraphics{ly/s-enhContext}
  \caption{Zwei Beispiele, die bei denselben zwei Tasten am Klavier enden.  Im
    ersten Fall klingt die Kombination schräg, während sie im zweiten Fall
    stabil wirkt.}\label{fig:enh}
\end{figure}

\noindent Um dem gerecht zu werden, muss unsere Benennung so beschaffen sein,
dass aus ihr nicht nur die Anzahl an Halbtonschritten, sondern auch die Anzahl
an involvierten Ausgangstönen hervorgeht.%  Mit anderen Worten: Unsere Benennung
% muss \emph{enharmonisch präzise} sein.

\subsection{Grundintervalle}
\label{subsec:int}

Als erste Näherung wollen wir Abständen zwischen Ausgangstönen (also weißen
Tasten) Namen zuweisen.  Dies geht wie folgt: Man fixiert die untere der beiden
weißen Tasten, nennt sie die »erste« und nummeriert dann so lange nach
rechts durch, bis man bei der Zieltaste angekommen ist. Starte ich
beispielsweise beim c’, so ist das f’ die \emph{vierte} Taste.  Und nun macht
man das alles auf Latein – und erhält so die Namen der sogenannten
\emph{Grundintervalle}.\looseness-1

In \cref{tab:GrundI} sind die ersten acht Grundintervalle gelistet.  Und
plötzlich ergibt der Name »Oktave« auch Sinn: Der Abstand zwischen der ersten
und der \emph{achten} weißen Taste.

\begin{table}[h]
  \centering
  \begin{tabular}{cccccccc}
    \toprule
    1. & 2. & 3. & 4. & 5. & 6. & 7. & 8.\\% & 9.\\
    \midrule
    Prime & Sekunde & Terz & Quarte & Quinte & Sexte & Septime & Oktave\\% & None\\
    \bottomrule
  \end{tabular}
  % \begin{tabular}{ll}
  %   \toprule
  %   Zieltaste & Grundintervall\\
  %   \midrule
  %   1. & Prime\\
  %   2. & Sekunde\\
  %   3. & Terz\\
  %   4. & Quarte\\
  %   5. & Quinte\\
  %   6. & Sexte\\
  %   7. & Septime\\
  %   8. & Oktave\\
  %   9. & None\\
  %   \bottomrule
  % \end{tabular}
  \caption{Die ersten acht Grundintervalle, benannt nach der Nummer der Zieltaste.}\label{tab:GrundI}
\end{table}

Die Benennung überträgt sich auf alterierte Töne, indem der Abstand einfach von
den jeweiligen Ausgangstönen geerbt wird.  So ist das Grund\-intervall zwischen
es’ und h’ eine Quinte, während das Grundintervall zwischen dis’ und h’ eine
Sexte ist, passend zu \cref{fig:enh}.

\subsection{Rein – groß, klein – übermäßig, vermindert}

Leider sagen uns die Grundintervalle nur wenig über den tatsächlichen Abstand,
also die Anzahl an Halbtonschritten, die zwischen den Tönen liegen. So liegen
beispielsweise zwischen c’ und e’ vier Halbtonschritte, zwischen d’ und f’ aber
drei; doch in beiden Fällen ist das Grundintervall eine Terz. Die notwendige
Präzisierung erfolgt in drei Schritten:

\begin{enumerate}
% \item Da sich das »Muster« alle acht weiße Tasten wiederholt, genügt es, im
%   Folgenden nur die Grundintervalle Prime bis Septime zu betrachten. Die
%   Überlegungen übertragen sich auf andere Grundintervalle, indem man
%   Oktaven vergisst; so »funktionieren« Nonen genau wie Sekunden.
\item Primen, Quarten, Quinten und Oktaven klingen nur in einer Ausprägungsform
  wirklich gut\footnote{Das hat wieder etwas mit den »einfachen Verhältnissen«
    1\,:\,1, 3\,:\,4, 2\,:\,3 und 1\,:\,2 zu tun.} und kommen in der Musik auch
  unverhältnismäßig häufiger in dieser Form vor, nämlich 0\,\acr{HTS},
  5\,\acr{HTS}, 7\,\acr{HTS} und 12\,\acr{HTS}. In diesen Fällen nennt man sie
  \emph{rein}. So ist beispielsweise c’\,:\,g’ eine reine Quinte, aber h\,:\,f’
  nicht, denn letztere hat nur 6\,\acr{HTS}.
\item Sekunden, Terzen, Sexten und Septimen kommen in der klassischen Musik
  ungefähr gleich häufig in jeweils zwei Ausprägungsformen vor. Diese nennen wir
  \emph{groß} und \emph{klein}, wobei die kleine Variante stets einen
  Halbtonschritt weniger hat als die große. Die großen Varianten umfassen exakt
  2\,\acr{HTS}, 4\,\acr{HTS}, 9\,\acr{HTS} und 11\,\acr{HTS}.  So ist c’\,:\,e’
  eine große Terz, aber d’\,:\,f’ eine kleine Terz.\looseness-1
\item Ausnahmen werden wie folgt benannt: Ist das Intervall größer als rein
  bzw. als groß, so nennt man es \emph{übermäßig}; ist es kleiner als rein
  bzw. als klein, so nennt man es \emph{vermindert}.  So ist beispielsweise
  es’\,:\,h’ eine übermäßige Quinte und e’\,:\,ges’ eine verminderte
  Terz.\looseness-1
\end{enumerate}

\noindent Man notiert solche Intervalle, indem man die Grundintervalle durch die
entsprechende Zahl abkürzt (also z.\,B. \rb{5} für »Quinte«) und ihr dann je
nach exakter Größe die Buchstaben \rb{r}, \rb{g}, \rb{k}, \rb{ü} oder \rb{v}
voranstellt.  Beispielsweise bezeichnet \rb{g2} eine große Sekunde.
So ergibt sich \cref{tab:ints}.\marginpar{2.1–2}

\begin{table}[h]
  \centering
  \begin{tabular}{cccccccccccccccc}
    \toprule
    \textminus\kern.2pt 1 & 0 & 1 & 2 & 3 & 4 & 5 & 6 & 7 & 8 & 9 & 10 & 11 & 12 & 13\\
    \midrule
    v1 & r1 & ü1\\
       & v2 & k2 & g2 & ü2\\
       &    &    & v3 & k3 & g3 & ü3\\
       &    &    &    &    & v4 & r4 & ü4\\
       &    &    &    &    &    &    & v5 & r5 & ü5\\
       &    &    &    &    &    &    &    & v6 & k6 & g6 & ü6\\
       &    &    &    &    &    &    &    &    &    & v7 & k7 & g7 & ü7\\
       &    &    &    &    &    &    &    &    &    &    &    & v8 & r8 & ü8\\
    \bottomrule
  \end{tabular}
  \caption{Eine Aufschlüsselung der genannten Intervallnamen, sortiert nach
    ihrer exakten Größe in Halbtonschritten. \emph{Fun fact:} Eine v1 aufwärts
    geht real abwärts.}\label{tab:ints}
\end{table}

\begin{figure}[h]
  \centering
  \includegraphics{ly/s-intervalle}\\[-1px]
  \small \hspace*{5.4px}%
  k3\hspace*{21.2px}%
  g6\hspace*{22.3px}%
  k7\hspace*{23.0px}%
  r4\hspace*{22.4px}%
  v5\hspace*{22.0px}%
  ü8\hspace*{24.0px}%
  ü7\hspace*{22.2px}%
  v6\hspace*{-1.6px}%\hspace*{3px}
  \caption{Eine Reihe willkürlich gewählter Intervalle und ihre Namen.}
\end{figure}

\subsection{Zusammengesetzte Intervalle}
\label{subsec:zus}

Da Intervalle den Abstand zwischen einem Start- und einem Zielton beschreiben,
können wir zwei Intervalle aneinanderlegen, wenn der Zielton des ersten
Intervalls mit dem Startton des zweiten Intervalls übereinstimmt. Auf diese
Weise entsteht ein \emph{zusammengesetztes} Intervall. So können wir beispielsweise die
Intervalle c’\,:\,es’ (kleine Terz) und es’\,:\,as’ (reine Quarte) zum Intervall
c’\,:\,as’ (kleine Sexte) zusammensetzen.
% zwei Tönen beschreiben, sollte man sie auch addieren
% können.  Dabei soll sich die Anzahl der Halbtonschritte einfach mitaddieren.
% Wenn wir aber das aber enharmonisch korrekt tun wollen, müssen wir sagen, wie
% sich Grundintervalle addieren.  Dies geschieht wie folgt: Die Zieltaste des
% ersten Intervalls wird zur Starttaste des zweiten Intervalls; und dann zählt man
% einfach weiter.  Gehen wir also zuerst eine Terz nach oben und dann eine Quarte,
% so haben wir insgesamt bis 6 gezählt, und folglich ist das resultierende
% Intervall eine Sexte.

\begin{quote}
  Florian, 2015: »Und deshalb, liebe Kinder, ist 3\,+\,4\;=\;6!«
\end{quote}

\noindent Zwei Intervalle heißen zueinander \emph{komplementär}, wenn sie
gemeinsam eine Oktave zusammensetzen. So sind zum Beispiel die große Sekunde
(g2) und die kleine Septime (k7) zueinander komplementär. Man sagt dann auch, g2
ist das \emph{Komplementärintervall} der k7 (und umgekehrt).\marginpar{2.3}

\subsection{Größere Intervalle}

Um ein großes Intervall wie c’\,:\,des” zu benennen, geht man wie folgt vor: Wir zählen
weiter und landen bei dem Begriff »None«. Nun nutzen wir, dass sich dieses
Intervall aus einer Oktave (c’\,:\,c”) und einer Sekunde (c”\,:\,des”)
zusammensetzt. Wir vergessen die Oktave, kopieren die Größenbezeichnung der
Sekunde (»klein«) und nennen das Intervall somit »kleine None« (k9).

\subsection{Ganztonschritt, Hiatus, Tritonus}

%Zusätzlich zu obiger Benennung gibt es noch einige weitere Bezeichnungen:

\begin{enumerate}
\item Ein \emph{Ganztonschritt} ist dasselbe wie eine große Sekunde. Zwischen
  den meisten aufeinanderfolgenden weißen Tasten besteht ein Ganztonschritt, was
  die Benennung nahelegt. Dass er zwei Halbtonschritte umfasst, rechtfertigt nun
  endlich den Begriff »Halbtonschritt«.\looseness-1
\item Eine übermäßige Sekunde wird auch als \emph{Hiatus} bezeichnet.  Sie
  besteht wie die kleine Terz aus drei Halbtonschritten.
\item Und schließlich: der \emph{Tritonus}. Er setzt sich aus drei
  Ganztonschritten zusammen (daher der Name) und ist folglich dasselbe wie eine
  übermäßige Quarte.  Damit fasst er sechs Halbtonschritte, was klingend eine
  »halbe Oktave« ist (aber ziemlich schräg klingt).  Sein Komplementärintervall,
  die verminderte Quinte, hat genauso viele Halbtonschritte und wird daher auch
  oft als Tritonus bezeichnet.\marginpar{2.4}
\end{enumerate}

\subsection{Intervalle hören}

\begin{table}[b]
  \centering
  \begin{tabular}{ccccccccccccc}
    \toprule
    0 & 1 & 2 & 3 & 4 & 5 & 6 & 7 & 8 & 9 & 10 & 11 & 12\\
    \midrule
    r1 & k2 & g2 & k3 & g3 & r4 & trit. & r5 & k6 & g6 & k7 & g7 & r8\\
    \bottomrule
  \end{tabular}
  \caption{Standardbezeichnungen für kontextfreie klingende Intervalle.}\label{tab:inthoer}
\end{table}

Ohne musikalischen Kontext können wir natürlich nur den Abstand in
Halbtonschritten wahrnehmen. Wollen wir also zur Übung Intervalle hören, so
machen wir uns den Umstand zunutze, dass das Spektrum von 0 bis 12\,\acr{HTS}
bis auf den Tritonus eindeutig durch reine, kleine und große Intervalle
abgedeckt ist, und verwenden nur die in \cref{tab:inthoer} dargestellten
Begriffe.\marginpar{Hören}\looseness-1

\section{Skalen, Tonarten und der Quintenzirkel}
\label{sec:Tonart}

\subsection{Verwendung von Tönen in Stücken}
\label{subsec:tonst}

Betrachtet man konkrete klassische Musikstücke, so stellt man fest, dass nicht
alle Tonklassen gleich häufig vorkommen. Mehr noch: Es gibt sogar berühmte
Kompositionen, die über weite Strecken nur die weißen Tasten benötigen. Ein
eindrückliches Beispiel findet sich in \cref{fig:mozart}, wo jede schwarze Taste
direkt als Ausnahme herausstechen würde.  Es ist daher sinnvoll, für jedes Stück
einen »Standardvorrat« an verwendeten Tonklassen zu definieren und alle übrigen
Tonklassen als Ausnahmen zu betrachten.\looseness-1

\begin{figure}[h]
  \centering
  \includegraphics{ly/s-mozart}
  \caption{Die ersten sieben Takte aus Mozarts Sonate Nr.\,16 (\acr{KV}\,545).}\label{fig:mozart}
\end{figure}

\subsection{Heptatonische Skalen}

Die »Linien« in den Takten 5–7 aus \cref{fig:mozart} sind vor allem deshalb so
eingängig, weil man sich hier in Sekunden bewegt, also aus der Sicht der
Ausgangstöne »einen Fuß vor den nächsten« setzt.

Etwas genauer werden in \cref{fig:mozart} verschiedene Oktaven (a’\,:\,a”,
g’\,:\,g” und f’\,:\,f”) in sieben großen und kleine Sekunden »durchschritten«.
Dies nennt man eine \emph{heptatonische Skala}.  Allgemein formuliert: Eine
heptatonische Skala ist eine Abfolge von acht Tönen (durchnummeriert von 1 bis
8), bei der es von einem Ton zum nächsten jeweils um eine k2 oder g2 nach oben
geht und bei der der Abstand zwischen Ton 1 und Ton 8 eine r8 ist.

Eine kurze Rechnung zeigt, dass in einer heptatonischen Skala immer genau fünf
große Sekunden und zwei kleine Sekunden benötigt werden. Interessant ist nun,
wie diese verteilt sind. Hierfür genügt es, anzugeben, zwischen welchen der
durchnummerierten Töne die kleinen Sekunden liegen.  In \cref{fig:mozart}
begegnen uns schon drei verschiedene Fälle:
\begin{itemize}
\item In Takt 5 (a’ bis a”) liegen die beiden kleinen Sekunden zwischen dem
  2. und 3. sowie zwischen dem 5. und 6. Ton (schreibe 2–3 und 5–6).
\item In Takt 6 (g’ bis g”) liegen die beiden kleinen Sekunden zwischen dem
  3. und 4. sowie zwischen dem 6. und 7. Ton (schreibe 3–4 und 6–7).
\item In Takt 7 (f’ bis f”) liegen die beiden kleinen Sekunden zwischen dem
  4. und 5. sowie zwischen dem 7. und 8. Ton (schreibe 4–5 und 7–8).
\end{itemize}
Heptatonische Skalen lassen sich also bilden, indem man bei einer weißen Taste
startet und dann auf den weißen Tasten bis zur nächsten Oktave nach oben geht.
Dabei hängt die Verteilung der k2 davon ab, bei welcher weißen Taste man
startet.  Es ergeben sich daher ingesamt sieben Möglichkeiten, die \emph{Modi}
(Singular: \emph{Modus}) oder \emph{Kirchentonarten} genannt werden, siehe
\cref{tab:modi}.

\begin{table}[h]
  \centering
  \begin{tabular}{lccc}
    \toprule
    Name            & erste k2 & zweite k2 & Startton\\
    \midrule
    Ionisch (Dur)   & 3–4      & 7–8       & c\\
    Dorisch         & 2–3      & 6–7       & d\\
    Phrygisch       & 1–3      & 5–6       & e\\
    Lydisch         & 4–5      & 7–8       & f\\
    Mixolydisch     & 3–4      & 6–7       & g\\
    Aeolisch (Moll) & 2–3      & 5–6       & a\\
    Lokrisch        & 1–2      & 4–5       & h\\
    \bottomrule
  \end{tabular}
  \caption{Die sieben Modi für heptatonische Skalen, die Verteilung der kleinen
    Sekunde sowie der Startton, wenn man nur weiße Tasten nutzen
    will.}\label{tab:modi}
\end{table}

\subsection{Skalen mit beliebigem Startton}

Wollen wir für eine Skala nur weiße Tasten nutzen, so schreibt uns der Modus
vor, bei welchem Ton wir starten müssen. Wollen wir aber sowohl den Modus als
auch den Startton frei wählen, so müssen wir auf Alterationen zurückgreifen. So
sieht beispielsweise eine Moll-Skala auf e’ wie folgt aus:
\begin{quote}e’ fis’ g’ a’ b’ c” d” e”\end{quote}%
Dabei kann auch der Statton selbst alteriert sein. So sieht beispielsweise eine
Dur-Skala auf es wie folgt aus:
\begin{quote}es f g as b c’ d’ es’\end{quote}

\subsection{Tonart}

Wir wollen nun auf die Beobachtung aus \cref{subsec:tonst} zurückkommen:
Wie ermittelt man, welche Töne bei einem vorliegenden Stück zum »Standardvorrat«
gehören und welche Töne Ausnahmen sind? Was hat uns das Studium von Skalen
genutzt, wenn doch in einem Stück mehrere vorkommen?

Der entscheidende Punkt ist, dass in der klassischen Musik Stücke üblicherweise
ein klar erkennbares »Zentrum« haben, also einen Ton, der über weite Strecken
als Bezugspunkt wahrgenommen wird, auch \emph{Grundton} genannt. In
\cref{fig:mozart} ist es das c. Der »Standardvorrat« an Tönen ergibt sich nun
aus einer heptatonischen Skala, die auf dem Grundton aufbaut. Im Prinzip käme
hier jeder Modus in Frage, seit dem Frühbarock kommen aber Dur und Moll
unverhältnismäßig häufiger vor, sodass wir uns hierauf konzentrieren können.
Die Kombination aus Grundton und Modus (also Dur oder Moll) konstituiert
dann die \emph{Tonart} des Stücks. Im Fall von Dur notieren wir den Grundton
kapitalisiert, im Fall von Moll vollständig mit Kleinbuchstaben, also »Fis-Dur«,
aber »fis-Moll«. So steht das Beispiel aus \cref{fig:mozart} in C-Dur.

\subsection{Generalvorzeichen und Paralleltonarten}
\label{subsec:parT}

Steht ein Stück in G-Dur, so wird in ihm das fis deutlich häufiger vorkommen als
das f. Deswegen bietet es sich an, das fis als Generalvorzeichen wie in
\cref{Vorzeichen} zu notieren, um Tinte zu sparen. Zudem bringt diese
Vorgehensweise Hörempfinden und Notation noch näher zusammen, denn so sind die
überraschenden Ausnahmetöne genau diejenigen, die ein singuläres Vorzeichen
erhalten (z.\,B. ein f, das ein Auflösungszeichen
braucht).\looseness-1\marginpar{3.1}

Zwei Tonarten heißen zueinander \emph{parallel}, wenn sie dieselben
Generalvorzeichen haben. Auf diese Weise werden Paare gebildet, die immer aus
einer Dur- und einer Molltonart bestehen. Beispielsweise haben G-Dur und e-Moll
dieselben Generalvorzeichen und sind dementsprechend Paralleltonarten
voneinander. Die beiden einzigen Tonarten ohne Generalvorzeichen sind C-Dur und
a-Moll.  Man bemerke, dass die Grundtöne paralleler Tonarten stets um eine
kleine Terz versetzt sind.

Umgekehrt heißt das: Wenn man von einem Stück die Tonart bestimmen möchte,
lassen die Generalvorzeichen noch zwei Möglichkeiten zu. Es muss also noch der
Grundton gefunden werden. Ein guter Anhaltspunkt ist der Schlusston, denn am
Ende will man eigentlich zum Zentrum zurückgekehrt sein.  Ein Glück, dass
Mozarts Sonate nicht schon nach 7 Takten zuende ist!\marginpar{3.2–3}

\subsection{Quintenzirkel}

Abschließend wollen wir folgendes Problem lösen: Es ist ja ziemlich unpraktisch,
für jede Tonart nachrechnen zu müssen, welche Generalvorzeichen benötigt werden,
sodass ein einfaches Schema wünschenswert ist.

Hierzu ist es sinnvoll, sämtliche Tonklassen wie folgt anzuordnen: Man startet
mit dem c und geht nach rechts in Quinten aufwärts (g d a e …) und nach links in
Quinten abwärts (f b es as …). In beide Richtungen kann man beliebig lange
weitergehen (benötigt aber irgendwann mehrfache Alterationen), sodass eine
unendlich lange Kette entsteht. In dieser ist jedoch jedes 12. Glied \emph{bis
  auf Enharmonik} gleich, weswegen die Kette üblicherweise zu einem Kreis
»aufgerollt« wird (auch wenn dadurch Information verloren geht). Dieser Kreis
heißt \emph{Quintenzirkel} und ist ein Teil von \cref{fig:QZ}

Die Besonderheit ist nun, dass auf diese Weise die oben erwähnten Tonarten
geordnet werden: Baut man auf den entsprechenden Tönen eine Durskala auf, kommt
mit jedem Schritt im Kreis nach »unten« ein Generalvorzeichen hinzu: links vom c
Tiefalterationen und rechts vom c Hochalterationen.\footnote{Für einen Beweis
  verweisen wir auf den Kurs »Mathematische Musiktheorie«.} Das gleiche können
wir für Moll machen: Hier fangen wir bei a an, weil a-Moll die Molltonart ohne
Generalvorzeichen ist. Es ergeben sich zwei Kreise, die wir wie in \cref{fig:QZ}
ineinander zeichnen können. Zu jeder vollen Stunde gibt es also eine Dur- und
eine Molltonart mit den gleichen Generalvorzeichen (die dann per Definition
zueinander parallel sind).

So lassen sich nun die Generalvorzeichen einer Tonart leicht ablesen: Um
beispielsweise von C-Dur zu E-Dur zu gelangen, muss ich unterwegs fis, cis, gis
und dis als Generalvorzeichen mitnehmen.\marginpar{3.4}

\begin{figure}[h]
  \centering
  \begin{tikzpicture}[scale=1]
  \node at ({4.4*sin(15)},{4.4*cos(15)}) {\footnotesize+\,fis};
  \node at ({4.4*sin(45)},{4.4*cos(45)}) {\footnotesize+\,cis};
  \node at ({4.4*sin(75)},{4.4*cos(75)}) {\footnotesize+\,gis};
  \node at ({4.4*sin(105)},{4.4*cos(105)}) {\footnotesize+\,dis};
  \node at ({4.4*sin(135)},{4.4*cos(135)}) {\footnotesize+\,ais};
  \node at ({4.3*sin(165)},{4.3*cos(165)}) {\footnotesize+\,eis};
  \node at ({-4.4*sin(15)},{4.4*cos(15)}) {\footnotesize+\,b};
  \node at ({-4.4*sin(45)},{4.4*cos(45)}) {\footnotesize+\,es};
  \node at ({-4.4*sin(75)},{4.4*cos(75)}) {\footnotesize+\,as};
  \node at ({-4.4*sin(105)},{4.4*cos(105)}) {\footnotesize+\,des};
  \node at ({-4.4*sin(135)},{4.4*cos(135)}) {\footnotesize+\,ges};
  \node at ({-4.3*sin(165)},{4.3*cos(165)}) {\footnotesize+\,ces};
  \foreach \y in {0,...,5} {
    \pgfmathsetmacro\XA{84 - \y*30};
    \pgfmathsetmacro\XB{\XA -18};
    \pgfmathsetmacro\YA{83 - \y*30};
    \pgfmathsetmacro\YB{\YA -16};
    \pgfmathsetmacro\SA{96 + \y*30};
    \pgfmathsetmacro\SB{\SA +18};
    \pgfmathsetmacro\TA{97 + \y*30};
    \pgfmathsetmacro\TB{\TA +16};
    \centerarc[-to](0,0)(\XA:\XB:4);
    \centerarc[-to](0,0)(\YA:\YB:3);
    \centerarc[-to](0,0)(\SA:\SB:4);
    \centerarc[-to](0,0)(\TA:\TB:3);
  }
  \foreach \y in {0,...,11} {
    \draw ({3.3*sin(\y*30)},{3.3*cos(\y*30)}) -- ({3.7*sin(\y*30)},{3.7*cos(\y*30)});
  }
  \filldraw[white] (-.1,-3.28) rectangle (.1,-3.4);
  \filldraw[white] (-.1,-3.64) rectangle (.1,-3.72);
  \node at (0,4) {C};
  \node at (4,0) {A};
  \node at (0,-3.83) {Ges};
  \node at (0.0,-4.18) {Fis};
  \node at (-4,0) {Es};
  \node at (2,3.46) {G};
  \node at (3.46,2) {D};
  \node at (3.46,-2) {E};
  \node at (2,-3.46) {H};
  \node at (-2,3.46) {F};
  \node at (-3.46,2) {B};
  \node at (-3.46,-2) {As};
  \node at (-2,-3.46) {Des};
  \node at (0,3) {a};
  \node at (3,0) {fis};
  \node at (0,-2.83) {es};
  \node at (0,-3.17) {dis};
  \node at (-3,0) {c};
  \node at (1.5,2.6) {e};
  \node at (2.6,1.5) {h};
  \node at (2.6,-1.5) {cis};
  \node at (1.5,-2.6) {gis};
  \node at (-1.5,2.6) {d};
  \node at (-2.6,1.5) {g};
  \node at (-2.6,-1.5) {f};
  \node at (-1.5,-2.6) {b};
\end{tikzpicture}

  \caption{Dur- und Moll-Tonarten, angeordnet im Quintenzirkel. Auf 6 Uhr haben
    wir wie zuvor erwähnt enharmonisch blind »verklebt«.}\label{fig:QZ}
\end{figure}



% Hingegen heißt der möglichst fließende und kunstvolle Wechsel der Tonart im
% Verlauf eines Stückes \emph{Modulation}. Es gibt verschiedene Techniken, so
% etwas zu erreichen, von denen wir einige besprechen werden.

\section{Dreiklänge, Stufen und Funktionen}

Wir wollen nun mehrere Töne gleichzeitig erklingen lassen und fragen uns, welche
Kombination (auch \emph{Akkord} genannt) gut klingt.

Zunächst stellen wir fest, dass zur Beantwortung dieser Frage wegen der
Oktavidentität eigentlich nur Tonklassen relevant sind (beispielsweise klingt
c\,:\,e” genauso gut wie c’\,:\,e’).  Im nächsten Schritt könnten wir einzelne
Intervalle auf ihren Wohlklang untersuchen, was uns zu der Unterscheidung in
\emph{konsonante} und \emph{dissonante} Intervalle führen würde.

Wir können aber die Diskussion ein bisschen abkürzen: Aus Gründen™ klingt eine
Kombination aus mehreren Tönen »stabil«, wenn die beteiligten Tonklassen
Bestandteile eines Dur- oder Moll-Dreiklangs sind.\footnote{In reiner Stimmung
  hat das etwas mit den Primzahlen 3 und 5 zu tun.} Wir müssen also zunächst
definieren, was wir damit meinen.

\subsection{Dreiklänge durch Terzschichtung}

Schichtet man auf einem gegebenen Ton zwei Terzen, so bildet sich ein
\emph{Dreiklang}. Hier gibt es vier Möglichkeiten, siehe \cref{tab:Drei}.

\begin{table}[h]
  \centering
  \begin{tabular}{cll}
    \toprule
    Terzschichtung & Name                    & Beispiel\\
    \midrule
    k3\,+\,k3      & Verminderter Dreiklang  & d\,:\,f\,:\,as\\
    k3\,+\,g3      & Molldreiklang           & d\,:\,f\,:\,a\\
    g3\,+\,k3      & Durdreiklang            & d\,:\,fis\,:\,a\\
    g3\,+\,g3      & Übermäßiger Dreiklang   & d\,:\,fis\,:\,ais\\
    \bottomrule
  \end{tabular}
  \caption{Möglichkeiten, Dreiklänge durch das Schichten von Terzen zu
    bilden.}\label{tab:Drei}
\end{table}

\noindent Die Bezeichnungen \emph{Dur-} bzw. \emph{Molldreiklang} fußen auf der
Tatsache, dass die Dreiklänge aus dem jeweils 1., 3. und 5. Ton einer Dur-
bzw. Mollskala bestehen. Man benennt sie genau so wie die entsprechende
Tonart. So ist d’\,:\,fis’\,:\,a’ ein D-Dur-Dreiklang und d’\,:\,f’\,:\,a’ ein
d-Moll-Dreiklang.

Übermäßige und verminderte Dreiklänge klingen sehr instabil (vor allem, weil sie eine
v5 oder ü5 enthalten), weswegen sie nur in Ausnahmefällen anzutreffen sind. Wir
beschränken uns daher auf Dur- und Molldreiklänge.

Oft interessieren uns bei einem Dreiklang nur die beteiligten Tonklassen.  Ein
D-Dur-Dreiklang besteht beispielsweise für uns aus \emph{irgendeinem} d,
\emph{irgendeinem} fis und \emph{irgendeinem} a – oder sogar mehreren.


\subsection{Verwandtschaften zwischen Dreiklängen}

Wir wollen ein erstes Kriterium für »Nähe« zwischen zwei \mbox{(Dur- und
Moll-)} Dreiklängen einführen.  Hierfür untersuchen wir, ob die zu vergleichenden
Dreiklänge Töne gemeinsam haben – dies ist besonders praktisch, weil dann Töne
»liegenbleiben« können, wenn zwischen ihnen gewechselt wird.

Wir nennen daher Dreiklänge \emph{verwandt}, wenn sie zwei von drei Tonklassen
gemeinsam haben. Hier gibt es drei Möglichkeiten:
\begin{itemize}
\item Zwei Dreiklänge nennen einander \emph{Parallelklänge}, wenn sie sich
  eine große Terz teilen. So sind zum Beispiel F-Dur und d-Moll ihre jeweiligen
  Parallelklänge, weil sie sich die große Terz f\,:\,a teilen.\footnote{Zwei
    Dreiklänge sind genau dann zueinander parallel, wenn ihre entsprechenden
    Tonarten zueinander parallel im Sinne von \cref{subsec:parT} sind; dies
    rechtfertigt die Benennung.}
\item Zwei Dreiklänge nennen einander \emph{Gegenklänge} (seltener
  \emph{Gegenparallelklänge}), wenn sie sich eine kleine Terz teilen.  So sind
  F-Dur und a-Moll ihre jeweiligen Gegenklänge, weil sie sich die kleine Terz
  a\,:\,c teilen.
\item Zwei Dreiklänge nennen einander \emph{Variantklänge}, wenn sie sich eine
  Quinte teilen. So sind zum Beispiel F-Dur und f-Moll ihre jeweiligen
  Variantklänge, weil sie sich die Quinte f\,:\,c teilen.
\end{itemize}
In den ersten beiden Fällen nennt man die beteiligten Dreiklänge aus
naheliegenden Gründen auch \emph{terzverwandt}. Terzverwandte Dreiklänge klingen
einander näher als Variantdreiklänge, obwohl sich bei letzteren der Grundton
nicht ändert. Dies liegt daran, dass die Töne terzverwandter Dreiklänge in einer
gemeinsamen heptatonischen Skala vorkommen können, was bei Variantklängen nicht
möglich ist.\marginpar{Netz}

\subsection{Stufen und Funktionen in Dur}
\label{subsec:SFDur}

Wir fragen uns nun, welche Dreiklänge im Rahmen eines gegebenen Musikstücks
plausibel eingesetzt werden können. Die Antwort hierauf ist einfach: Am
naheliegendsten sind diejenigen Dreiklänge, die mit den Tönen der durch die
Tonart festgesetzten Skala gebildet werden.

Wir wollen dies zunächst im Falle von Dur durchspielen. Hierzu bauen wir auf
jedem Skalenton (wobei wir nur 1 bis 7 betrachten, weil 8 wieder wie 1
funktioniert) einen Dreiklang aus skaleneigenen Tönen auf. Diese Dreiklänge
nennt man auch \emph{Stufen} und nummeriert sie mit römischen Zahlen von
\textsc{i} bis \textsc{vii} durch.  In \cref{fig:stufenDur} ist das beispielhaft
in C-Dur durchexerziert. Die dortigen sieben Dreiklänge sind also diejenigen,
die im Kontext von C-Dur am naheliegendsten sind.\footnote{Anders gesagt: Dies
  sind genau die Dreiklänge, die mit weißen Tasten gebildet werden.}

\begin{figure}[h]
  \centering
  \includegraphics{ly/s-stufen-dur}\\[-3px]
  \textsc{\hspace*{10px}%
    i\hspace{26.3px}%
    ii\hspace{22.8px}%
    iii\hspace{21.5px}%
    iv\hspace{23.7px}%
    v\hspace{24.3px}%
    vi\hspace{20.5px}%
    vii\hspace*{-2px}%
  }\par{}
  \small
  \hspace*{10px}%
  T\hspace{23.3px}%
  Sp\hspace{20.8px}%
  Dp\hspace{22.5px}%
  S\hspace{26.2px}%
  D\hspace{23.3px}%
  Tp\hspace{32.0px}%
  \hspace{-2px}%
  \caption{Dreiklänge, die aus den Tönen einer C-Dur-Skala bestehen, und ihre
    unten erläuterten Funktionsbezeichnungen.}\label{fig:stufenDur}
\end{figure}


\noindent Hierbei sehen wir, dass die Dreiklänge \textsc{i}, \textsc{iv},
\textsc{v} Durakkorde sind, die Dreiklänge \textsc{ii}, \textsc{iii},
\textsc{vi} Mollakkorde sind, und der Dreiklang \textsc{vii} vermindert ist. Die
drei Durstufen sind die wichtigsten; man nennt sie auch
\emph{Hauptfunktionen}:\looseness-1
\begin{itemize}
\item Der Durakkord \textsc{i} heißt \emph{Tonika} (T) und bildet das »Zentrum«
  des Stücks, zu dem man in der Regel spätestens am Ende zurückkehren möchte.
\item Der Durakkord \textsc{v} heißt \emph{Dominante} (D) und bildet den
  »Gegenspieler« zur Tonika, strebt aber dennoch zu ihr zurück.
\item Der Durakkord \textsc{iv} heißt \emph{Subdominante} (S) und »durchbricht«
  quasi als dritter Weg den Gegensatz zwischen Tonika und Dominante.
\end{itemize}
Wir bemerken, dass die Stufen \textsc{vi}, \textsc{ii} bzw.\ \textsc{iii} genau
die parallelen Molldreiklänge von Tonika, Subdominante bzw. Dominante sind, und
nennen sie daher \emph{Tonika-}, \emph{Subdominant-} und
\emph{Dominantparallele} (Tp, Sp, Dp). Sie bilden die
\emph{Nebenfunktionen}. Die verminderte Stufe \textsc{vii} wird vorerst
vernachlässigt.
\enlargethispage{\baselineskip}

Obige Diskussion ist kein Spezifikum von C-Dur, sondern funktioniert in jeder
Durtonart (weil die Verteilung von kleinen und großen Terzen innerhalb einer
Skala nur vom Modus abhängt).  So können wir beispielsweise von der
Subdominantparallele in A-Dur sprechen: sie besteht aus den Tönen h, d und fis,
und ist folglich ein h-Moll-Dreiklang.

% \begin{table}[h]
%   \centering
%   \begin{tabular}{clcl}
%     \toprule
%     Stufe       & Name         & Symbol & Charakter\\
%     \midrule
%     \textsc{i}  & Tonika       & T      & Tonales Zentrum\\
%     \textsc{iv} & Subdominante & S      & Entfernung vom Zentrum\\
%     \textsc{v}  & Dominante    & D      & Spannung zum Zentrum\\
%     \bottomrule
%   \end{tabular}
%   \caption{Hauptfunktionen in Dur}\label{tab:HF}
% \end{table}

\subsection{Leitton}

Bevor wir eine ähnliche Konstruktion in Moll durchführen können, müssen wir noch
über einen Ton sprechen: Der Ton, der eine kleine Sekunde (k2) unter
dem Grundton liegt (unabhängig davon, ob er skaleneigen ist oder nicht), heißt
\emph{Leitton}. Er hat eine starke Strebetendenz aufwärts zum Grundton (»man hat
den Gipfel schon fast erreicht«) und ist daher ein wichtiges Mittel, um auf ein
Ende hinzuarbeiten.  Die Dominante einer Durtonart strebt nicht zuletzt deswegen
zurück zur Tonika, weil sie den Leitton als Terz enthält.\looseness-1

\subsection{Stufen und Funktionen in Moll}
\label{subsec:SFMoll}

Wir wollen nun \cref{subsec:SFDur} auf Moll übertragen. Soll dabei die Dominante
wie in Dur zur Tonika streben, müssen wir die Terz der Stufe \textsc{v} küstlich
hochalterieren, wodurch der sich der Dreiklang vom Moll- zum
Durakkord verändert. Es ergeben sich die in \cref{fig:SMoll} dargestellten Dreiklänge.

\begin{figure}[h]
  \centering
  \includegraphics{ly/s-stufen-moll}\\[-2px]
  \textsc{\hspace*{12.1px}%
    i\hspace{26.6px}%
    ii\hspace{22.7px}%
    iii\hspace{21.5px}%
    iv\hspace{24.3px}%
    v\hspace{24.3px}%
    vi\hspace{23.8px}%
    vii\hspace*{.5px}\hspace{1px}%
  }\par{}\small
  \hspace*{12.1px}
    t\hspace{24.4px}%
    ~\hspace{31.3px}%
    tP\hspace{25.8px}%
    s\hspace{27.1px}%
    D\hspace{23.9px}%
    sP\hspace{24.6px}%
    dP\hspace{3.9px}\hspace{1px}%
  \caption{Die sieben Stufen in Moll und ihre Funktionsbezeichnungen.}\label{fig:SMoll}
\end{figure}

\noindent Wieder bezeichnen wir die Stufen \textsc{i}, \textsc{iv}, \textsc{v}
als \emph{Hauptfunktionen} und nennen sie \emph{Tonika}, \emph{Subdominante} und
\emph{Dominante}. Da aber Tonika und Subdominante Mollakkorde sind, schreiben wir
t und s – für die Dur-Dominante jedoch D.
\enlargethispage{\baselineskip}

Die Stufen \textsc{iii}, \textsc{vi}, \textsc{vii} sind wieder
Parallelklänge, nämlich der Tonika (t), der Subdominante (s) – und der
\emph{Moll}dominante (die dann ja d heißt!). Sie heißen daher
\emph{Tonika-}, \emph{Subdominant-} und \emph{Dominantparallele} (tP, sP,
dP) und bilden die \emph{Nebenfunktionen}. Die verminderte Stufe \textsc{ii}
wird vorerst vernachlässigt.

\subsection{Tonvorrat in Moll}
\label{subsec:MollMehr}

\Cref{subsec:SFMoll} zeigt, dass der Standardtonvorrat in Moll nicht nur aus den
Tönen der Skala besteht, sondern zusätzlich noch die hochalterierte 7
enthält. Es geht sogar noch einen Schritt weiter: Möchte man den Leitton
schrittweise von unten erreichen, ohne dabei einen Hiatus (also eine ü2) zu
erzeugen, so muss man zwangsläufig auch noch den 6. Ton hochalterieren.

Tatsächlich ist es daher in hinreichend spannenden klassischen Stücken durchaus
üblich, dass sowohl der 6. als auch der 7. Skalenton in zwei Al\-te\-rationsformen
vorkommt, siehe \cref{fig:wo}. Dabei gilt als grobe Regel:\looseness-1
\begin{enumerate}
\item Läuft man von unten auf den Grundton zu, so will man zum Grundton
  »hinführen« und nutzt deswegen den Leitton und zur Hiatusvermeidung eben auch
  die hochalterierte 6 (so wie im dritten Takt zu sehen).
\item Läuft man vom Grundton aus eine längere Linie nach unten, so nutzt man
  skaleneigene Töne (so wie im vierten Takt zu sehen).
\end{enumerate}
  
\begin{figure}[h]
  \centering
  \includegraphics{ly/s-wo}
  \caption{Die Takte 143–148 aus dem Eingangschor der 1. Kantate des
    Weihnachtsoratoriums von Bach (\acr{BWV\,248}). Das Stück steht eigentlich in
    D-Dur, den abgebildeten Ausschnitt können wir aber auch im Kontext von
    h-Moll interpretieren. Im 3. abgebildeten Takt tauchen Töne 6 und 7 in
    hochalterierter Form (gis und ais) auf, im 4. Takt tauchen die Töne 6 und 7 in ihrer
    Ursprungsform (a und g) auf.}\label{fig:wo}
\end{figure}

\noindent Es gibt noch eine andere Möglichkeit, einen Hiatus zu vermeiden: Durch
einen Wechsel der Oktave. Dabei wird der Hiatus zu einer verminderten Septime,
die eingängiger ist (vermutlich weil sie gar nicht mehr den Anspruch
erhebt, ein »Schritt« zu sein). Ein Beispiel findet sich in \cref{fig:singet}.\marginpar{4.4}
  
\begin{figure}[h]
  \centering
  \includegraphics{ly/s-singet}
  \caption{Die Basslinie der Takte 177–179 aus der Motette »Singet dem Herrn ein
    neues Lied« (\acr{BWV\,225}) von Bach. Der Satz steht in B-Dur, dieser
    Ausschnitt ist aber leichter in d-Moll zu lesen. Von cis zu b
    wird eine v7 nach oben gesprungen.}\label{fig:singet}
\end{figure}

% Manchmal werden die oben beschriebenen drei Situationen als verschiedene
% „Moll\-skalen“ bezeichnet. Da jedoch in einem Stück sämtliche Situationen
% auftreten können, scheint es uns sinnvoller, von einem großen Tonvorrat zu
% sprechen, aus dem je nach Kontext geschöpft wird.

\subsection{Grundfunktionen im Quintenzirkel}

Die drei Haupt- und drei Nebenfunktionen einer Tonart (sowohl in Dur als auch in
Moll) bilden zusammen die \emph{Grundfunktionen}. Im Quintenzirkel liegen sie sehr
nah beieinander, wie \cref{fig:KadR} zeigt. Dieser Ausschnitt des Quintenzirkels
nennt sich \emph{harmonischer Rahmen} der Tonart.

\begin{figure}[h]
  \centering
  \tikzstyle{very densely dashed}=[dash pattern=on 1.8pt off 1.3pt]
  \begin{tikzpicture}
    \node at (0,4) {T};
    \node at (-2,3.46) {S};
    \node at (2,3.46) {D};
    \node at (0,3) {Tp};
    \node at (-1.5,2.6) {Sp};
    \node at (1.5,2.6) {Dp};
    \draw (0,3.32) -- (0,3.75);
    \draw (1.64,2.84) -- (1.87,3.22);
    \draw (-1.68,2.91) -- (-1.87,3.22);
    \centerarc[-to](0,0)(85:65:4);
    \centerarc[-to](0,0)(95:115:4);
    \centerarc[-to](0,0)(83:68:3);
    \centerarc[-to](0,0)(97:113:3);
    \centerarc[very densely dashed](0,0)(44:56:4);
    \centerarc[very densely dashed](0,0)(124:136:4);
    \centerarc[very densely dashed](0,0)(44:54:3);
    \centerarc[very densely dashed](0,0)(126:136:3);
  \end{tikzpicture}
  \hspace*{.7cm}
  \begin{tikzpicture}
    \node at (0,4) {tP};
    \node at (-2,3.46) {sP\kern-.8px/\kern-.1pxtG};
    \node at (2,3.46) {dP};
    \node at (0,3) {t};
    \node at (-1.5,2.6) {s};
    \node at (1.5,2.6) {[D]};
    \draw (0,3.3) -- (0,3.7);
    \draw (1.66,2.875) -- (1.87,3.22);
    \draw (-1.62,2.805) -- (-1.87,3.22);
    \centerarc[-to](0,0)(85:65:4);
    \centerarc[-to](0,0)(95:111:4);
    \centerarc[-to](0,0)(86:68:3);
    \centerarc[-to](0,0)(94:115:3);
    \centerarc[very densely dashed](0,0)(44:56:4);
    \centerarc[very densely dashed](0,0)(126:136:4);
    \centerarc[very densely dashed](0,0)(44:54:3);
    \centerarc[very densely dashed](0,0)(124:136:3);
  \end{tikzpicture}
  \caption{Der harmonische Rahmen in Dur (links) und Moll (rechts), verortet im
    Quintenzirkel. Nur bei der Dominante in Moll haben wir ein bisschen
    gemogelt.}\label{fig:KadR}
\end{figure}


\subsection{Varianten und Gegenklänge als Funktionen}

Taucht in einer Tonart die Variante einer Funktion auf, so kann man dies leicht
notieren, indem man die Groß-/Kleinschreibung ändert. Hier zwei Beispiele in
Dur, die die Subtilität dieser Konvention verdeutlichen:
\begin{itemize}
\item Die \emph{Tonikavariantparallele} (also die Parallele der Variante der Tonika)
  wird als tP notiert.  In G-Dur wäre das ein B-Dur-Dreiklang.
\item Die \emph{Tonikaparallelvariante} (also die Variante der Parallele der Tonika)
  wird als TP notiert. In G-Dur wäre das ein E-Dur-Dreiklang.
\end{itemize}
Solche Funktionen sind allerdings sehr selten.  Eine Fall, der tatsächlich
recht häufig vorkommt, ist die Dur-Variante $\muT$ am Ende eines Stücks, das in
Moll steht. Die darin enthaltene große Terz wird \emph{picardische Terz}
genannt. Dieser Klang hat einen »erlösenden« Charakter.

Darüber hinaus könnte man einige Stufen auch als Gegenklänge anderer Stufen
auffassen, zum Beispiel ist in Dur die Tonikaparallele (Tp) dasselbe wie der
Subdominantgegenklang (Sg). In der Praxis ist das nur an einer Stelle relevant:
Die sP in Moll ist oft ein »Ersatz« für die Tonika und wird daher auch als tG,
also als \emph{Tonikagegenklang}, bezeichnet.\marginpar{4.1}

% \pagebreak
\subsection{Kadenzen}
\label{subsec:Kadenzen}

Eine \emph{Kadenz} ist eine Akkordverbindung, die zur Schlussbildung verwendet
wird. In ihrer Grundform lautet sie D\,–\,T in Dur und D\,–\,t in Moll. Es
gibt nun verschiedene Erweiterungen und Abwandlungen:
\begin{itemize}
\item Endet eine Kadenz nicht wie erwartet in der Tonika, spricht man von einem
  \emph{Trugschluss}. In der Melodie will man aber dennoch die Abfolge »Leitton
  – Grundtton« hören, weswegen der »Ersatz« den Grundton
  enthalten sollte. Daher sind die üblichen Formen des Trugschlusses D\,–\,Tp in
  Dur und D\,–\,tG in Moll; in Stufen beide Male \textsc{v}\,–\,\textsc{vi}.
\item Oft taucht eine längere Version der Kadenz auf, die sich
  \emph{Vollkadenz} nennt und die Form T\,–\,S\,–\,D\,–\,T in Dur
  bzw. t\,–\,s\,–\,D\,–\,t in Moll hat.
\item Zuweilen taucht auch die Schlussbildung S\,–\,T bzw. s\,–\,t auf, die
  \emph{plagale Kadenz} heißt. Sie klingt wie ein »Abstieg« zum
  Ausgangspunkt.\marginpar{4.2–3}
\end{itemize}

\section{Struktur von Notensätzen}

Nachdem wir uns in den vorangegangenen Abschnitten eine Sprache zur Beschreibung
von Tönen und Harmonien erarbeitet haben, wollen wir nun \emph{in} dieser
Sprache Methoden erarbeiten, um Notensätze zu schreiben.

Der Blickwinkel, den wir dabei einnehmen, ist folgender: Würden wir einen
Zufallsgenerator ein mehrstimmiges Stück schreiben lassen, so wäre das Ergebnis
vermutlich sehr chaotisch, selbst wenn wir ihn auf eine Skala festlegen. Aufgabe
dieses Abschnitts ist es daher, einige recht strenge Regeln zu definieren, die
dieses Chaos eindämmen und das Zufallsprodukt Stück für Stück mehr nach
klassischer Musik klingen lassen.

Um Einwänden der Form »Aber Bach macht das ganz anders!« vorzubeugen: Erstens
werden wir in den Folgeabschnitten unser Repertoire an Möglichkeiten erweitern,
wodurch einige Regeln wieder gelockert werden.  Zweitens können selten alle
Regeln vollständig berücksichtigt werden, sodass Abwägungen getroffen werden
müssen. Und drittens ist Musik in letzter Instanz \emph{Kunst} und keine formale
Sprache. Dies vorweggeschickt: Lasst uns beginnen, das Chaos zu zähmen!

\subsection{Melodien}

Wir fangen klein an und wollen unseren imaginären Zufallsgenerator einstimmige
Stücke schreiben lassen – also \emph{Melodien}. Wie nach den vorherigen
Abschnitten zu erwarten ist, wird das gröbste Chaos dadurch beseitigt, dass man
eine Tonart festlegt und fortan nur noch skaleneigene Töne verwendet. In Moll
gibt es die in \cref{subsec:MollMehr} angesprochenen harmonisch begründeten
zusätzlichen Töne, die aber nur wie dort beschrieben verwendet werden.

Über diese Grundregel hinaus ist das Schreiben ausgewogener Melodien eine
Wissenschaft für sich und extrem individuell.  Wir können daher nur einige
»Faustregeln« angeben, von denen wir empirisch behaupten, dass ihre Beachtung
Melodien zugänglicher (z.\,B. singbarer) macht.

Bevor wir diese Regeln formulieren, wollen wir die Möglichkeiten von
\emph{Bewegungen} in einer Stimme (d.\,h. Wechsel zur nächsten Note) qualitativ
beschreiben: Eine Bewegung hat eine \emph{Richtung} (Steigen, Fallen,
Liegenbleiben) und eine \emph{Größe}, bei der wir hier sehr grobgranular nur
zwischen \emph{Schritt} (kleine oder große Sekunde) und \emph{Sprung} (größer)
unterscheiden wollen.

Dieses Vokabular reicht aus, um die versprochenen »Faustregeln« für die
Gestaltung von Melodien zu formulieren:
\begin{itemize}[itemsep=2px]
\item Kleine Bewegungen sind zu bevorzugen (»sparsame Stimmführung«): Also
  möglichst viele Schritte, möglichst wenige Sprünge.
\item Oktavsprünge sind okay; größere Sprünge sind zu vermeiden. Sprünge, die
  größer als eine Quinte sind, sollten individuell geprüft werden.
\item Große Sprünge sollten durch sich an sie anschließende Schritte in die andere
  Richtung »abgefedert« werden.
\item Man sollte sich nur in reinen, kleinen und großen Intervallen
  bewegen. Eine Ausnahme bildet der \emph{chromatische Halbtonschritt} ü1
  (z.\,B.  c–cis).
\end{itemize}

\subsection{Chorsatz}

Nun wollen wir \emph{mehrstimmige} Stücke schreiben, also Stücke, die mehreren
Stimmen Anweisungen geben, gleichzeitig zu erklingen und dabei eigenständige
Melodien zu produzieren, die zueinander passen.
\enlargethispage{\baselineskip}

Wir wollen uns im Rahmen des Kurses auf den \emph{vierstimmigen gemischten
  Chorsatz} beschränken und alle anderen Formen als Abwandlungen davon
betrachten. Wie der Name vermuten lässt, sind hieran vier Gesangsstimmen
beteiligt, deren vertikale Reihenfolge – bis auf wenige Ausnahmen, die wir
\emph{Stimmkreuzung} nennen – gleich bleibt. Diese heißen (von hoch nach tief)
\emph{Sopran}, \emph{Alt}, \emph{Tenor} und \emph{Bass}. Ihre Stimmumfänge (also
die Bereiche, in denen sie gut singen können) finden sich in
\cref{tab:Tonumfaenge}.

\begin{table}[h]
  \centering
  \setlength{\belowrulesep}{1pt}
  \renewcommand{\arraystretch}{1.3}
  \begin{tabular}{lll}
    \toprule
    Stimme & Beginn & Ende\\
    \midrule
    Sopran & c’     & g”\\
    Alt    & g      & d”\\
    Tenor  & d      & g’\\
    Bass   & F      & e’\\
    \bottomrule
  \end{tabular}
  \caption{Die groben Tonumfänge verschiedener Chorstimmen im
    Standardchorsätzen. (Viele Sänger:innen können auch über diesen Rahmen
    hinaus singen.)}\label{tab:Tonumfaenge}
\end{table}

In der sogenannten \emph{kompakten Notation} werden diese vier Stimmen in zwei
Systeme eingetragen, wobei die jeweils obere Stimme (Sopran bzw. Tenor) nach
oben »gehalst« wird, die anderen (Alt und Bass) nach unten. Auf diese Weise
können auch unterschiedliche Rhythmen in den einzelnen Stimmen notiert werden,
siehe \cref{fig:palest}.

\begin{figure}[h]
  \centering
  \includegraphics{ly/s-palestrina}\\
  \caption{Die Takte 19–23 als Palestrinas \emph{Sicut cervus} (Psalm 42). Im
    oberen System stehen Sopran und Alt, im unteren Tenor und Bass.}\label{fig:palest}
\end{figure}
\enlargethispage{\baselineskip}
\subsection{Grundregeln im vierstimmigen Chorsatz}

Alles klar: Wir wissen nun, welche Regeln beim Schreiben von Einzelstimmen zu
beachten sind und in welchen Bereichen sich die vier Chorstimmen aufhalten.
Damit können wir doch unseren Zufallsgenerator beauftragen.

Mitnichten! Auch wenn es in obigem Beispiel nicht so aussieht, sind die Stimmen
aufeinander abgestimmt. Wir wollen einige Regeln kennenlernen, beschränken uns
dabei aber auf den \emph{homophonen} Fall (d.\,h. alle Stimmen haben
den gleichen Rhythmus).  Folgende Regeln erledigen das Gröbste:

\begin{enumerate}
\item Alle Stimmen sind in derselben Tonart geschrieben. Gleichzeitig
  erklingenden Töne müssen Bestandteile eines Dreiklangs innerhalb des
  harmonischen Rahmens dieser Tonart sein. Dabei gilt:
  \begin{itemize}[nosep]
  \item Grundton und Terz müssen vorkommen, die Quinte darf fehlen.
  \item Jeder der Dreiklangstöne kann gedoppelt werden, meistens wird aber der
    Grundton verdoppelt.
  \item Im Bass darf der Grundton oder die Terz stehen; die Quinte aber nur in
    Ausnahmefällen. Ist es nicht der Grundton, wird ein Index an die Funktion
    geschrieben, z.\,B. $\muT_3$ (»Tonika mit Terz im Bass«).
  \end{itemize}
% \item Bei Regel 1 ist folgendes zu beachten: Grundton und Terz 
\item Die vertikalen Abstände zwischen Sopran und Alt sowie zwischen Alt und
  Tenor sollten nicht größer als eine Oktave sein.
\item Bei jedem Wechsel \emph{sollten} wenigstens zwei Bewegungsrichtungen
  (Steigen, Fallen, Liegen) vorkommen, damit die einzelnen Melodien unabhängiger
  klingen. Dies wird aber gerade bei Kadenzen oft
  missachtet.\looseness-1\marginpar{5.1¹}
% \item Bis auf noch zu erläuternde Ausnahmen ergeben die Stimmen einen Dreiklang
%   innerhalb des harmonischen Rahmens der jeweiligen Tonart ist. Dabei darf die
%   Quinte fehlen.
% \item Grundton und Terz dürfen verdoppelt werden, die Quinte nicht.
% \item Im Bass darf der Grundton oder die Terz stehen; die Quinte hingegen nur
%   in Ausnahmefällen, etwa wenn sie „auf dem Weg“ liegt. Oft möchten wir bei
%   der Funktionsbezeichnung notieren, welcher Dreiklangston im Bass steht. Dies
%   geschieht als Index, z.\,B. $\muT_3$.
\end{enumerate}

\subsection{Quint- und Oktavparallelen}

Beachtet man die Regeln des vorigen Abschnitts, so kann sich das Ergebnis schon
sehen (oder eher: hören) lassen.  Es gibt jedoch noch eine weitere sehr wichtige
Regel: Bewegen sich zwei Stimmen \emph{parallel} im Abstand von Quinten oder
Oktaven (was rein ausgestimmt die »einfachsten« Intervalle sind), so werden sie
nicht wirklich als eigenständig und unabhängig wahrgenommen. Deswegen sollten
sämtliche Wechsel vermieden werden, bei denen sich zwei der Stimmen in parallel
im Abstand von Quinten oder Oktaven bewegen.

Das ist allerdings leichter gesagt als getan. Als praxisnahes Beispiel mag
\cref{fig:parall} dienen: Dort sind alle Regeln beachtet, aber trotzdem hat sich
eine einzelne Quintparallele zwischen Alt und Tenor eingeschlichen.

\begin{figure}[h]
  \centering
  \includegraphics{ly/s-parallAT}
  \caption{Ein kurzer Chorsatz, an dem es wenig zu beanstanden gibt. Allerdings
    findet sich vom 3. zum 4. Schlag eine Quintparallele zwischen Alt und Tenor.}\label{fig:parall}
\end{figure}

\noindent Tatsächlich ist die Angelegenheit noch vertrackter: \Cref{fig:QPb} zeigt viele
Variationen von Quintparallelen, die ebenfalls stilfremd klingen und daher zu
vermeiden sind. In Worten sehen wir hier folgendes:
\begin{itemize}[nosep]
\item Parallelen, die durch zusätzliche Zwischennoten entstehen,
\item Parallelen, die nur unwesentlich unterbrochen werden (\emph{Akzentparallele}),
\item Parallelen, die bis auf Oktaven bestehen (\emph{Antiparallele}),
\item Wechsel von v5 nach r5 (\emph{vermindert\,–\,rein: das lass sein}).
\end{itemize}
\begin{figure}[h]
  \centering
  \includegraphics{ly/s-parallBad}\vspace*{-10px}
  \caption{Quintparallelen, die nicht so leicht als solche erkannt
    werden.}\label{fig:QPb}
\end{figure}%
 Dasselbe gilt für Oktavparallelen, mit einer Ausnahme:
\emph{Oktav}antiparallelen, bei denen sich die Stimmen wie in \cref{fig:QPb}:3
in verschiedene Richtungen bewegen, sind üblich, siehe \cref{fig:QPg}:4.\looseness-1

\pagebreak
\noindent Interessanterweise sind einige andere Parallelbewegungen, die sehr
ähnlich aussehen (siehe \cref{fig:QPg}), in der klassischen Musik durchaus
häufig anzutreffen.  In Worten sehen wir hier folgendes:
\begin{itemize}[nosep]
\item Quartparallelen (also »komplementäre« Quintparallelen),
\item Wechsel von r5 nach v5 (\emph{rein\,–\,vermindert: ungehindert}),
\item Parallelen aus zeitversetzten Wechseln (hier: e”\,–\,d” kommt vor a’\,–\,g’),
\item Oktavantiparallelen wie zuvor erwähnt.
\end{itemize}
Zur Begründung, warum diese Fälle weniger problematisch sind, können wir
leider kein besseres Argument als blanke Empirie anbieten.\marginpar{5.1²}

\begin{figure}[h]
  \centering
  \includegraphics{ly/s-parallOk}\vspace*{-5pt}
  \caption{Unproblematische Parallelbewegungen.}\label{fig:QPg}
\end{figure}

\subsection{Auflösung des Leittons}

In jeder nicht-plagalen Kadenz enthält der vorletzte Akkord (also die
Dur-Dominante) den Leitton und der letzte Akkord den Grundton der Tonart (im
Falle eines Trugschlusses ist er die Terz des Dreiklangs). Dabei möchte jede
Stimme, die den Leitton hat, gerne auch danach den Grundton singen.

Würde man also in einer Kadenz den Leitton verdoppeln, so würden Oktavparallelen
entstehen, wenn wir alle Stimmen zufriedenstellen. Deswegen darf der Leitton
nicht verdoppelt werden – ohne Ausnahme.

\enlargethispage{.5\baselineskip}
Liegt der Leitton in einer der Mittelstimmen, so ergibt sich bei seiner
korrekten Auflösung oft ein unvollständiger Schlussakkord (z.\,B. ohne
Quinte). Deshalb ist es üblich, in diesem Fall den Leitton in einen anderen Ton
zu führen, siehe \cref{fig:leitton}:2–3; dies nennt sich \emph{abspringender
  Leitton}. Eine Leittonverdopplung lässt sich dadurch allerdings nicht
rechtfertigen.\marginpar{5.1³–\textsc{p}}

\begin{figure}[h]
  \centering
  \includegraphics{ly/s-leitton}\vspace*{-4px}
  \caption{Eine reguläre Leittonauflösung und zwei abspringende Leittöne.}\label{fig:leitton}
\end{figure}

\section{Harmoniefremde Töne}

Im letzten Abschnitt haben wir strenge Regeln besprochen, die einen Chorsatz
klassisch und geordnet klingen lassen. Ab jetzt werden wir verschiedene
Möglichkeiten kennen lernen, diese Regeln kontrolliert aufzuweichen.

Vor allem werden wir Situationen besprechen, in denen zusätzlich zum
vorliegenden Dreiklang andere Töne erklingen können: zum Beispiel weil sie eine
Linie weiterführen oder bewusst »verzögern«, oder weil sie dem Akkord einen
besonderen Charakter (Wärme, Spannung) verleihen.

\subsection{Zwischennoten}

Es gibt die Möglichkeit, \emph{zwischen} den Schlägen in einzelnen Stimmen weitere
Töne unterzubringen. Dies geht auf zwei Weisen:

\begin{itemize}
\item Ein Terzsprung kann mithilfe einer \emph{Durchgangsnote} überbrückt werden:
  \begin{figure}[h]
    \centering
    \includegraphics{ly/s-durchgang}
    \caption{Ein Notenbeispiel mit etlichen Durchgangsnoten.}
  \end{figure}
\item Zwischen Tonwiederholungen kann eine \emph{Wechselnote}, also ein
  Schritt nach oben oder unten (und wieder zurück), eingebaut werden:
  \begin{figure}[h]
    \centering
    \includegraphics{ly/s-wechsel}
    \caption{Ein Notenbeispiel mit einigen Wechselnoten.}
  \end{figure}
\end{itemize}

\noindent Auf diese Weise können die Melodien der Einzelstimmen individueller
und singbarer gestaltet werden. Außerdem wird dadurch zumindest teilweise
erreicht, dass die Stimmen rhythmisch voneinander unabhängig werden; so löst man
sich ein bisschen von der starren Regel der Homophonie. Trotzdem solte man beim
Verwenden von Zwischennoten ein bisschen aufpassen:

\begin{itemize}
\item Der Chorsatz muss auch ohne die Zwischennoten »funktionieren«. Wir raten
  deshalb dazu, einen Satz zunächst ohne sie zu entwerfen und sie danach sparsam
  dort einzubauen, wo sie sich anbieten.
\item Tauchen in mehreren Stimmen gleichzeitig Zwischennoten auf, so müssen sie
  zueinander \emph{konsonant geführt} werden, d.\,h. sie müssen (bis auf
  Oktaven) eine Terz, Sexte oder Oktave bilden.
\item Durch Zwischennoten können Quint- und Oktavparallelen entstehen, die nicht
  leicht auffallen. Hier sollte man besonders aufpassen.\marginpar{6.2}
\end{itemize}

\subsection{Vorhalte}

Während Zwischennoten eine Möglichkeit darstellen, auf unbetonten Zeiten
harmoniefremde Töne zu platzieren, bieten Vorhalte eine Möglichkeit,
\emph{auf} der Zählzeit einen fremden Ton zu setzen.

Anstelle eines Dreiklangstones kann im Rahmen eines \emph{Vorhalts} in einer der
drei Oberstimmen der Ton einen Schritt darüber erklingen, sofern dieser danach
schrittweise abwärts in den »richtigen« Ton geführt wird.  Dies kann für alle
Dreiklangstöne\footnote{Warum 8 und nicht 1: Oft hat man den Grundton im Bass,
  der Vorhalt des Grundtons geschieht aber in einer anderen Stimme eine oder
  mehrere Oktaven darüber.} (3, 5 oder 8) geschehen; es gibt demnach
\emph{Quart-}, \emph{Sext-} und \emph{Nonvorhalte}. In \cref{fig:Vorhalt}
finden sich viele Beispiele.\vspace*{-2px}

\begin{figure}[h]
  \centering
  \includegraphics{ly/s-vorhalt}\vspace*{-3pt}
  \caption{Viele verschiedene Vorhalte. Im dritten Beispiel sind Quart- und
    Sextvorhalt zu einem Quartsextvorhalt kombiniert.}\label{fig:Vorhalt}
\end{figure}

\noindent Wie das dritte Beispiel zeigt, können auch mehrere Vorhalte kombiniert
werden.  Recht häufig ist zum Beispiel ein \emph{Quartsextvorhalt}, also ein
Quart- und ein Sextvorhalt gleichzeitig. Dabei steht auf dem Schlag kurzzeitig
ein anderer Dreiklang (im Beispiel: a-Moll), allerdings mit Quinte im Bass.
\enlargethispage{\baselineskip}

Oft möchten wir Vorhalte auch in die Funktionsbezeichnung mit aufnehmen. Haben
wir zum Beispiel die Dominante mit Quartvorhalt, so schreiben wir
$\muD{}^{\kern.5px 4\kern1px3}$. Bei kombinierten Vorhalten notieren wir einfach
beide, also zum Beispiel $\muD{}^{\kern.5px\smash{^6_4\kern1px{}^5_3}}$ für den
Quartsextvorhalt.  Wieder gibt es ein paar Regeln:

\begin{itemize}
\item %\textbf{Konsonante Einführung der Vorhaltsnote}\\
  Vorhalte sollten \emph{konsonant eingeführt}\footnote{Diese Regel ist in der
    Wiener Klassik nicht mehr ganz so wichtig.} werden: Die Vorhaltsnote muss
  regulärer Bestandteil des zuvor erklingenden Akkords sein und dann bis zum
  Vorhalt liegen bleiben. Oft wird diese Note übergebunden, wodurch die Stimmen
  rhythmisch versetzt sind, siehe \cref{fig:Vorhalt}.\looseness-1
\item %\textbf{Skaleneigene Vorhaltsnoten}\\
  Es sind skaleneigene Vorhaltsnoten zu verwenden. Dies hat zur Folge, dass die
  Quarte im $\muS{}^{\smash{\kern.5px 4\kern1px3}}$ übermäßig ist, und dass beim
  Nonvorhalt in Moll eine k2-Reibung zwischen der g9 und der k3 entsteht.
\item %\textbf{Keine Vorwegnahme der Auflösung}\\
  Vorenthaltene Terzen und Quinten dürfen nicht bereits in einer anderen Stimme
  erklingen. Vorenthaltene Oktaven sind unproblematisch.\marginpar{6.1+\textsc{p}}
\end{itemize}

\section{Sept- und Quintsextakkorde}

Es hat sich etabliert, die Akkorde der Dominante und Subdominante um je
einen weiteren Ton anzureichern, der den Charakter der Funktion verstärkt.

\subsection{Dominantseptakkord}

In Kadenzen wird der Dominante oft die leitereigene Septime (die sowohl in Dur
als auch in Moll klein ist) hinzugefügt. Der entstehende Vierklang heißt dann
\emph{Dominantseptakkord}, kurz $\muD^{\kern.4px 7}$.  Durch den zwischen g3 und
k7 bestehenden Tritonus wird die Strebetendenz zur Tonika und somit der
dominantische Charakter verstärkt. Einige Regeln:

\begin{itemize}
\item Bei der Auflösung muss die Septime schrittweise abwärts geführt werden
  (also in die Terz der Tonika oder die Quinte des Trugschlusses). Wie beim
  Leitton wird hiervon in den Mittelstimmen oft abgewichen (siehe
  \cref{fig:d7}:3); dennoch wird die Septime nie verdoppelt.
\item Terz und Septime müssen in diesem Akkord also genau einmal vorkommen. Da
  wir (vorerst) nicht auf den Grundton verzichten wollen, bleiben für die
  übrigen Stimmen bleiben zwei Möglichkeiten: Grundton und Quinte, oder zweimal
  Grundton (siehe \cref{fig:d7}:2).
\item Im Bass darf der Grundton, die Terz und sogar die Septime liegen.  Liegt
  die Septime im Bass und folgt die Tonika, so hat dieser Folgeakkord die Terz
  im Bass, siehe \cref{fig:d7}:4.
\end{itemize}
Durch die Septime kann ein Quartvorhalt der Tonika vorbereitet werden. Außerdem
entspricht die Septime im $\muD^{\kern.4px 7}$ der Oktave der Subdominante,
sodass bei der Verbindung $\muS$\,–\,$\muD^{\kern.4px 7}$ ein Ton liegen bleiben
kann, siehe \cref{fig:d7}:2.

% Hier einige Beispiele, die die verschiedenen Möglichkeiten zeigen:
\begin{figure}[h]
  \centering
  \includegraphics{ly/s-d7}\vspace*{-4px}%\\[-57.9px]\hspace*{-1.98cm}\rotatebox{-55}{\rule{3.5mm}{.7px}}~\\[32px]
  \caption{Diverse Formen des Dominantseptakkords und seiner Auflösung. Im
    letzten Beispiel sehen wir einen Trugschluss in Moll, der in den tG führt.}\label{fig:d7}
\end{figure}

\noindent Eine Variation des Dominantseptakkord erlaubt es, die Quinte in den
Bass zu legen und dafür den Grundton auszulassen (dann muss die 5 verdoppelt
werden).  Der resultierende Akkord nennt sich \emph{verkürzter
  Dominantseptakkord}, notiert
\mbox{\raisebox{-.2px}{\rotatebox{40}{\rule{4mm}{.7px}}}\hspace*{-8.4px}$\muD_5^7$}.
Er besteht aus zwei kleinen Terzen und ist somit ein verminderter Dreiklang. In
Dur besteht er genau aus den Tönen der Stufe \textsc{vii}.

Die Quinte im Bass sollte dann schrittweise aufgelöst werden.  Dies geht sowohl
nach oben als auch nach unten, meistens möchte man aber Gegenbewegungen zum
Sopran haben, siehe \cref{fig:vD}.

\begin{figure}[h]
  \centering
  \includegraphics{ly/s-d7verk}
  \caption{Der verkürzte Dominantseptakkord und seine Auflösungen. Im zweiten
    Beispiel geht der Bass schrittweise nach oben in die Terz der
    Tonika.}\label{fig:vD}
\end{figure}

\subsection{Subdominant-Quintsextakkord (\emph{sixte ajoutée})}

Der Subdominante wird oft die leitereigene Sexte (die sowohl in Dur als auch in
Moll groß ist) hinzugefügt. Wichtig ist, dass \emph{sowohl Quinte als auch
  Sexte} vorkommen, sonst läge (jedenfalls in Dur) ein anderer Dreiklang
vor. Deshalb heißt der entstehende Klang \emph{Quintsextakkord}, kurz
\smash{$\muS^{\kern.4px{}^6_5}$} bzw.\ \smash{$\mus^{\kern.4px{}^6_5}$}.  Üblich
ist auch der französische Name \textit{sixte ajoutée} (»hinzugefügte Sexte«).

Da die Sexte der Subdominante der Quinte der Dominante entspricht, können bei der
der Akkordfolge $\smash{\muS^{^6_5}}$\,–\,$\muD$ oder gar $\smash{\muS^{^6_5}}$\,–\,$\muD^7$
noch mehr Töne liegen bleiben, siehe \cref{fig:s6}:3.  In Moll hat der
Quintsextakkord einen Tritonus zwischen der k3 und der g6.  Einige Regeln:
\begin{itemize}
\item Wie schon erwähnt darf die Quinte nicht weggelassen werden.
\item Im Bass darf der Grundton oder die Terz liegen, in Moll auch die Sexte
  (siehe \cref{fig:s6}:2). Letzteres sollte in Dur vermieden werden, um eine
  Verwechslung mit der Sp (mit hinzugefügter 7) auszuschließen.
\item Die Quinte muss abwärts geführt werden, fast immer in die Terz der
  Dominante, ggf. über einen Quartvorhalt (wie in \cref{fig:s6}:3).\marginpar{7.1–\textsc{p}}
\end{itemize}

\begin{figure}[h]
  \centering
  \includegraphics{ly/s-s6}
  \caption{Einige Erscheinungsformen des Subdominant-Quintsextakkords.}\label{fig:s6}
\end{figure}

\section{Erweiterung des harmonischen Rahmens}

Um einen Satz harmonisch reichhaltiger zu gestalten, können auch Funktionen
außerhalb des harmonischen Rahmens verwendet werden. Die verwendeten
zusätzlichen Funktionen sollen aber zu einer Grundfunktion hinführen, also »aus deren
Perspektive« eine Dominante sein.  Wir beginnen mit der üblichsten
Erscheinungsform: der Doppeldominante.

\subsection{Doppeldominanten}

Die Dominante der Dominante heißt \emph{Doppeldominante}, notiert $\muDD$. Um
sie zu bilden, braucht man eine skalenfremde Quarte. Beispielsweise braucht man
in C-Dur das fis, um den Doppeldominant-Dreiklang (d\,:\,fis\,:\,a) zu bilden.

Analog können wir den Doppeldominantsept sowie seine Verkürzung verwenden. Die
Doppeldominante führt zur Dominante; es gelten die gleichen Regeln zur
Auflösung der einzelnen Töne wie bei der Dominante selbst.

\begin{figure}[h]
  \centering
  \includegraphics{ly/s-dd}\\[-3px]
  \small\hspace*{31px}%
  $\muT$\hspace*{14px}%
  $\muS^{\kern.5px{}^6_5}$\hspace*{3.5px}%
  $\muDD\kern.3px{}^7_3$\hspace*{4px}%
  $\muD^{\kern.5px{}^4_6\hspace*{4px}\kern1px{}^3_5\hspace*{2px}}$\hspace*{13px}%
  $\muT$\hspace*{7px}%
  $\muDD$\hspace*{7px}%
  $\muD$\hspace*{6px}%
  $\muDD\kern.3px{}^7_3$\hspace*{9px}%
  $\muD$\hspace*{5px}%
  \raisebox{-.45px}{\rotatebox{40}{\rule{4.7mm}{.7px}}}\hspace*{-9.7px}$\muDD\kern.3px{}^7_3$\hspace*{3.6px}%
  $\muD$\hspace*{8.4px}%
  $\muT$\hspace*{13px}%
  $\muDD_7$\hspace*{4.5px}%
  $\muD_{\kern.4px3}^{\kern.4px 8\,\hspace*{1px}7\hspace*{-1.4px}}$\hspace*{6.5px}%
  $\muT$\vspace*{-3px}
  \caption{Ein Notenbeispiel mit vielen Doppeldominanten, aus \cite[S.\,199]{deLaMotte}}
\end{figure}

\subsection{Zwischendominanten}
Allgemeiner können zu allen Funktionen Do\-mi\-nan\-ten gebildet werden, die
wir \emph{Zwischendominanten} nennen. Sie werden mit (D) notiert; die Klammer
bedeutet, dass die Dominante relativ zur Folgefunktion zu verstehen ist.%   Wir
% wollen einige übliche Zwischendominanten
% hervorheben:
\begin{itemize}
%\item In Dur hat der Dominantsept zur Subdominante eine skalenfremde 7.
\item In Dur sind Zwischendominanten zu den Nebenfunktionen Sp bzw.\ Tp
  beliebt. Hierzu wird der Grundton bzw.\ die Quinte hochalteriert.
\item Zwischendominantsept-Akkorde zur Subdominante sind reizvoll: In Dur ist
  ihre Septime skalenfremd, in Moll ihre Terz.\marginpar{8.1–\textsc{p}}\vspace*{-5pt}
\end{itemize}

\begin{figure}[h]
  \centering
  \includegraphics{ly/s-zwischend}\\[-3px]
  \small\hspace*{10.2mm}%
  $\muT$\hspace*{-.1mm}%
  $(\hspace*{-.3mm}\muD\kern.3px{}_3^7\hspace*{-.2mm})$\hspace*{0mm}%
  $\muS\mup$\hspace*{.5mm}%
  $\muD\kern.3px{}_3$\hspace*{2.8mm}%
  $\muT$\hspace*{13.6mm}%%%%%%%
  $\muT$\hspace*{.1mm}%
  $(\hspace*{-.2mm}\muD\kern.3px{}^7\hspace*{-.2mm})$\hspace*{.1mm}%
  $\muS$\hspace*{4.6mm}%
  $\muD^{\kern.3px 8\hspace*{.3mm}7}$\hspace*{4.2mm}%
  $\muT$\hspace*{18mm}%%%%%%%%%
  $\mut$\hspace*{.8mm}%
  $(\hspace*{-.2mm}\muD\kern.3px{}_7\hspace*{-.4mm})$\hspace*{.6mm}%
  $\mus$\hspace*{2.1mm}%
  $\muD$\hspace*{3.5mm}
  $\muT$\hspace*{1mm}
  \caption{Einige übliche Zwischendominanten.}\label{fig:ZD}
\end{figure}

\subsection{Zwischenkadenz – Modulation?}

Manchmal ist nicht nur eine einzelne Dominante, sondern eine ganze Kadenz
relativ zu einer bestimmten Zielfunktion gesetzt, siehe \cref{fig:ZK}. Auf diese
Weise wird es möglich, sich ein bisschen länger im harmonischen Rahmen einer
anderen Tonart (hier: d-Moll) aufzuhalten.  Meistens geschieht dies nur
übergangsweise. Bleibt man jedoch in der neuen Tonart, so hat man den
Bezugspunkt verschoben, d.\,h. man hat \emph{moduliert}.\looseness-1

\begin{figure}[h]
  \centering
  \includegraphics{ly/s-zwischenkadenz}\vspace*{-3px}
  \small \hspace*{15.6mm}%
  $\muT$\hspace*{6.0mm}%
  $(\mus^{\kern.3px{}^6_5}$\hspace*{2mm}%
  $\muD^{\kern.3px 4~3})$\hspace*{.1mm}%
  $\muD\mup$\hspace*{.8mm}%
  $(\muD^{7}_3)$\hspace*{4.9mm}%
  $\muS^{\kern.3px{}^6_5}$\hspace*{2.5mm}%
  $\muD^{\kern.3px8\kern5px7\kern-1px}$\hspace*{3.4mm}%
  $\muT$\hspace*{5.3mm}
  \caption{Eine Zwischenkadenz zur Dominantparallele in B-Dur.}\label{fig:ZK}
\end{figure}

\section{Dominanten mit Nonen}\label{None}

Es gibt drei sehr beliebte Abwandlungen der Dominante (oder einer beliebigen
Zwischendominante), deren Gemeinsamkeit ist, dass sie eine None enthalten. Sie
alle haben einen sehr individuellen Klang und sind ziemlich eindeutig einer
Musikepoche zuzuordnen.

\subsection{Barock: Verminderter Septakkord}

Im Dominantseptakkord kann der Grundton durch die \emph{kleine} None ersetzt
werden. Während in Moll die k9 der Dominante leitereigen ist (sie entspricht der
k6 der Skala), wird sie in Dur durch Tiefalteration erreicht. So wie der Leitton
um eine k2 nach oben und die Septime der Dominante um eine k2 nach unten
aufgelöst wird, will nun auch die k9 durch eine k2 nach unten aufgelöst werden,
in einer regulären Kadenz zur Quinte der Tonika. Dadurch wird die Dominante
sogar noch spannungsreicher. Einige Regeln:
\begin{itemize}
\item Bei diesem Vierklang kann jeder Ton (3, 5, 7, 9) im Bass
  liegen. Verdopplungen kommen im vierstimmigen Satz nicht in Frage.
\item Beim Akkordwechsel muss die k9 schrittweise abwärts aufgelöst
  werden. Die Auflösungsregeln von Terz und Septime bleiben
  unverändert.
\item Es ist also nur noch die Quinte des Akkords in ihrer Bewegung
  frei. Allerdings können Parallelen der Form »vermindert\,–\,rein«
  entstehen, wenn die 5 unter der 9 liegt auch schrittweise abwärts geführt
  wird.
\end{itemize}

% Im Falle der Doppeldominante können None bzw.\ Septime in einen Sext- bzw.\
% Quartvorhalt geführt werden. In Dur entsteht hier ein sehr interessanter
% Halbtonschritt von der k9 der Doppeldominante zur g6 der Dominante.

\begin{figure}[h]
  \centering
  \includegraphics{ly/s-Dv1}\vspace*{-7px}
  \caption{Einige verminderte Septakkorde und ihre Auflösungen.}
\end{figure}

\noindent Die Frage, wie dieser Akkord genannt und notiert werden soll, ist
nicht leicht zu beantworten: Unsere Beschreibung als Dominante ohne Grundton ist
künstlich, weil sie auf einem Ton aufbaut, der nicht vorhanden ist.
Historisch richtiger wäre, ihn als Vierklang, der auf dem Leitton fußt und
aus \emph{drei} kleinen Terzen besteht, zu sehen. Deswegen wird er
\emph{verminderter Septakkord} genannt. Relativ zum Leitton enthält
er also die r1, k3, v5 und v7.

Nachteil dieser Sichtweise ist, dass sich die Rollen der beteiligten Töne um
eine Terz verschöben, wenn der Leitton als Fundament des Akkords gesehen
würde. Außerdem würden die Auflösungsbewegungen nicht erklärt.  Wir wollen daher
die Bestandteile weiterhin relativ zum nicht vorhandenen Grundton bezeichnen,
also als g3, r5, k7 und k9. Wir notieren den Akkord als $\muD^v$ und schreiben
den Basston in den Index: Es gibt also $\muD^v_3$, $\muD_5^v$, $\muD_7^v$,
$\muD_9^v$.


\begin{figure}[h]
  \centering
  \includegraphics{ly/s-Dv2}\\[-3px]
  \small\hspace*{16.7mm}%
  $\mut$\hspace*{3.1mm}%
  $\muD^v_3$\hspace*{2.65mm}%
  $\mut$\hspace*{2.95mm}%
  $\muD_7^v$\hspace*{3.85mm}%
  $\mut_3$\hspace*{2.05mm}%
  $\muD^v_5$\hspace*{1.93mm}%
  $\mut_{3}$\hspace*{7.47mm}%
  $\muD^v_9$\hspace*{2mm}%
  $\mut_5$\hspace*{2.9mm}%
  $\mus^{^5_6}$\hspace*{1.3mm}%
  $\muDD^v_3$\hspace*{2.9mm}%
  $\muD^{\kern.3px{}^{4\hspace*{2mm}3}_{8\hspace*{2mm}7}}$\hspace*{5mm}%
  $\mut$\hspace*{10mm}
  \caption{Beispiel mit etlichen verminderten Septakkorden, frei nach
    \cite[S.\,99]{deLaMotte}}
\end{figure}

\subsection{Klassik: $\muDDZ^{\bm{v}}$ mit tiefalterierter Quinte}

Der zur Doppeldominante gehörende verminderte Septakkord kann weiter
angeschärft werden, indem man die Quinte tiefalteriert. Damit hat nun jeder
der vier Töne eine Strebetendenz um eine k2 nach oben oder unten.  Die
tiefalterierte Quinte der Doppeldominante entspricht der k6 der Skala und ist
daher in Dur skalenfremd, jedoch in Moll skaleneigen.

Die tiefalterierte Quinte muss im Bass liegen und zur Auflösung abwärts in den
Grundton der Dominante geführt werden. Auf diese Weise entstehen
Quintparallelen, die nur durch einen Quartsextvorhalt der Dominante verhindert
werden können, oft aber in Kauf genommen werden (immerhin sind wir mit diesem
Akkord schon in der Zeit nach Bach). Sie haben sogar einen eigenen Namen:
\emph{Mozart-Quinten}. Der Akkord wird notiert als
$\muDD^v_{5-}$.\vspace*{-3px}

\begin{figure}[h]
  \centering
  \includegraphics{ly/s-Dv5-}\vspace*{-8px}
  \caption{Verminderte Septakkorde mit tiefalterierter Quinte.}
\end{figure}

\subsection{Romantik: Dominantseptnonakkord}
In Dur kann dem Dominantseptakkord auch die skaleneigene g9 hinzugefügt
werden. So entsteht ein Fünfklang, den wir \emph{Dominantseptnonakkord}
nennen. Im vierstimmigen Satz muss also ein Ton fehlen; das darf die Quinte oder
der Grundton sein. Entsprechend notieren wir ihn als
$\smash{\muD^{\kern.3px{}^9_7}}$ oder
\smash{\mbox{\raisebox{-.2px}{\rotatebox{40}{\rule{4mm}{.7px}}}\hspace*{-8.4px}$\smash{\muD^{\kern.3px{}^9_7}}$}}.

Wieder kann jeder der Töne im Bass liegen. Die None und die Septime müssen
schritt\-weise abwärts geführt werden, der Leitton aufwärts.\marginpar{9.1–\textsc{p}}\vspace*{-3px}

\begin{figure}[h]
  \centering
  \includegraphics{ly/s-D79}\vspace*{-8px}
  \caption{Dominantseptnonakkorde in G-Dur sowie ihre Auflösungen. Im letzten
    Beispiel liegt ein Dominentseptnonakkord auf der Doppeldominante vor.}
\end{figure}

\section{Größere Linien}

In diesem letzten Abschnitt wollen wir den Blick etwas weiten und
einige grobe Gestaltungsprinzipien zur Inspiration studieren.

\subsection{Gegenbewegungen}%
Führt man Sopran und Bass in entgegengesetzte Richtun\-gen, können damit nicht
nur Parallelen vermieden werden; es klingt auch gut, weil die Randstimmen so
sehr unabhängig voneinander klingen, siehe \cref{fig:gegen}.

\begin{figure}[h]
  \centering
  \includegraphics{ly/s-gegen}\vspace*{-3pt}
  \caption{Die ersten Takte aus Händels, \emph{Tochter Zion}, Siegeschor
    aus \emph{Judas Maccabäus} (\acr{HWV\,63}), mit Gegenbewegungen zwischen
    Sopran und Bass.}\label{fig:gegen}\vspace*{-8px}
\end{figure}


\subsection{Chromatik im Bass}
Vor allem bei der Verwendung von Zwischendominanten bietet es sich an,
die Basslinie in Halbtonschritten zu führen.
\begin{figure}[h]
  \centering
  \includegraphics{ly/s-chroma}\vspace*{-3pt}
  \caption{Skizze eines Chorsatzes zu \emph{We wish you a merry christmas} mit
    vielen Zwischendominanten und einer Basslinie in Halbtonschritten.}\vspace*{-8px}
\end{figure}

\subsection{Quintfälle}
\emph{Quintfälle} sind Akkordfolgen, bei denen der Grundton stets
um eine Quinte abnimmt
(z.\,B. \textsc{i}\,–\,\textsc{iv}\,–\,\textsc{vii}\,–\,\textsc{iii}\,–\,\textsc{vi}
in Moll). Da hierbei stets ein Dominante-Tonika-Verhältnis angedeutet wird,
klingen solche Ketten sehr »folgerichtig«.

In der Bassstimme entsteht hierbei meistens eine Kette aus »r5 abwärts« oder »r4
abwärts«, siehe \cref{fig:vivaldi}. Im Laufe dieser Kette bewegt man sich im
Quintenzirkel Schritt für Schritt »gegen den Uhrzeigersinn«. Um wieder
zurückzukommen, ohne einmal im Kreis zu laufen, wird oft nach einiger Zeit eine
einzelne verminderte Quinte (im Beispiel es–A) genutzt.
\begin{figure}[h]
  \centering
  \includegraphics{ly/s-quintfall}
  \caption{Ein Quintfall in Vivaldis \emph{Le quattro stagioni} (op.\,8,
    \acr{RV\,269}), \emph{Herbst}, Satz 1 in F-Dur, Takte 60–62.}\label{fig:vivaldi}\vspace*{-8px}
\end{figure}

\subsection{Sequenz}
\enlargethispage{\baselineskip}
Eine andere Möglichkeit, den Einruck von Folgerichtigkeit zu erzeugen,
stellen \emph{Sequenzen} dar: Hier wird ein melodisches oder rhythmisches Motiv
nacheinander auf unterschiedlichen Höhen gespielt.

\begin{figure}[h]
  \centering
  \includegraphics{ly/s-sequenz}
  \caption{Die Bassstimme der Takte 117–128 des »Cum sancto spiritu« aus Bachs
    h-Moll-Messe (\acr{BWV\,232}), mit ausufernder Sequenzierung.}\vspace*{-8px}
\end{figure}

\printbibliography

\end{document}

%%% Local Variables:
%%% TeX-engine: default
%%% End:
