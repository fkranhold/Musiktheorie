\documentclass[ngerman]{scrartcl}
%% Serifen in Überschriften
\KOMAoptions{paper=b5,numbers=enddot,DIV=11,headings=standardclasses}

%% Encoding
\usepackage[utf8]{inputenc}
\usepackage[T1]{fontenc}

%% Sprache
\usepackage{babel}
\usepackage[cleanlook]{isodate}

%% Schriftpakete
\usepackage{lmodern}
\usepackage[osf,sc]{mathpazo}
\usepackage{classico}
\usepackage[scaled=.85]{beramono}

%% Palatino braucht mehr Platz
\usepackage{setspace}
\setstretch{1.07}
\renewcommand{\arraystretch}{1.07} % Für tabular- und array-Umgebungen

%% Mikrotypographie!
\usepackage{ellipsis}
\usepackage[babel,tracking=true]{microtype}
\UseMicrotypeSet[tracking]{smallcaps}
\SetTracking{encoding=*,shape=sc}{30}

%% Akronyme
\newcommand{\acr}[1]{\textls[40]{\textsc{\MakeLowercase{#1}}}}

%% Eigene Listing styles mit kurzer Definition
\usepackage[shortlabels]{enumitem}

%% Captions for figures and tables
\setkomafont{captionlabel}{\footnotesize\bfseries}
\setkomafont{caption}     {\footnotesize}
\setcapwidth{.9\textwidth}
\renewcommand*{\captionformat}{.\ }
\setcapindent{0em}
\addtokomafont{caption}{\setstretch{1.07}}

%% Fancy tables
\usepackage{booktabs}

%% Mehrspaltinge Aufzählungen
\usepackage{multicol}

%% Bibliographie
\usepackage{biblatex}
\bibliography{bibliography.bib}

%% Footnote-Design
\deffootnote{1.5em}{1.5em}{\thefootnotemark.\ }
\addtokomafont{footnote}{\setstretch{1.07}}

%% Todo
\usepackage[size=tiny]{todonotes}

%% Grundlegende mathematische Symbole
\usepackage{amssymb,amsmath} % Standard
\usepackage{mathtools}       % \coloneqq, \eqqcolon

%% TikZ and TikZ-CD
\usepackage{tikz-cd}
\usepackage{tikz}
\usetikzlibrary{arrows,calc}

%% Mini-Indizes (\scaleto{…}{3pt})
\usepackage{scalerel}

%% Bold
\newcommand{\bm} [1]{\mathbold{#1}}

%% TikZ: centerarc
 \def\centerarc[#1](#2)(#3:#4:#5)% Syntax: [draw options] (center) (initial angle:final angle:radius)
     { \draw[#1] ($(#2)+({#5*cos(#3)},{#5*sin(#3)})$) arc (#3:#4:#5); }

%% Musikalische Symbole
\usepackage{musicography}

%% Harmonische Funktionen
\newcommand{\mut}         {\text{t}}
\newcommand{\mus}         {\text{s}}
\newcommand{\mud}         {\text{d}}
\newcommand{\muT}         {\text{T}}
\newcommand{\muS}         {\text{S}}
\newcommand{\muD}         {\text{D}}
\newcommand{\mup}         {\text{p}}
\newcommand{\muP}         {\text{P}}
\newcommand{\mug}         {\text{g}}
\newcommand{\muG}         {\text{G}}
\newcommand{\muDD}        {\mbox{\muD\hspace{-5.5px}\raisebox{1.3px}{\muD}}}
\newcommand{\muDDZ}       {\mbox{\muD\hspace{-6.1px}\raisebox{1.3px}{\muD}}}
\newcommand{\muSS}        {\raisebox{1.5px}{\muS}\hspace{-4px}\muS}
\newcommand{\muss}        {\raisebox{1.5px}{\mus}\hspace{-3px}\mus}
\newcommand{\musDD}       {\muD\hspace{-6px}\raisebox{2px}{\sout{\muD}}}
\newcommand{\vermsept}    {\hspace{1mm}\raisebox{4px}{\mus}\hspace{-3px}\muD}
\newcommand{\sevsept}     {\hspace{1mm}\raisebox{3px}{\muS}\hspace{-5px}\muD}
\newcommand{\dvermsept}   {\hspace{1mm}\raisebox{4px}{\mut}\hspace{-1px}\muD\hspace{-6px}\raisebox{2px}{\muD}}

%% Hyperref and Cleveref
\usepackage{amsthm}
\usepackage[pdfusetitle,bookmarksnumbered=true]{hyperref}
\usepackage[noabbrev]{cleveref}

%% Aufgaben
\theoremstyle{definition}
\newtheorem{aufg}{Aufgabe}[section]

%\usepackage[dvipsnames]{xcolor}
%\newcommand{\rb}[1]{\textcolor{Maroon}{#1}}
\newcommand{\rb}[1]{#1}


\begin{document}

\title {Musiktheorie}
\author{Florian Kranhold\and Charlotte Mertz} 
\subject{CdE-WinterAkademie 2025\,·\,26}
\subtitle{Aufgaben}
\hypersetup{pdftitle={Musiktheorie: Aufgaben}}

\maketitle

Zum ersten Abschnitt im Skript gibt es noch keine Übungsaufgaben. Wir fangen
hier also mit Abschnitt 2 an.%
\setcounter{section}{1}

\section{Intervalle}

\begin{aufg}
  Bestimme folgende Töne:\vspace*{-.25\baselineskip}
  \begin{multicols}{2}
    \begin{enumerate}[itemsep=0em]
    \item r5 über g’,
    \item k2 unter h’’,
    \item g3 unter g,
    \item k7 über A,
    \item v2 über cis,
    \item ü4 über H,
    \item v1 über c”,
    \item k6 über a”.
    \end{enumerate}
  \end{multicols}
\end{aufg}

\begin{aufg}
  Benenne folgende Intervalle:
  \begin{center}
    \includegraphics{ly/e-04-2}
    \bigskip
    \bigskip
  \end{center}
\end{aufg}

\begin{aufg}
  Bestimme die Komplementärintervalle von:
  \begin{center}r1,\qquad k3,\qquad r4,\qquad ü4,\qquad k6,\qquad g7\vspace*{2pt}\end{center}
\end{aufg}

\begin{aufg}
  Zwischen welchen weißen Tasten besteht ein Tritonus? Zwischen welchen
  schwarzen Tasten?
\end{aufg}

\newpage
\section{Skalen, Tonarten und der Quintenzirkel}

\begin{aufg}
  Bestimme die Vorzeichen folgender Tonarten:\vspace*{-.25\baselineskip}
  \begin{multicols}{3}
    \begin{enumerate}[itemsep=0em]
    \item Es-Dur,
    \item g-Moll,
    \item c-Moll,
    \item Fis-Dur,
    \item cis-Moll,
    \item H-Dur,
    \item A-Dur,
    \item f-Moll,
    \item b-Moll.
    \end{enumerate}
  \end{multicols}
\end{aufg}

\begin{aufg}
  Bestimme die Tonarten folgender Beispiele:
  \begin{quote}
    \includegraphics{ly/e-05-1}\\[.5\baselineskip]
    \includegraphics{ly/e-05-2}\\[.5\baselineskip]
    \includegraphics{ly/e-05-3}
  \end{quote}
\end{aufg}

% \begin{aufg}
%   Bildet man eine Skala mit Grundton c, die nur weiße Tasten verwendet, so
%   ergibt sich eine Dur-Skala. Hier sind alle Intervalle zum Grundton rein oder
%   groß. Bei welchem Ton müsste eine Skala von weißen Tasten starten, damit
%   jeweils folgendes gilt:
%   \begin{enumerate}[itemsep=0em]
%   \item Alle Intervalle zum Grundton sind rein oder klein.
%   \item Die Quarte zum Grundton ist übermäßig.
%   \item Die Quinte zum Grundton ist vermindert.
%   \end{enumerate}
%   Welche Kirchentonart erhält man jeweils?
% \end{aufg}

\begin{aufg}
  Lässt man den Grundtton gleich, unterscheiden sich Modi nur in den
  Vorzeichen, die benötigt werden, um sie zu bilden.
  \begin{enumerate}
  \item Welche Vorzeichen muss man bei einer Durskala ändern, um sie zu einer lydischen Skala zu machen?
  \item Welche Vorzeichen muss man bei einer Mollskala ändern, um sie zu einer dorischen Skala zu machen?
  \item Welche Vorzeichen muss man bei einer Mollskala ändern, um sie zu einer phrygischen Skala zu machen?
  \end{enumerate}
\end{aufg}

\begin{aufg}
  Stapelt man sukzessive Quinten, so erreicht man jeden klingenden Ton. Welche
  anderen Intervalle könnte man anstatt Quinten nehmen? Welche nicht? Wie kann man
  das auch mathematisch ausdrücken?
\end{aufg}

\newpage
\section{Dreiklänge, Stufen und Funktionen}

\begin{aufg}
  Bestimme folgende Funktionen:\vspace*{-.25\baselineskip}
  \begin{multicols}{3}
    \begin{enumerate}[itemsep=0em]
    \item D in d-Moll,
    \item Sp in H-Dur,
    \item S in B-Dur,
    \item Tp in A-Dur,
    \item tG in a-Moll,
    \item Sp in E-Dur,
    \item dP in c-Moll,
    \item DP in F-Dur,
    \item s der sP in C-Dur.
    \end{enumerate}
  \end{multicols}
\end{aufg}
% \lipsum

\begin{aufg}
  Durch welche Akkordfolge ist eine Vollkadenz in folgenden Tonarten
  beschrieben? Wohin würde jeweils der Trugschluss führen?\vspace*{-.25\baselineskip}
  \begin{multicols}{3}
    \begin{enumerate}[itemsep=0em]
    \item F-Dur,
    \item a-Moll,
    \item E-Dur,
    \item c-Moll,
    \item A-Dur,
    \item b-Moll.
    \end{enumerate}
  \end{multicols}
\end{aufg}

\begin{aufg}
  Betrache folgenden vierstimmigen Satz:
  \begin{center}
    \includegraphics{ly/e-06-4}
  \end{center}
  \begin{enumerate}
  \item In welcher Tonart steht dieses Beispiel?
  \item Auf jedem Schlag ergeben die vier Stimmen einen Akkord innerhalb des
    harmonischen Rahmens. Bestimme die jeweiligen Funktionen.
  \end{enumerate}
\end{aufg}

\begin{aufg}
  Zwischen welchen zwei Tönen liegt der problematische Hiatus im Tonvorrat von
  g-Moll? Wo in c-Moll? Wo in e-Moll?
\end{aufg}

\newpage
\section{Struktur von Notensätzen}

\begin{aufg}
  Finde so viele Fehler wie möglich:
  \begin{center}
    \includegraphics{ly/e-fehler}
  \end{center}
\end{aufg}

\begin{aufg}
  Ergänze in nachfolgendem Beispiel die Mittelstimmen (also Alt und Tenor). Beachte
  dabei alle erlernten Regeln. %\vspace*{-5px}
  \begin{center}
    \includegraphics{ly/e-07-1}\\[-3px]
    \small \hspace*{17.8mm}%
    $\mut$\hspace*{3.7mm}%
    $\mus_3$\hspace*{1.3mm}%
    $\mud\muP$\hspace*{2.1mm}%
    $\mut\muP$\hspace*{4.3mm}%
    $\muD$\hspace*{2.1mm}%
    $\mut\muG$\hspace*{2.5mm}%
    $\muD$\hspace*{3.2mm}%
    $\muT$\hspace*{9.5mm}\vspace*{4pt}
  \end{center}
\end{aufg}

% \begin{aufg}
%   Vervollständige den Satz:%\vspace*{-5px}
%   \begin{center}
%     \includegraphics{ly/e-07-2}\\[-10px]
%     \small\hspace*{22mm}%
%     $\muT$\hspace*{2.5mm}%
%     $\muD\mup$\hspace*{2.4mm}%
%     $\muS$\hspace*{3.3mm}%
%     $\muT$\hspace*{4.5mm}%
%     $\muS\mup_3$\hspace*{.7mm}%
%     $\muD$\hspace*{2.6mm}%
%     $\muT$\hspace*{11.8mm}\vspace*{-4pt}
%   \end{center}
% \end{aufg}

\paragraph{Permanente Übung.}
~\kern-8px Schreibe einen Satz zu folgender Melodie\footnote{Von Hans Leo Haßler
  (1601), von Bach unzählige Male als Choral gesetzt.}  und notiere die
verwendeten Funktionen. Alles bisher Bespro\-chene darf benutzt
werden. (Beschränke Dich ruhig auf die Takte bis zum
Wiederholungszeichen.)%\vspace*{7px}
\begin{center}
  \includegraphics{ly/e-perm}
  % 1}\\[5px]
  % \includegraphics{ly/e-perm2}
\end{center}

\newpage
\section{Harmoniefremde Töne}

\begin{aufg}
  Benenne alle Vorhalte und Durchgänge:
  \begin{center}
    \includegraphics{ly/e-08-1}
  \end{center}
\end{aufg}

\begin{aufg}
  Ergänze Durchgänge und Wechselnoten:
  \begin{center}
    \includegraphics{ly/e-08-2}\vspace*{5pt}
  \end{center}
\end{aufg}

\paragraph{Permanente Übung.}
~\kern-8px Schreibe einen Satz zu folgenden Melodien und notiere die verwendeten
Funktionen. Alles bisher Bespro\-chene darf benutzt werden.
(Beschränke Dich ruhig auf die Takte bis zum Wiederholungszeichen.)
\begin{center}
  \includegraphics{ly/e-perm}
\end{center}

\newpage
\section{Sept- und Quintsextakkorde}

\begin{aufg}
  Bestimme die Bestandteile folgender Vierklänge und gib ggf. an, zwischen
  welchen Tönen ein Tritonus liegt:
  \vspace*{-.25\baselineskip}
  \begin{multicols}{3}
    \begin{enumerate}[itemsep=0em]
    \item $\muD^7$ in As-Dur,
    \item $\muS^{^6_5}$ in Es-Dur,
    \item $\mus^{^6_5}$ in fis-Moll,
    \item $\muD^7$ in b-Moll,
    \item $\muS^{^6_5}$ in A-Dur,
    \item $\mus^{^6_5}$ in f-Moll.
    \end{enumerate}
  \end{multicols}
\end{aufg}

\begin{aufg}
  Benenne in folgendem Beispiel die Funktionen und markiere dabei Septimen und
  Sexten. Sind alle besprochenen Regeln berücksichtigt? %\vspace*{-10px}
  \begin{center}
    \hspace*{-3px}
    \includegraphics{ly/e-09-2}\vspace*{12px}
    \vspace*{-5pt}
  \end{center}
\end{aufg}

\paragraph{Permanente Übung.}
~\kern-8px Schreibe einen Satz zu folgenden Melodien und notiere die verwendeten
Funktionen. Alles bisher Bespro\-chene darf benutzt werden.
(Beschränke Dich ruhig auf die Takte bis zum Wiederholungszeichen.)
\begin{center}
  \includegraphics{ly/e-perm}% 1}\\
  % \includegraphics{ly/e-perm2}
\end{center}

\newpage
\section{Erweiterung des harmonischen Rahmens}

\begin{aufg}
  Bestimme die Töne folgender Akkorde. Welche davon sind in der vorgegebenen
  Tonart skalenfremd?\vspace*{-.25\baselineskip}
  \begin{multicols}{2}
    \begin{enumerate}[itemsep=0em]
    \item $\muDD^7$ in f-Moll,
    \item $\muS^{^6_5}$ der $\muS$ in B-Dur,
    \item $\muD$ der $\muD\mup$ in As-Dur,
    \item $\muD\kern.3px{}^7$ der $\mus$ in a-Moll,
    \item
      \mbox{\raisebox{-.2px}{\rotatebox{40}{\rule{4mm}{.7px}}}\hspace*{-8.4px}$\muD_5^7$}
      der $\muS\mup$ in E-Dur,
    \item D der \mbox{$\muDD$} in F-Dur.
    \end{enumerate}
  \end{multicols}
\end{aufg}

\begin{aufg}
  Mit hinreichend vielen Zwischendominanten (und einer Doppel\emph{sub}dominante,
  die wir mit $\muSS$ bezeichnen) können wir sogar eine chromatische Tonleiter
  sinnvoll aussetzen. Ergänze die Mittelstimmen:
  \begin{center}
    \hspace*{-2px}
    \includegraphics{ly/e-10-2}\\[-6px]
    \small \hspace*{38.2mm}$\muSS_3$\hspace*{-30mm}
  \end{center}
\end{aufg}

\begin{aufg}
  Schreibe einen Chorsatz in g-Moll, der mit möglichst geringem Bewegungsaufwand
  die vorgeschriebene Akkordfolge realisiert. (Tipps: Beginne mit dem
  Bass. Kümmere Dich nicht um die Taktart.)
  \begin{center}
    \hspace*{-2px}
    \includegraphics{ly/e-10-3}\\[-4px]
    \smash{~~~~~~~~~~~$\mut$~~~~\hspace*{.7mm}$\mus\muP$\hspace*{-.7mm}~~~~$(\muD\kern.3px{}^7)$~~~~$\mut\muP$~~~~$(\muD\kern.3px{}_3^7)$~~~~$\mus\kern.3px{}^{^6_5}$~~~~$\muDD\kern.3px{}_3^7$~~~~$\muD\kern.3px{}^{4~~3}$~~~~~$\mut$}
  \end{center}
\end{aufg}

\newpage
\section{Dominanten mit Nonen}

\begin{aufg}
  Analysiere folgenden (überladenen) Satz:
  \begin{center}
    \includegraphics{ly/e-11-1}\vspace*{10px}
  \end{center}
\end{aufg}

\begin{aufg}
  Schreibe einen Chorsatz in D-Dur, der mit möglichst geringem
  Bewegungsaufwand die vorgeschriebene Akkordfolge realisiert. (Tipps: Beginne
  mit dem Bass. Kümmere Dich nicht um die Taktart.)
  \begin{center}
    \hspace*{-2px}\includegraphics{ly/e-11-2}\\
    \smash{~~~~~~~~~~$\muT$~~~~$\muD\kern.3px{}^7_3$~~~~\kern-1mm$(\muD\kern.3px{}^v_7)$~~~~$\muS\kern.3px{}^{^6_5}_{\smash{{}_3}}$~~~~$\muDD\kern.3px{}^v_{5-}$~~~~\kern-1mm$\muD\kern.3px^{^{6~5}_{4~3}}$\kern-1mm~~~~$(\muD\kern.3px{}^v_3)$\kern-1mm~~~~$\muT\mup$~~~~$\muDD\kern.3px{}^{^9_7}_{\smash{{}_3}}$~~~~$\muD^{4~~3}$~~~$\muT$}
  \end{center}
\end{aufg}

\paragraph{Permanente Übung.}
~\kern-8px Schreibe einen Satz zu folgenden Melodien und notiere die verwendeten
Funktionen. Alles bisher Bespro\-chene darf benutzt werden.  (Beschränke Dich
ruhig auf die Takte bis zum Wiederholungszeichen.)
 \begin{center}
  \includegraphics{ly/e-perm}
\end{center}

\end{document}

%%% Local variables:
%%% TeX-engine: default
%%% End:
