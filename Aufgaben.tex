\documentclass[ngerman]{scrartcl}
%% Serifen in Überschriften
\KOMAoptions{paper=b5,numbers=enddot,DIV=11,headings=standardclasses}

%% Encoding
\usepackage[utf8]{inputenc}
\usepackage[T1]{fontenc}

%% Sprache
\usepackage{babel}
\usepackage[cleanlook]{isodate}

%% Schriftpakete
\usepackage{lmodern}
\usepackage[osf,sc]{mathpazo}
\usepackage{classico}
\usepackage[scaled=.85]{beramono}

%% Palatino braucht mehr Platz
\usepackage{setspace}
\setstretch{1.07}
\renewcommand{\arraystretch}{1.07} % Für tabular- und array-Umgebungen

%% Mikrotypographie!
\usepackage{ellipsis}
\usepackage[babel,tracking=true]{microtype}
\UseMicrotypeSet[tracking]{smallcaps}
\SetTracking{encoding=*,shape=sc}{30}

%% Akronyme
\newcommand{\acr}[1]{\textls[40]{\textsc{\MakeLowercase{#1}}}}

%% Eigene Listing styles mit kurzer Definition
\usepackage[shortlabels]{enumitem}

%% Captions for figures and tables
\setkomafont{captionlabel}{\footnotesize\bfseries}
\setkomafont{caption}     {\footnotesize}
\setcapwidth{.9\textwidth}
\renewcommand*{\captionformat}{.\ }
\setcapindent{0em}
\addtokomafont{caption}{\setstretch{1.07}}

%% Fancy tables
\usepackage{booktabs}

%% Mehrspaltinge Aufzählungen
\usepackage{multicol}

%% Bibliographie
\usepackage{biblatex}
\bibliography{bibliography.bib}

%% Footnote-Design
\deffootnote{1.5em}{1.5em}{\thefootnotemark.\ }
\addtokomafont{footnote}{\setstretch{1.07}}

%% Todo
\usepackage[size=tiny]{todonotes}

%% Grundlegende mathematische Symbole
\usepackage{amssymb,amsmath} % Standard
\usepackage{mathtools}       % \coloneqq, \eqqcolon

%% TikZ and TikZ-CD
\usepackage{tikz-cd}
\usepackage{tikz}
\usetikzlibrary{arrows,calc}

%% Mini-Indizes (\scaleto{…}{3pt})
\usepackage{scalerel}

%% Bold
\newcommand{\bm} [1]{\mathbold{#1}}

%% TikZ: centerarc
 \def\centerarc[#1](#2)(#3:#4:#5)% Syntax: [draw options] (center) (initial angle:final angle:radius)
     { \draw[#1] ($(#2)+({#5*cos(#3)},{#5*sin(#3)})$) arc (#3:#4:#5); }

%% Musikalische Symbole
\usepackage{musicography}

%% Harmonische Funktionen
\newcommand{\mut}         {\text{t}}
\newcommand{\mus}         {\text{s}}
\newcommand{\mud}         {\text{d}}
\newcommand{\muT}         {\text{T}}
\newcommand{\muS}         {\text{S}}
\newcommand{\muD}         {\text{D}}
\newcommand{\mup}         {\text{p}}
\newcommand{\muP}         {\text{P}}
\newcommand{\mug}         {\text{g}}
\newcommand{\muG}         {\text{G}}
\newcommand{\muDD}        {\mbox{\muD\hspace{-5.5px}\raisebox{1.3px}{\muD}}}
\newcommand{\muDDZ}       {\mbox{\muD\hspace{-6.1px}\raisebox{1.3px}{\muD}}}
\newcommand{\muSS}        {\raisebox{1.5px}{\muS}\hspace{-4px}\muS}
\newcommand{\muss}        {\raisebox{1.5px}{\mus}\hspace{-3px}\mus}
\newcommand{\musDD}       {\muD\hspace{-6px}\raisebox{2px}{\sout{\muD}}}
\newcommand{\vermsept}    {\hspace{1mm}\raisebox{4px}{\mus}\hspace{-3px}\muD}
\newcommand{\sevsept}     {\hspace{1mm}\raisebox{3px}{\muS}\hspace{-5px}\muD}
\newcommand{\dvermsept}   {\hspace{1mm}\raisebox{4px}{\mut}\hspace{-1px}\muD\hspace{-6px}\raisebox{2px}{\muD}}

%% Hyperref and Cleveref
\usepackage{amsthm}
\usepackage[pdfusetitle,bookmarksnumbered=true]{hyperref}
\usepackage[noabbrev]{cleveref}

%% Aufgaben
\theoremstyle{definition}
\newtheorem{aufg}{Aufgabe}[section]

%\usepackage[dvipsnames]{xcolor}
%\newcommand{\rb}[1]{\textcolor{Maroon}{#1}}
\newcommand{\rb}[1]{#1}


\begin{document}

\title {Musiktheorie}
\author{Florian Kranhold\and Charlotte Mertz} 
\subject{CdE-WinterAkademie 2025\,·\,26}
\subtitle{Aufgaben}
\hypersetup{pdftitle={Musiktheorie: Aufgaben}}

\maketitle

Zu den ersten drei Abschnitten im Skript gibt es noch keine Übungen. Wir fangen
hier also mit Abschnitt 4 an.  \setcounter{section}{3}

\section{Intervalle}

\begin{aufg}
  Bestimme folgende Töne:\vspace*{-.25\baselineskip}
  \begin{multicols}{2}
    \begin{enumerate}[itemsep=0em]
    \item r5 über g’,
    \item k2 unter h’’,
    \item g3 unter g,
    \item k7 über A,
    \item v2 über cis,
    \item ü4 über H,
    \item v1 über c”,
    \item k6 über a”.
    \end{enumerate}
  \end{multicols}
\end{aufg}

\begin{aufg}
  Benenne folgende Intervalle:
  \begin{center}
    \includegraphics{ly/e-04-2}
  \end{center}
\end{aufg}

\begin{aufg}
  Bestimme die Komplementärintervalle von:
  \begin{center}r1,\quad k3,\quad r4,\quad ü4,\quad k6,\quad g7\vspace*{2pt}\end{center}
\end{aufg}

\begin{aufg}
  Zwischen welchen weißen Tasten besteht ein Tritonus? Zwischen welchen
  schwarzen Tasten?
\end{aufg}

\newpage
\section{Tonarten und der Quintenzirkel}

\begin{aufg}
  Bestimme die Vorzeichen folgender Tonarten:\vspace*{-.25\baselineskip}
  \begin{multicols}{2}
    \begin{enumerate}[itemsep=0em]
    \item Es-Dur,
    \item g-Moll,
    \item c-Moll,
    \item cis-Moll,
    \item H-Dur,
    \item b-Moll.
    \end{enumerate}
  \end{multicols}
\end{aufg}

\begin{aufg}
  Bildet man eine Skala mit Grundton c, die nur weiße Tasten verwendet, so
  ergibt sich eine Dur-Skala. Hier sind alle Intervalle zum Grundton rein oder
  groß. Bei welchem Ton müsste eine Skala von weißen Tasten starten, damit
  jeweils folgendes gilt:
  \begin{enumerate}[itemsep=0em]
  \item Alle Intervalle zum Grundton sind rein oder klein.
  \item Die Quarte zum Grundton ist übermäßig.
  \item Die Quinte zum Grundton ist vermindert.
  \end{enumerate}
  Welche Kirchentonart erhält man jeweils?
\end{aufg}

\begin{aufg}
  Stapelt man sukzessive Quinten, so erreicht man jeden klingenden Ton. Welche
  anderen Intervalle könnte man anstatt Quinten nehmen? Welche nicht?
\end{aufg}

\newpage
\section{Stufen und Funktionen}

\begin{aufg}
  Bestimme folgende Funktionen:\vspace*{-.25\baselineskip}
  \begin{multicols}{2}
    \begin{enumerate}[itemsep=0em]
    \item D in d-Moll,
    \item tG in a-Moll,
    \item Sp in E-Dur,
    \item dP in c-Moll,
    \item D der D in F-Dur,
    \item s der sP in C-Dur.
    \end{enumerate}
  \end{multicols}
\end{aufg}
% \lipsum

\begin{aufg}
  Zwischen welchen zwei Tönen liegt der problematische Hiatus im Tonvorrat von
  g-Moll, c-Moll bzw.\ e-Moll?
\end{aufg}

\begin{aufg}
  Durch welche Akkordfolge ist eine Vollkadenz in folgenden Tonarten
  beschrieben:
  \begin{enumerate}[itemsep=0em]
  \item F-Dur,
  \item c-Moll,
  \item A-Dur.
  \end{enumerate}
  Wohin würde jeweils der Trugschluss führen?
\end{aufg}

\begin{aufg}
  Betrache folgenden vierstimmigen Satz:
  \begin{center}
    \includegraphics{ly/e-06-4}
  \end{center}
  \begin{enumerate}
  \item In welcher Tonart steht dieses Beispiel?
  \item Auf jedem Schlag ergeben die vier Stimmen einen Akkord im
    Kadenzrahmen. Bestimme die entsprechenden Funktionen.
  \end{enumerate}
\end{aufg}

\newpage
\section{Struktur von Notensätzen}
\begin{aufg}
  Ergänze die Mittelstimmen:%\vspace*{-5px}
  \begin{center}
    \includegraphics{ly/e-07-1}\\[-3px]
    \small \hspace*{17.8mm}%
    $\mut$\hspace*{3.7mm}%
    $\mus_3$\hspace*{1.3mm}%
    $\mud\muP$\hspace*{2.1mm}%
    $\mut\muP$\hspace*{4.3mm}%
    $\muD$\hspace*{2.1mm}%
    $\mut\muG$\hspace*{2.5mm}%
    $\muD$\hspace*{3.2mm}%
    $\muT$\hspace*{9.5mm}\vspace*{4pt}
  \end{center}
\end{aufg}

\begin{aufg}
  Vervollständige den Satz:%\vspace*{-5px}
  \begin{center}
    \includegraphics{ly/e-07-2}\\[-10px]
    \small\hspace*{22mm}%
    $\muT$\hspace*{2.5mm}%
    $\muD\mup$\hspace*{2.4mm}%
    $\muS$\hspace*{3.3mm}%
    $\muT$\hspace*{4.5mm}%
    $\muS\mup_3$\hspace*{.7mm}%
    $\muD$\hspace*{2.6mm}%
    $\muT$\hspace*{11.8mm}\vspace*{-4pt}
  \end{center}
\end{aufg}

\paragraph{Permanente Übung.}
~\kern-8px Setze die beiden folgenden Melodien aus und schreibe die verwendeten
Funktionen dazu. Alles bisher Bespro\-chene darf angewendet werden.%\vspace*{7px}
\begin{center}
  \includegraphics{ly/e-perm1}\\[5px]
  \includegraphics{ly/e-perm2}
\end{center}

\newpage
\section{Harmoniefremde Töne}

\begin{aufg}
  Benenne alle Vorhalte und Durchgänge:
  \begin{center}
    \includegraphics{ly/e-08-1}
  \end{center}
\end{aufg}

\begin{aufg}
  Ergänze Durchgänge und Wechselnoten:
  \begin{center}
    \includegraphics{ly/e-08-2}\vspace*{-5pt}
  \end{center}
\end{aufg}

\paragraph{Permanente Übung.}
~\kern-8px Setze die beiden folgenden Melodien aus und schreibe die verwendeten
Funktionen dazu. Alles bisher Bespro\-chene darf angewendet werden.%\vspace*{7px}
\begin{center}
  \includegraphics{ly/e-perm1}\\[5px]
  \includegraphics{ly/e-perm2}
\end{center}

\newpage
\section{Charakteristische Zusatztöne}

\begin{aufg}
  Bestimme die Töne folgender Vierklänge und gib ggf. an, wo ein Tritonus
  liegt:
  \begin{enumerate}[itemsep=0em]
  \item $\muD^7$ in f-Moll,
  \item $\muS^{^6_5}$ in Es-Dur,
  \item $\mus^{^6_5}$ in fis-Moll.
  \end{enumerate}
\end{aufg}

\begin{aufg}
  Benenne die Funktionen und markiere alle charakteristischen Dissonanzen:%\vspace*{-10px}
  \begin{center}
    \hspace*{-3px}
    \includegraphics{ly/e-09-2}
    \vspace*{-5pt}
  \end{center}
\end{aufg}

\paragraph{Permanente Übung.}
~\kern-8px Setze die beiden folgenden Melodien aus und schreibe die verwendeten
Funktionen dazu. Alles bisher Bespro\-chene darf angewendet werden.%\vspace*{7px}
\begin{center}
  \includegraphics{ly/e-perm1}\\
  \includegraphics{ly/e-perm2}
\end{center}

\newpage
\section{Erweiterung des Kadenzraums}

\begin{aufg}
  Bestimme die Töne folgender Akkorde:\vspace*{-.25\baselineskip}
  \begin{multicols}{2}
    \begin{enumerate}[itemsep=0em]
    \item die $\muDD^7$ in f-Moll,
    \item die $\muD$ der $\muS$ in B-Dur,
    \item die $\mus\kern.3px{}^{^6_5}$ der $\muD\mup$ in As-Dur,
    \item die Dp der \mbox{$\muDD$} in F-Dur,\vspace*{-4px}
    \item die $\muD\kern.3px{}^7$ der $\mus$ in a-Moll,\vspace*{-4px}
    \item die $\muDD\kern.3px{}^7$ der $\muS\mup$ in E-Dur.\vspace*{2pt}
    \end{enumerate}
  \end{multicols}
\end{aufg}

\begin{aufg}
  Mit hinreichend vielen Zwischendominanten (und einer Doppel\emph{sub}dominante,
  die wir mit $\muSS$ bezeichnen) können wir sogar eine chromatische Tonleiter
  sinnvoll aussetzen. Ergänze:
  \begin{center}
    \hspace*{-2px}
    \includegraphics{ly/e-10-2}\\[-6px]
    \small \hspace*{38.2mm}$\muSS$\hspace*{-30mm}
  \end{center}
\end{aufg}

\begin{aufg}
  Schreibe einen Satz in einer Durtonart Deiner Wahl,
  der mit möglichst geringem Bewegungsaufwand folgende
  Akkordfolge realisiert: \smash{$\mut$~$(\muS\kern.3px{}_{\smash{{}_3}}^{^6_5}~\muD\kern.3px{}^7)$~$\mut\muP$~$(\muD\kern.3px{}_3^7)$~$\mus\kern.3px{}^{^6_5}$~$\muDD\kern.3px{}_3^7$~$\muD\kern.3px{}^{4-3}$~$\mut$.}
  \begin{center}
    \hspace*{-2px}
    \includegraphics{ly/e-10-3}
  \end{center}
\end{aufg}

\newpage
\section{Dominanten mit Nonen}

\begin{aufg}
  Analysiere folgenden (überladenen) Satz:
  \begin{center}
    \includegraphics{ly/e-11-1}
  \end{center}
\end{aufg}

\begin{aufg}
  Schreibe einen Satz in einer Durtonart, der folgende Akkordfolge
  realisiert:
  \smash{$\muT$~$\muD\kern.3px{}^7_3$~$(\muD\kern.3px{}^v_7)$~$\muS\kern.3px{}^{^6_5}_{\smash{{}_3}}$~$\muDD\kern.3px{}^v_{5-}$~$\muD\kern.3px^{^{6-5}_{4-3}}$~$(\muD\kern.3px{}^v_3)$~$\muT\mup$~$\muDD\kern.3px{}^{^9_7}_{\smash{{}_3}}$~$\muD^{4-3}$~$\muT$.}
  \begin{center}
    \hspace*{-2px}\includegraphics{ly/e-11-2}\vspace*{-7pt}
  \end{center}
\end{aufg}

\paragraph{Permanente Übung.}
~\kern-8px Setze die beiden folgenden Melodien aus und schreibe die verwendeten
Funktionen dazu. Alles bisher Bespro\-chene darf angewendet werden.%\vspace*{7px}
\begin{center}
  \includegraphics{ly/e-perm1}\\[5px]
  \includegraphics{ly/e-perm2}
\end{center}

\end{document}

%%% Local variables:
%%% TeX-engine: default
%%% End:
