%% Serifen in Überschriften
\KOMAoptions{paper=b5,numbers=enddot,DIV=11,headings=standardclasses}

%% Encoding
\usepackage[utf8]{inputenc}
\usepackage[T1]{fontenc}

%% Sprache
\usepackage{babel}
\usepackage[cleanlook]{isodate}

%% Schriftpakete
\usepackage{lmodern}
\usepackage[osf,sc]{mathpazo}
\usepackage{classico}
\usepackage[scaled=.85]{beramono}

%% Palatino braucht mehr Platz
\usepackage{setspace}
\setstretch{1.07}
\renewcommand{\arraystretch}{1.07} % Für tabular- und array-Umgebungen

%% Mikrotypographie!
\usepackage{ellipsis}
\usepackage[babel,tracking=true]{microtype}
\UseMicrotypeSet[tracking]{smallcaps}
\SetTracking{encoding=*,shape=sc}{30}

%% Akronyme
\newcommand{\acr}[1]{\textls[40]{\textsc{\MakeLowercase{#1}}}}

%% Eigene Listing styles mit kurzer Definition
\usepackage[shortlabels]{enumitem}

%% Captions for figures and tables
\setkomafont{captionlabel}{\footnotesize\bfseries}
\setkomafont{caption}     {\footnotesize}
\setcapwidth{.9\textwidth}
\renewcommand*{\captionformat}{.\ }
\setcapindent{0em}
\addtokomafont{caption}{\setstretch{1.07}}

%% Fancy tables
\usepackage{booktabs}

%% Mehrspaltinge Aufzählungen
\usepackage{multicol}

%% Bibliographie
\usepackage{biblatex}
\bibliography{bibliography.bib}

%% Footnote-Design
\deffootnote{1.5em}{1.5em}{\thefootnotemark.\ }
\addtokomafont{footnote}{\setstretch{1.07}}

%% Todo
\usepackage[size=tiny]{todonotes}

%% Grundlegende mathematische Symbole
\usepackage{amssymb,amsmath} % Standard
\usepackage{mathtools}       % \coloneqq, \eqqcolon

%% TikZ and TikZ-CD
\usepackage{tikz-cd}
\usepackage{tikz}
\usetikzlibrary{arrows,calc}

%% Mini-Indizes (\scaleto{…}{3pt})
\usepackage{scalerel}

%% Bold
\newcommand{\bm} [1]{\mathbold{#1}}

%% TikZ: centerarc
 \def\centerarc[#1](#2)(#3:#4:#5)% Syntax: [draw options] (center) (initial angle:final angle:radius)
     { \draw[#1] ($(#2)+({#5*cos(#3)},{#5*sin(#3)})$) arc (#3:#4:#5); }

%% Musikalische Symbole
\usepackage{musicography}

%% Harmonische Funktionen
\newcommand{\mut}         {\text{t}}
\newcommand{\mus}         {\text{s}}
\newcommand{\mud}         {\text{d}}
\newcommand{\muT}         {\text{T}}
\newcommand{\muS}         {\text{S}}
\newcommand{\muD}         {\text{D}}
\newcommand{\mup}         {\text{p}}
\newcommand{\muP}         {\text{P}}
\newcommand{\mug}         {\text{g}}
\newcommand{\muG}         {\text{G}}
\newcommand{\muDD}        {\mbox{\muD\hspace{-5.5px}\raisebox{1.3px}{\muD}}}
\newcommand{\muDDZ}       {\mbox{\muD\hspace{-6.1px}\raisebox{1.3px}{\muD}}}
\newcommand{\muSS}        {\raisebox{1.5px}{\muS}\hspace{-4px}\muS}
\newcommand{\muss}        {\raisebox{1.5px}{\mus}\hspace{-3px}\mus}
\newcommand{\musDD}       {\muD\hspace{-6px}\raisebox{2px}{\sout{\muD}}}
\newcommand{\vermsept}    {\hspace{1mm}\raisebox{4px}{\mus}\hspace{-3px}\muD}
\newcommand{\sevsept}     {\hspace{1mm}\raisebox{3px}{\muS}\hspace{-5px}\muD}
\newcommand{\dvermsept}   {\hspace{1mm}\raisebox{4px}{\mut}\hspace{-1px}\muD\hspace{-6px}\raisebox{2px}{\muD}}

%% Hyperref and Cleveref
\usepackage{amsthm}
\usepackage[pdfusetitle,bookmarksnumbered=true]{hyperref}
\usepackage[noabbrev]{cleveref}

%% Aufgaben
\theoremstyle{definition}
\newtheorem{aufg}{Aufgabe}[section]

%\usepackage[dvipsnames]{xcolor}
%\newcommand{\rb}[1]{\textcolor{Maroon}{#1}}
\newcommand{\rb}[1]{#1}
